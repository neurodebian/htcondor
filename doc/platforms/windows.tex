%%%%%%%%%%%%%%%%%%%%%%%%%%%%%%%%%%%%%%%%%%%%%%%%%%%%%%%%%%%%%%%%%%%%%%
\section{\label{sec:platform-windows}Microsoft Windows}
%%%%%%%%%%%%%%%%%%%%%%%%%%%%%%%%%%%%%%%%%%%%%%%%%%%%%%%%%%%%%%%%%%%%%%
\index{platform-specific information!Windows|(}

Windows is a strategic platform for HTCondor,
and therefore we have been working toward a complete
port to Windows.
Our goal is to make HTCondor every bit as capable on Windows as it is on
Unix -- or even more capable.

Porting HTCondor from Unix to Windows is a formidable task,
because many
components of HTCondor must interact closely with the underlying operating
system.
Provided is a clipped version of HTCondor for Windows.
A clipped version is one in which there is no checkpointing
and there are no remote system calls.

This section contains additional information specific to running
HTCondor on Windows.
In order to effectively use HTCondor, first read the overview
chapter (section~\ref{sec:overview})
and the user's manual (section~\ref{sec:usermanual}).
If administrating or customizing the policy and set up of HTCondor,
also read the administrator's manual 
chapter (section~\ref{sec:Admin-Intro}).
After reading these chapters,
review the information in this chapter for
important information and differences when using and administrating
HTCondor on Windows.
For information on installing HTCondor for Windows, see
section~\ref{sec:Windows-Install}.



%%%%%%%%%%%%%%%%%%%%%%%%%%%%%%%%%%%%%%%%%%%%%%%%%%%%%%%%%%%%%%%%%%%%%%
\subsection{Limitations under Windows}
%%%%%%%%%%%%%%%%%%%%%%%%%%%%%%%%%%%%%%%%%%%%%%%%%%%%%%%%%%%%%%%%%%%%%%
\index{Windows!release notes}

In general, this release for Windows works the same as the 
release of HTCondor for Unix.
However, the following items are not supported in this version:

\begin{itemize}

\item The standard job universe is not present.  This means
transparent process checkpoint/migration and remote system calls are
not supported.

\item For \SubmitCmd{grid} universe jobs, the only supported grid type is
\SubmitCmd{condor}.

\item Accessing files via a network share that requires a Kerberos ticket
(such as AFS) is not yet supported.

\end{itemize}

%%%%%%%%%%%%%%%%%%%%%%%%%%%%%%%%%%%%%%%%%%%%%%%%%%%%%%%%%%%%%%%%%%%%%%
\subsection{Supported Features under Windows}
%%%%%%%%%%%%%%%%%%%%%%%%%%%%%%%%%%%%%%%%%%%%%%%%%%%%%%%%%%%%%%%%%%%%%%

Except for those items listed above, most everything works
the same way in HTCondor as it does in the Unix release.
This release is based on the HTCondor \VersionNotice\ source tree, and thus the
feature set is the same as HTCondor \VersionNotice\ for Unix.  
For instance, all of the following work in HTCondor:
\begin{itemize}

\item The ability to submit, run, and manage queues of jobs running on a
cluster of Windows machines.

\item All tools such as \Condor{q}, \Condor{status}, \Condor{userprio},
are included. Only \Condor{compile} is
\emph{not} included.

\item The ability to customize job policy using ClassAds.
The machine ClassAds contain all the information included in the Unix version,
including current load average, RAM and virtual memory sizes, integer and
floating-point performance, keyboard/mouse idle time, etc.  Likewise, job
ClassAds contain a full complement of information, including system
dependent entries such as dynamic updates of the job's image size and CPU
usage.

\item Everything necessary to run an HTCondor central manager on Windows.

\item Security mechanisms.

\item Support for SMP machines.

\item HTCondor for Windows can run jobs at a lower operating system
priority level.
Jobs can be suspended, soft-killed by using a WM\_CLOSE message,
or hard-killed automatically based upon policy expressions.
For example, HTCondor can automatically suspend a job
whenever keyboard/mouse or non-HTCondor created CPU activity is detected, and
continue the job after the machine has been idle for a specified amount
of time.

\item HTCondor correctly manages jobs which create multiple processes.  For
instance, if an HTCondor job spawns multiple processes and HTCondor
needs to kill the job,
all processes created by the job will be terminated.

\item In addition to interactive tools, users and administrators can receive
information from HTCondor by e-mail (standard SMTP) and/or by log files.

\item HTCondor includes a friendly GUI installation and set up program,
which can perform a full install or deinstall of HTCondor.
Information specified by the user in the set up program is stored in the
system registry.  
The set up program can update a current installation with a
new release using a minimal amount of effort.

\item HTCondor can give a job access to the running user's Registry hive.

\end{itemize}

%%%%%%%%%%%%%%%%%%%%%%%%%%%%%%%%%%%%%%%%%%%%%%%%%%%%%%%%%%%%%%%%%%%%%%
\subsection{\label{sec:windows-sps}Secure Password Storage}
%%%%%%%%%%%%%%%%%%%%%%%%%%%%%%%%%%%%%%%%%%%%%%%%%%%%%%%%%%%%%%%%%%%%%%

In order for HTCondor to operate properly, it must at times be able to
act on behalf of users who submit jobs.  
This is required on submit machines,
so that HTCondor can access a job's input files, 
create and access the job's output files, 
and write to the job's log file from within the appropriate security context.
On Unix systems, arbitrarily changing what user HTCondor performs its
actions as is easily done when HTCondor is started with root privileges.
On Windows, however, performing an action as a particular user
or on behalf of a particular user requires knowledge of that user's password,
even when running at the maximum privilege level.
HTCondor provides secure password storage through the use of the 
\Condor{store\_cred} tool.
Passwords managed by HTCondor are encrypted and stored 
in a secure location within the Windows registry.
When HTCondor needs to perform an action as or on behalf of a particular user,
it uses the securely stored password to do so.
This implies that a password is stored for every user that will
submit jobs from the Windows submit machine.

\index{HTCondor daemon!condor\_credd@\Condor{credd}}
\index{daemon!condor\_credd@\Condor{credd}}
\index{condor\_credd daemon}
\index{submit commands!run\_as\_owner}
A further feature permits HTCondor to execute the job itself under the
security context of its submitting user, specifying 
the \SubmitCmd{run\_as\_owner} command in the job's
submit description file. 
With this feature, 
it is necessary to configure and run a centralized \Condor{credd} daemon
to manage the secure password storage.
This makes each user's password available, via an encrypted
connection to the \Condor{credd}, to any execute machine that may need it.

By default, the secure password store for a submit machine when no
\Condor{credd} is running is managed by the \Condor{schedd}.
This approach works in environments where
the user's password is only needed on the submit machine.

%%%%%%%%%%%%%%%%%%%%%%%%%%%%%%%%%%%%%%%%%%%%%%%%%%%%%%%%%%%%%%%%%%%%%%
\subsection{\label{sec:windows-run-as-owner}Executing Jobs as the Submitting User}
%%%%%%%%%%%%%%%%%%%%%%%%%%%%%%%%%%%%%%%%%%%%%%%%%%%%%%%%%%%%%%%%%%%%%%
\index{submit commands!run\_as\_owner}

By default, HTCondor executes jobs on Windows using dedicated run
accounts that have minimal access rights and privileges,
and which are recreated for each new job.
As an alternative, HTCondor can be configured to allow users to run jobs using
their Windows login accounts. 
This may be useful if jobs need access to files on a network share,
or to other resources that are not available to the low-privilege run account.

This feature requires use of a \Condor{credd} daemon for secure
password storage and retrieval. 
With the \Condor{credd} daemon running,
the user's password must be stored, using the \Condor{store\_cred} tool.
Then,
a user that wants a job to run using their own account
places into the job's submit description file
\begin{verbatim}
  run_as_owner = True
\end{verbatim}

%%%%%%%%%%%%%%%%%%%%%%%%%%%%%%%%%%%%%%%%%%%%%%%%%%%%%%%%%%%%%%%%%%%%%%
\subsection{\label{sec:credd}The condor\_credd Daemon}
%%%%%%%%%%%%%%%%%%%%%%%%%%%%%%%%%%%%%%%%%%%%%%%%%%%%%%%%%%%%%%%%%%%%%%
\index{HTCondor daemon!condor\_credd@\Condor{credd}}
\index{daemon!condor\_credd@\Condor{credd}}
\index{condor\_credd daemon}
The \Condor{credd} daemon manages secure password storage.
A single running instance of the \Condor{credd} within an HTCondor pool
is necessary in order to provide the feature described in 
section \ref{sec:windows-run-as-owner},
where a job runs as the submitting user, 
instead of as a temporary user that has strictly limited access capabilities.

It is first necessary to select
the single machine on which to run the \Condor{credd}. 
Often, the machine acting as the pool's central manager is a good choice.
An important restriction, however, is that the \Condor{credd} host must be a
machine running Windows.

All configuration settings necessary to enable the \Condor{credd} are
contained in the example file \verb|etc\condor_config.local.credd|
from the HTCondor distribution. Copy these settings into a local
configuration file for the machine that will run the \Condor{credd}.
Run \Code{condor\_restart} for these new settings to take effect, then
verify (via Task Manager) that a \Condor{credd} process is running.

A second set of configuration variables specify security for the
communication among HTCondor daemons.
These variables must be set for all machines in the pool.
The following example settings are in the comments contained in the
\verb|etc\condor_config.local.credd| example file.
These sample settings rely on the \Expr{PASSWORD} method for
authentication among daemons, 
including communication with the \Condor{credd} daemon.
The \Macro{LOCAL\_CREDD} variable must be customized to point to the machine
hosting the \Condor{credd} and the \Macro{ALLOW\_CONFIG} variable
will be customized, if needed, to refer to an administrative account 
that exists on all HTCondor nodes. 
\begin{verbatim}
CREDD_HOST = credd.cs.wisc.edu
CREDD_CACHE_LOCALLY = True

STARTER_ALLOW_RUNAS_OWNER = True

ALLOW_CONFIG = Administrator@*
SEC_CLIENT_AUTHENTICATION_METHODS = NTSSPI, PASSWORD
SEC_CONFIG_NEGOTIATION = REQUIRED
SEC_CONFIG_AUTHENTICATION = REQUIRED
SEC_CONFIG_ENCRYPTION = REQUIRED
SEC_CONFIG_INTEGRITY = REQUIRED
\end{verbatim}

In order for \Expr{PASSWORD} authenticated communication to work,
a \Term{pool password} must be chosen and distributed.
The chosen pool password must be stored identically for each machine.
The pool password first should be
stored on the \Condor{credd} host, then on the other machines in the pool.

To store the pool password on a Windows machine, run
\begin{verbatim}
  condor_store_cred add -c
\end{verbatim}
when logged in with the administrative account on that machine,
and enter the password when prompted. 
If the administrative account is shared across all machines,
that is if it is a domain account or has the same password on all machines,
logging in separately to each machine in the pool can be avoided.
Instead, the pool password can be securely pushed out for each Windows machine
using a command of the form
\begin{verbatim}
  condor_store_cred add -c -n exec01.cs.wisc.edu
\end{verbatim}

Once the pool password is distributed, 
but before submitting jobs,
all machines must reevaluate their configuration,
so execute
\begin{verbatim}
  condor_reconfig -all
\end{verbatim}
from the central manager.
This will cause each execute machine to test its ability
to authenticate with the \Condor{credd}.
To see whether this test worked for each machine in the pool, run the command
\footnotesize
\begin{verbatim}
  condor_status -f "%s\t" Name -f "%s\n" ifThenElse(isUndefined(LocalCredd),\"UNDEF\",LocalCredd)
\end{verbatim}
\normalsize
Any rows in the output with the \Expr{UNDEF} string indicate machines where
secure communication is not working properly. Verify that the pool password
is stored correctly on these machines.

%%%%%%%%%%%%%%%%%%%%%%%%%%%%%%%%%%%%%%%%%%%%%%%%%%%%%%%%%%%%%%%%%%%%%%
\subsection{\label{sec:windows-load-profile}Executing Jobs with the User's Profile Loaded}
%%%%%%%%%%%%%%%%%%%%%%%%%%%%%%%%%%%%%%%%%%%%%%%%%%%%%%%%%%%%%%%%%%%%%%
\index{Windows!loading account profile}
HTCondor can be configured when using dedicated run accounts, 
to load the account's profile.  A user's profile includes a set of personal 
directories and a registry hive loaded under \texttt{HKEY\_CURRENT\_USER}.

This may be useful if the job requires direct access to the user's registry 
entries.
It also may be useful when the job requires an application, 
and the application requires registry access. 
This feature is always enabled on the \Condor{startd}, 
but it is limited to the dedicated run account.
For security reasons, the profiles are
removed after the job has completed and exited.  
This ensures 
that malicious jobs cannot discover what any previous job has done, nor 
sabotage the registry for future jobs. It also ensures the next job has 
a fresh registry hive.

A user that then wants a job to run with a profile uses the
\SubmitCmd{load\_profile} command in the job's submit description file:
\begin{verbatim}
load_profile = True
\end{verbatim}

This feature is currently not compatible with \SubmitCmd{run\_as\_owner}, 
and will be ignored if both are specified.

%%%%%%%%%%%%%%%%%%%%%%%%%%%%%%%%%%%%%%%%%%%%%%%%%%%%%%%%%%%%%%%%%%%%%%
\subsection{\label{sec:windows-scripts-as-executables}Using Windows Scripts as Job Executables}
%%%%%%%%%%%%%%%%%%%%%%%%%%%%%%%%%%%%%%%%%%%%%%%%%%%%%%%%%%%%%%%%%%%%%%

HTCondor has added support for scripting jobs on Windows.
Previously, HTCondor jobs on Windows were limited to executables or batch files.
With this new support,
HTCondor determines how to interpret the script using 
the file name's extension.
Without a file name extension, 
the file will be treated as it has been in the past:
as a Windows executable.

This feature may not require any modifications to HTCondor's configuration.  
An example that does not require administrative intervention
are Perl scripts using \Prog{ActivePerl}.

\Prog{Windows Scripting Host} scripts do require
configuration to work correctly.
The configuration variables set values to be used in registry look up,
which results in a command that invokes the correct interpreter,
with the correct command line arguments for the specific scripting
language.
In Microsoft nomenclature, 
\Term{verbs} are actions that can be taken upon a given a file.
The familiar examples of
\textbf{Open}, \textbf{Print}, and \textbf{Edit},
can be found on the context menu when a user right clicks on a file.
The command lines to be used for each of these verbs are stored in
the registry under the \Code{HKEY\_CLASSES\_ROOT} hive.
In general, a registry look up uses the form:

\footnotesize
\begin{verbatim}
HKEY_CLASSES_ROOT\<FileType>\Shell\<OpenVerb>\Command
\end{verbatim}
\normalsize

Within this specification, 
\verb@<FileType>@ is the name of a file type
(and therefore a scripting language),
and is obtained from the file name extension.
\verb@<OpenVerb>@ identifies the verb,
and is obtained from the HTCondor configuration.

The HTCondor configuration sets the selection of a verb,
to aid in the registry look up.
The file name extension sets the name of the HTCondor configuration variable.
This variable name is of the form:
\begin{verbatim}
OPEN_VERB_FOR_<EXT>_FILES
\end{verbatim}
\verb@<EXT>@ represents the file name extension.
The following configuration example uses the \verb@Open2@ verb for
a \Prog{Windows Scripting Host} registry look up for several scripting
languages:

\begin{verbatim}
OPEN_VERB_FOR_JS_FILES  = Open2
OPEN_VERB_FOR_VBS_FILES = Open2
OPEN_VERB_FOR_VBE_FILES = Open2
OPEN_VERB_FOR_JSE_FILES = Open2
OPEN_VERB_FOR_WSF_FILES = Open2
OPEN_VERB_FOR_WSH_FILES = Open2
\end{verbatim}

In this example, HTCondor specifies the \verb@Open2@ verb,
instead of the default \verb@Open@ verb,
for a script with the file name extension of \verb@wsh@.
The \Prog{Windows Scripting Host}'s \verb@Open2@ verb allows standard input,
standard output, and standard error to be redirected
as needed for HTCondor jobs.

A common difficulty is encountered when
a script interpreter requires access to the user's registry.
Note that the user's registry is different than the root registry.
If not given access to the user's registry,
some scripts, such as \Prog{Windows Scripting Host} scripts,
will fail.
The failure error message appears as: 

\begin{verbatim}
CScript Error: Loading your settings failed. (Access is denied.)
\end{verbatim}

The fix for this error is to give explicit access to the submitting
user's registry hive.  This can be accomplished with the addition of
the \SubmitCmd{load\_profile} command in the job's submit description
file:

\begin{verbatim}
load_profile = True
\end{verbatim}

With this command,
there should be no registry access errors.
This command should also work for other interpreters.
Note that not all interpreters will require access.
For example,
\Prog{ActivePerl} does not by default require access to the user's
registry hive.

%%%%%%%%%%%%%%%%%%%%%%%%%%%%%%%%%%%%%%%%%%%%%%%%%%%%%%%%%%%%%%%%%%%%%%
\subsection{How HTCondor for Windows Starts and Stops a Job}
%%%%%%%%%%%%%%%%%%%%%%%%%%%%%%%%%%%%%%%%%%%%%%%%%%%%%%%%%%%%%%%%%%%%%%
\index{platform-specific information!Windows!starting and stopping a job}

This section provides some details on how HTCondor starts and stops jobs.
This discussion is geared for the HTCondor administrator or advanced user who is
already familiar with the material in the Administrator's Manual
and wishes to know detailed information on what HTCondor does when
starting and stopping jobs.

When HTCondor is about to start a job, the \Condor{startd} on the execute
machine spawns a \Condor{starter} process.  The \Condor{starter} then
creates:
\begin{enumerate}

\item a run account on the machine with a login name of
\Login{condor-reuse-slot<X>}, where \Expr{<X>} is the slot number of the
\Condor{starter}.  This account is added to group Users.  This step is
skipped if the job is to be run using the submitting user's account,
as specified in section \ref{sec:windows-run-as-owner}.

\item a new temporary working directory for the job on the execute machine.
This directory is
named \File{dir\_XXX}, where \Expr{XXX} is the process ID of the 
\Condor{starter}.
The directory is created in the \MacroUNI{EXECUTE} directory,
as specified in HTCondor's configuration file.  
HTCondor then grants write
permission to this directory for the user account newly created for the
job.

\item a new, non-visible Window Station and Desktop for the job.
Permissions are set so that only the account that will run the job has
access rights to this Desktop.  Any windows created by this job are
not seen by anyone; the job is run in the background.
Setting \Macro{USE\_VISIBLE\_DESKTOP} to \Expr{True}
will allow the job to access the
default desktop instead of a newly created one.

\end{enumerate}

Next, the \Condor{starter} daemon contacts the \Condor{shadow} daemon, 
which is running on the submitting machine, 
and the \Condor{starter} pulls over the job's executable and input files.
These files are placed into the temporary working directory
for the job.  After all files have been received, the \Condor{starter} spawns
the user's executable.  Its current working directory set to the
temporary working directory.

While the job is running, the \Condor{starter} closely monitors the CPU
usage and image size of all processes started by the job.
Every 20 minutes the \Condor{starter} sends this information,
along with the total size of all files contained in the job's
temporary working directory, to the \Condor{shadow}.
The \Condor{shadow} then
inserts this information into the job's ClassAd so that policy and
scheduling expressions can make use of this dynamic information.

If the job exits of its own accord (that is, the job completes),
the \Condor{starter}
first terminates any processes started by the job which could still be
around if the job did not clean up after itself.
The \Condor{starter} examines the job's temporary working directory for any
files which have been created or modified and sends these files back
to the \Condor{shadow} running on the submit machine.
The \Condor{shadow}
places these files into the \SubmitCmd{initialdir} specified in the
submit description file; 
if no \SubmitCmd{initialdir} was specified,
the files go into the directory where the user invoked \Condor{submit}.
Once all the output files are safely transferred back,
the job is removed from the queue.
If, however, the \Condor{startd} forcibly kills the job before all output files
could be transferred, the job is not removed from the queue but instead
switches back to the Idle state.  

If the \Condor{startd} decides to vacate a job prematurely,
the \Condor{starter} sends a WM\_CLOSE message to the job.
If the job spawned multiple child processes, the WM\_CLOSE message is only
sent to the parent process.
This is the one started by the \Condor{starter}.
The WM\_CLOSE message is the preferred way to terminate a process on Windows,
since this method allows the job to clean up and free any resources it may
have allocated.
When the job exits, the \Condor{starter} cleans up any processes left behind.
At this point, if \SubmitCmd{when\_to\_transfer\_output} is set to
\Expr{ON\_EXIT} (the default) in the job's submit description file,
the job switches states, from Running to Idle,
and no files are transferred back.
If \SubmitCmd{when\_to\_transfer\_output} is set to \Expr{ON\_EXIT\_OR\_EVICT},
then any files
in the job's temporary working directory which were changed or modified are
first sent back to the submitting machine.
But this time, the \Condor{shadow} places these
intermediate files into a subdirectory created in the
\MacroUNI{SPOOL} directory on the submitting machine.
The job is then switched back to the Idle state until HTCondor finds
a different machine on which to run.
When the job is started again,
HTCondor places into the job's temporary working directory the executable
and input files as before,
\emph{plus} any files stored in the submit machine's \MacroUNI{SPOOL} directory for that job.  

\Note A Windows console process can intercept a WM\_CLOSE message
via the Win32 \Procedure{SetConsoleCtrlHandler} function,
if it needs to do special cleanup work at vacate time; 
a WM\_CLOSE message generates a CTRL\_CLOSE\_EVENT.
See \Procedure{SetConsoleCtrlHandler} in the Win32
documentation for more info.

\Note The default handler in Windows for a WM\_CLOSE message is for the
process to exit.  Of course, the job could be coded to ignore it and not
exit, but eventually the \Condor{startd} will become impatient and hard-kill
the job, if that is the policy desired by the administrator.

Finally, after the job has left and any files transferred back, 
the \Condor{starter} deletes the temporary working directory, 
the temporary account if one was created,
the Window Station and the Desktop before exiting. 
If the \Condor{starter} should terminate abnormally, 
the \Condor{startd} attempts the clean up.  
If for some reason the \Condor{startd} should disappear as well
(that is, if the entire machine was power-cycled hard),
the \Condor{startd} will clean up when HTCondor is restarted.

%%%%%%%%%%%%%%%%%%%%%%%%%%%%%%%%%%%%%%%%%%%%%%%%%%%%%%%%%%%%%%%%%%%%%%
\subsection{Security Considerations in HTCondor for Windows}
%%%%%%%%%%%%%%%%%%%%%%%%%%%%%%%%%%%%%%%%%%%%%%%%%%%%%%%%%%%%%%%%%%%%%%

% WRT the backslash character, extra spaces are added before it
% as viewed from the html generated.
%   Karen has tried
%         \File{C:$\backslash$WINNT}
%         \File{C:\Bs WINNT}
% and neither works.

On the execute machine (by default), the user job is run using the
access token of an account dynamically created by HTCondor which has
bare-bones access rights and privileges.  For instance, if your
machines are configured so that only Administrators have write access
to
%\File{C:\Bs WINNT},
\verb@C:\WINNT@, then certainly no HTCondor job run on that machine
would be able to write anything there.  The only files the job should
be able to access on the execute machine are files accessible by the
Users and Everyone groups, and files in the job's temporary working
directory.  Of course, if the job is configured to run using the
account of the submitting user (as described in section
\ref{sec:windows-run-as-owner}), it will be able to do anything that
the user is able to do on the execute machine it runs on.

On the submit machine, HTCondor impersonates the submitting user, therefore
the File Transfer mechanism has the same access rights as the submitting
user.  For example, say only Administrators can write to
%\File{C:\Bs WINNT}
\verb@C:\WINNT@
on the submit machine,
and a user gives the following to \Condor{submit} :
\begin{verbatim}
         executable = mytrojan.exe
         initialdir = c:\winnt
         output = explorer.exe
         queue
\end{verbatim}
Unless that user is in group Administrators, HTCondor will not permit
\File{explorer.exe} to be overwritten.  

If for some reason the submitting user's account disappears between the time
\Condor{submit} was run and when the job runs, HTCondor is not able to check
and see if the now-defunct submitting user has read/write access to a given
file.  In this case, HTCondor will ensure that group ``Everyone'' has read or
write access to any file the job subsequently tries to read or write.  This
is in consideration for some network setups, where the user account only
exists for as long as the user is logged in.

HTCondor also provides protection to the job queue.  It would be bad if the
integrity of the job queue is compromised, because a malicious user could
remove other user's jobs or even change what executable a user's job will
run.  To guard against this, in HTCondor's default configuration all connections to the \Condor{schedd} (the
process which manages the job queue on a given machine) are authenticated
using Windows' eSSPI security layer.  The user is then authenticated
using the same challenge-response protocol that Windows uses to authenticate
users to Windows file servers.  Once authenticated, the only users
allowed to edit job entry in the queue are:
\begin{enumerate}
\item the user who originally submitted that job (i.e. HTCondor allows users
to remove or edit their own jobs)
\item users listed in the \File{condor\_config} file parameter
\MacroNI{QUEUE\_SUPER\_USERS}.  In the default configuration, only the
``SYSTEM'' (LocalSystem) account is listed here.  
\end{enumerate}
\Warn Do not remove ``SYSTEM'' from \MacroNI{QUEUE\_SUPER\_USERS}, or
HTCondor itself will not be able to access the job queue when needed.  If the
LocalSystem account on your machine is compromised, you have all sorts of
problems!

To protect the actual job queue files themselves, the HTCondor installation
program will automatically set permissions on the entire HTCondor release
directory so that only Administrators have write access.

Finally, HTCondor has all the IP/Host-based security mechanisms present
in the full-blown version of HTCondor.  See section~\ref{sec:Host-Security}
starting on page~\pageref{sec:Host-Security} for complete information
on how to allow/deny access to HTCondor based upon machine host name or
IP address.


%%%%%%%%%%%%%%%%%%%%%%%%%%%%%%%%%%%%%%%%%%%%%%%%%%%%%%%%%%%%%%%%%%%%%%
\subsection{\label{sec:network-files-solutions}Network files and HTCondor}
%%%%%%%%%%%%%%%%%%%%%%%%%%%%%%%%%%%%%%%%%%%%%%%%%%%%%%%%%%%%%%%%%%%%%%

HTCondor can work well with a network file server.  The recommended
approach to having jobs access files on network shares is to configure
jobs to run using the security context of the submitting user (see
section \ref{sec:windows-run-as-owner}).  If this is done, the job
will be able to access resources on the network in the same way as the
user can when logged in interactively.

In some environments, running jobs as their submitting users is not a
feasible option.  This section outlines some possible
alternatives. The heart of the difficulty in this case is that on the
execute machine, HTCondor creates a temporary user that will run the
job.  The file server has never heard of this user before.

Choose one of these methods to make it work:

\begin{itemize}
\item METHOD A: access the file server as a different user via a net use command
with a login and password
\item METHOD B: access the file server as guest
\item METHOD C: access the file server with a "NULL" descriptor
\item METHOD D: create and have HTCondor use a special account 
\end{itemize}

All of these methods have advantages and disadvantages.

Here are the methods in more detail:

METHOD A - access the file server as a different user via a net use command 
with a login and password

Example: you want to copy a file off of a server before running it....

\footnotesize
\begin{verbatim}
   @echo off
   net use \\myserver\someshare MYPASSWORD /USER:MYLOGIN
   copy \\myserver\someshare\my-program.exe
   my-program.exe
\end{verbatim}
\normalsize

The idea here is to simply authenticate to the file server with a different 
login than the temporary HTCondor login.  This is easy with the "net use" 
command as shown above.  Of course, the obvious disadvantage is this user's 
password is stored and transferred as clear text.

METHOD B - access the file server as guest

Example: you want to copy a file off of a server before running it as GUEST

\begin{verbatim}
   @echo off
   net use \\myserver\someshare
   copy \\myserver\someshare\my-program.exe
   my-program.exe
\end{verbatim}

In this example, you'd contact the server MYSERVER as the HTCondor temporary 
user.  However, if you have the GUEST account enabled on MYSERVER, you will 
be authenticated to the server as user "GUEST".  If your file permissions 
(ACLs) are setup so that either user GUEST (or group EVERYONE) has access 
the share "someshare" and the directories/files that live there, you can 
use this method.  The downside of this method is you need to enable the 
GUEST account on your file server.   \Warn This should be done *with 
extreme caution* and only if your file server is well protected behind a 
firewall that blocks SMB traffic.

METHOD C - access the file server with a "NULL" descriptor

One more option is to use NULL Security Descriptors.  In this way, you
can specify which shares are accessible by NULL Descriptor by adding
them to your registry.  You can then use the batch file wrapper like:

\begin{verbatim}
net use z: \\myserver\someshare /USER:""
z:\my-program.exe
\end{verbatim}

so long as 'someshare' is in the list of allowed NULL session shares.  To
edit this list, run regedit.exe and navigate to the key:

\begin{verbatim}
HKEY_LOCAL_MACHINE\
   SYSTEM\
     CurrentControlSet\
       Services\
         LanmanServer\
           Parameters\
             NullSessionShares
\end{verbatim}

and edit it.  unfortunately it is a binary value, so you'll then need to
type in the hex ASCII codes to spell out your share.  each share is
separated by a null (0x00) and the last in the list is terminated with
two nulls.

although a little more difficult to set up, this method of sharing is a
relatively safe way to have one quasi-public share without opening the
whole guest account.  you can control specifically which shares can be 
accessed or not via the registry value mentioned above.


METHOD D -  create and have HTCondor use a special account

Create a permanent account (called condor-guest in this description)
under which HTCondor will run jobs.
On all Windows machines, and on the file server, create the
condor-guest account.

On the network file server, give the condor-guest user permissions
to access files needed to run HTCondor jobs.

Securely store the password of the condor-guest user in the
Windows registry using \Condor{store\_cred} on all Windows
machines.

Tell HTCondor to use the condor-guest user as the owner of jobs,
when required.
Details for this are in 
section~\ref{sec:RunAsNobody}.

%%%%%%%%%%%%%%%%%%%%%%%%%%%%%%%%%%%%%%%%%%%%%%%%%%%%%%%%%%%%%%%%%%%%%%
\subsection{Interoperability between HTCondor for Unix and HTCondor for Windows}
%%%%%%%%%%%%%%%%%%%%%%%%%%%%%%%%%%%%%%%%%%%%%%%%%%%%%%%%%%%%%%%%%%%%%%

Unix machines and Windows machines running HTCondor can happily
co-exist in the same HTCondor pool without any problems.
Jobs submitted on Windows can run on Windows or Unix,
and jobs submitted on Unix can run on Unix or Windows.
Without any specification
(using the \AdAttr{requirements} expression in the submit description file),
the default behavior will be to 
require the execute machine to be of the same architecture and operating
system as the submit machine.

There is absolutely no need to run more than one HTCondor central manager,
even if you have both Unix and Windows machines.  The HTCondor central manager
itself can run on either Unix or Windows; there is no advantage to choosing
one over the other.  Here at University of Wisconsin-Madison, for
instance, we have hundreds of Unix and
Windows machines in our Computer Science Department HTCondor pool.

%%%%%%%%%%%%%%%%%%%%%%%%%%%%%%%%%%%%%%%%%%%%%%%%%%%%%%%%%%%%%%%%%%%%%%
\subsection{Some differences between HTCondor for Unix -vs- HTCondor for Windows}
%%%%%%%%%%%%%%%%%%%%%%%%%%%%%%%%%%%%%%%%%%%%%%%%%%%%%%%%%%%%%%%%%%%%%%

\begin{itemize}

\item On Unix, we recommend the creation of a \Login{condor} account
when installing HTCondor.  On Windows, this is not necessary, as HTCondor is
designed to run as a system service as user LocalSystem.

\item On Unix, HTCondor finds the \File{condor\_config} main configuration
file by looking in \Tilde condor, in \File{/etc}, 
or via an environment variable.
On Windows, the location of \File{condor\_config} file is determined
via the registry key \File{HKEY\_LOCAL\_MACHINE/Software/Condor}.
Override this value by setting an environment variable named
\Env{CONDOR\_CONFIG}.

\item On Unix, in the vanilla universe at job vacate time,
HTCondor sends the
job a softkill signal defined in the submit description file,
which defaults to SIGTERM.
On Windows, HTCondor sends a WM\_CLOSE message to the job at vacate
time.

\item On Unix, if one of the HTCondor daemons has a fault, a core file
will be created in the \MacroUNI{Log} directory.  
On Windows, a core file will also be created, 
but instead of a memory dump of the process,
it will be a very short ASCII text file which describes what
fault occurred and where it happened.  This information can be used by
the HTCondor developers to fix the problem.

\end{itemize}

\index{platform-specific information!Windows|)}
