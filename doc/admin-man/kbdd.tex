%%%%%%%%%%%%%%%%%%%%%%%%%%%%%%%%%%%%%%%%%%%%%%%%%%%%%%%%%%%%%%%%%%%%%%
\subsection{\label{sec:kbdd}The \Condor{kbdd}}
%%%%%%%%%%%%%%%%%%%%%%%%%%%%%%%%%%%%%%%%%%%%%%%%%%%%%%%%%%%%%%%%%%%%%%
\index{HTCondor daemon!condor\_kbdd@\Condor{kbdd}}
\index{daemon!condor\_kbdd@\Condor{kbdd}}
\index{condor\_kbdd daemon}

The HTCondor keyboard daemon, \Condor{kbdd}, monitors X events on
machines where the operating system does not provide a way of
monitoring the idle time of the keyboard or mouse.  
On Linux platforms,
it is needed to detect USB keyboard activity.
Otherwise, it is not needed.  
On Windows platforms,
the \Condor{kbdd} is the primary way of monitoring the idle time of 
both the keyboard and mouse.

%%%%%%%%%%%%%%%%%%%%%%%%%%%%%%%%%%%%%%%%%%%%%%%%%%%%%%%%%%%%%%%%%%%%%%
\subsubsection{\label{sec:kbdd-Windows}The \Condor{kbdd} on Windows Platforms}
%%%%%%%%%%%%%%%%%%%%%%%%%%%%%%%%%%%%%%%%%%%%%%%%%%%%%%%%%%%%%%%%%%%%%%

Windows platforms need to use the \Condor{kbdd} to monitor the
idle time of both the keyboard and mouse.
By adding \MacroNI{KBDD} to configuration variable \MacroNI{DAEMON\_LIST},
the \Condor{master} daemon invokes the \Condor{kbdd},
which then does the right thing to monitor activity given the
version of Windows running.

With Windows Vista and more recent version of Windows,
user sessions are moved out of session 0.
Therefore, the \Condor{startd} service is no longer able to listen 
to keyboard and mouse events.
The \Condor{kbdd} will run in an invisible
window and should not be noticeable by the user,
except for a listing in the task manager. 
When the user logs out, the program is terminated by Windows. 
This implementation also appears in versions of Windows that predate Vista,
because it adds the capability of monitoring keyboard activity
from multiple users.

To achieve the auto-start with user login, the HTCondor installer adds a
\Condor{kbdd} entry to the registry key at
\verb|HKLM\Software\Microsoft\Windows\CurrentVersion\Run|. 
On 64-bit versions of Vista and more recent Windows versions,
the entry is actually placed in
\verb|HKLM\Software\Wow6432Node\Microsoft\Windows\CurrentVersion\Run|.  

In instances where the \Condor{kbdd} is unable to connect to the
\Condor{startd},
it is likely because an
exception was not properly added to the Windows firewall.

%%%%%%%%%%%%%%%%%%%%%%%%%%%%%%%%%%%%%%%%%%%%%%%%%%%%%%%%%%%%%%%%%%%%%%
\subsubsection{\label{sec:kbdd-Linux}The \Condor{kbdd} on Linux Platforms}
%%%%%%%%%%%%%%%%%%%%%%%%%%%%%%%%%%%%%%%%%%%%%%%%%%%%%%%%%%%%%%%%%%%%%%

On Linux platforms, great measures have been taken to make 
the \Condor{kbdd} as robust as possible, 
but the X window system was not designed to facilitate such a need,
and thus is not as efficient on machines where many users frequently log
in and out on the console.

In order to work with X authority, 
which is the system by which X authorizes processes to connect to X servers,
the \Condor{kbdd} needs to run with super user privileges.
Currently, the \Condor{kbdd} assumes that X
uses the \Env{HOME} environment variable in order to locate a file
named \File{.Xauthority}. 
This file contains keys necessary to connect to an X server.
The keyboard daemon attempts to set \Env{HOME}
to various users' home directories in order to gain a
connection to the X server and monitor events.
This may fail to work if the
keyboard daemon is not allowed to attach to the X server,
and the state of a machine may be incorrectly set to idle when a user is,
in fact,
using the machine.

In some environments, the  \Condor{kbdd} will not be able to connect to the X
server because the user currently logged into the system keeps their
authentication token for using the X server in a place that no local user on
the current machine can get to.  
This may be the case for files on AFS, 
because the user's \File{.Xauthority} file is in an AFS home directory.

There may also
be cases where the \Condor{kbdd} may not be run with super user privileges
because of political reasons,
but it is still desired to be able to monitor X activity.
In these cases, change the XDM configuration in order to
start up the \Condor{kbdd} with the permissions of the logged in user.
If running X11R6.3, 
the files to edit will probably be in \File{/usr/X11R6/lib/X11/xdm}.
The \File{.xsession}
file should start up the \Condor{kbdd} at the end,
and the \File{.Xreset} file
should shut down the \Condor{kbdd}.  
The \Opt{-l} option can be used to write the daemon's log file to a
place where the user running the daemon has permission to write a file.
The file's recommended location will be similar to
\File{\$HOME/.kbdd.log},
since this is a place where every
user can write, and the file will not get in the way.
The \Opt{-pidfile} and \Opt{-k}
options allow
for easy shut down of the \Condor{kbdd} by storing the process ID in a file.  
It will be necessary
to add lines to the XDM configuration similar to

\footnotesize
\begin{verbatim}
  condor_kbdd -l $HOME/.kbdd.log -pidfile $HOME/.kbdd.pid
\end{verbatim}
\normalsize

This will start the \Condor{kbdd} as the user who is currently logged in
and write the log to a file in the directory 
\File{\$HOME/.kbdd.log/}.  
This will also save the process ID of the daemon 
to \File{\Tilde/.kbdd.pid}, 
so that when the user logs out, XDM can do:

\footnotesize
\begin{verbatim}
  condor_kbdd -k $HOME/.kbdd.pid
\end{verbatim}
\normalsize

This will shut down the process recorded in file \File{\Tilde/.kbdd.pid} 
and exit.

To see how well the keyboard daemon is working, review
the log for the daemon and look for successful connections to the X
server.  If there are none, the \Condor{kbdd}
is unable to connect to the machine's X server.

