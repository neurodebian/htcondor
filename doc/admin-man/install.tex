%%%%%%%%%%%%%%%%%%%%%%%%%%%%%%%%%%%%%%%%%%%%%%%%%%%%%%%%%%%%%%%%%%%%%%
\section{\label{sec:install}Installation}
%%%%%%%%%%%%%%%%%%%%%%%%%%%%%%%%%%%%%%%%%%%%%%%%%%%%%%%%%%%%%%%%%%%%%%

This section contains the instructions for installing HTCondor.
The installation will have a default configuration that can
be customized.
Sections of the manual below explain customization.

Please read this \emph{entire} section before starting installation.

Please read the copyright and disclaimer information in
section~\ref{sec:license}.
Installation and
use of HTCondor is acknowledgment that you have read and agree to the
terms.

%%%%%%%%%%%%%%%%%%%%%%%%%%%%%%%%%%%%%%%%%%%%%%%%%%%%%%%%%%%%%%%%%%%%%%
\subsection{\label{sec:pre-install-procedure}
Obtaining HTCondor}
%%%%%%%%%%%%%%%%%%%%%%%%%%%%%%%%%%%%%%%%%%%%%%%%%%%%%%%%%%%%%%%%%%%%%%
\index{installation!download}
\index{Unix installation!download}
\index{download}
The first step to installing HTCondor is to download it from the HTCondor
web site, \URL{http://research.cs.wisc.edu/htcondor/}.
The downloads are available from the downloads page,
at \URL{http://research.cs.wisc.edu/htcondor/downloads/}.

The platform-dependent HTCondor files are currently available from two sites.
The main site is at the University of Wisconsin--Madison,
Madison, Wisconsin, USA.
A second site is the Istituto Nazionale di Fisica Nucleare Sezione di
Bologna, Bologna, Italy.
Please choose the site nearest to you.

Make note of the location of where you download the binary into.

The HTCondor binary distribution is packaged in the following files
and directories:

\begin{description}
\item[\File{LICENSE-2.0.TXT}] the licensing agreement.
                  By installing HTCondor, you agree to the contents of
		  this file
\item[\File{README}] general information
\item[\File{bin}] directory which contains the distribution HTCondor
		  user programs.
\item[\File{condor\_configure}] the Perl script used to install and
                  configure HTCondor.
\item[\File{condor\_install}] the Perl script used to install HTCondor.
\item[\File{etc}] directory which contains the distribution HTCondor
		  configuration data.
\item[\File{examples}] directory containing C, Fortran and C++ example
		  programs to run with HTCondor.
\item[\File{include}] directory containing HTCondor header files.
\item[\File{lib}] directory which contains the distribution HTCondor
		  libraries.
\item[\File{libexec}] directory which contains the distribution HTCondor
		  auxiliary programs for use internally by HTCondor.
\item[\File{man}] directory which contains the distribution HTCondor
		  manual pages.
\item[\File{sbin}] directory containing HTCondor daemon binaries and admin
		  tools.
\item[\File{src}] directory containing source for some interfaces.
\end{description}

Before you install, please consider joining the htcondor-world mailing
list.
Traffic on this list is kept to an absolute minimum;
it is only used to announce new releases of HTCondor.
To subscribe, go to
\URL{https://lists.cs.wisc.edu/mailman/listinfo/htcondor-world},
and fill out the online form.
%%%%%%%%%%%%%%%%%%%%%%%%%%%%%%%%%%%%%%%%%%%%%%%%%%%%%%%%%%%%%%%%%%%%%%
\subsection{\label{sec:Preparing-to-Install}Preparation} 
%%%%%%%%%%%%%%%%%%%%%%%%%%%%%%%%%%%%%%%%%%%%%%%%%%%%%%%%%%%%%%%%%%%%%%

Before installation, make a few important
decisions about the basic layout of your pool.
The decisions answer the questions:

\begin{enumerate}
\item What machine will be the central manager?
\item What machines should be allowed to submit jobs?
\item Will HTCondor run as root or not?
\item Who will be administering HTCondor on the machines in your pool?
\item Will you have a Unix user named condor and will its home directory be
   shared? 
\item Where should the machine-specific directories for HTCondor go?
\item Where should the parts of the HTCondor system be installed? 
	\begin{itemize}
	\item Configuration files
	\item Release directory
		\begin{itemize}
		\item user binaries
		\item system binaries 
		\item \File{lib} directory
	  	\item \File{etc} directory
		\end{itemize}
	\item Documentation
	\end{itemize}
\item Am I using AFS?
\item Do I have enough disk space for HTCondor?
\end{enumerate}

\begin{description}

\item[1. What machine will be the central manager?]

One machine in your pool must be the central manager.
\index{central manager!installation issues}
Install HTCondor on this machine first.
This is the centralized information repository for the HTCondor pool,
and it is also the
machine that does match-making between available machines and
submitted jobs.
If the central manager machine crashes, any currently active
matches in the system will keep running, but no new matches will be
made.  Moreover, most HTCondor tools will stop working.  Because of the
importance of this machine for the proper functioning of HTCondor,
install the central manager on a machine that is likely to stay up all the
time, or on one that will be rebooted quickly if it does crash.

Also consider
network traffic and your network layout when choosing your central
manager.
All the daemons send updates (by default, every 5 minutes) to this machine.
Memory requirements for the central manager differ by the number of machines
in the pool:
a pool with up to about 100 machines will require approximately
25 Mbytes of memory for the central manager's tasks,
and a pool with about 1000 machines will require approximately
100 Mbytes of memory for the central manager's tasks.

A faster CPU will speed up matchmaking. 

\item[2. Which machines should be allowed to submit jobs?]

HTCondor can restrict the machines allowed to submit jobs.  Alternatively, 
it can allow any machine the network allows to connect to a submit machine
to submit jobs.  If the HTCondor pool is behind a firewall, and all machines
inside the firewall are trusted, the \Macro{ALLOW\_WRITE} configuration
entry can be set to */*.  Otherwise, it should be set to reflect
the set of machines permitted to submit jobs to this pool.
HTCondor tries to be secure by default: it is shipped with an invalid value
that allows no machine to connect and submit jobs.

\item[3. Will HTCondor run as root or not?]

\index{installation!running as root}
Start up the HTCondor daemons as the Unix user root.
Without this,
HTCondor can do very little to enforce security and policy
decisions.
You can install HTCondor as any user,
however there are both serious security and performance consequences.
Please see section~\ref{sec:Non-Root} on page~\pageref{sec:Non-Root}
in the manual for the details and ramifications of
running HTCondor as a Unix user other than root.

\item[4. Who will administer HTCondor?]

\index{HTCondor!Unix administrator}
\index{Unix administrator}
\index{Unix user!root}

% administrator is a person, not an account
% responsibilities of the administrator needed here
% 1. to install
% 2. to customize policies
% 3. to receive e-mail

Either root will be administering HTCondor directly, or someone else
will be acting as the HTCondor administrator.  If root has delegated
the responsibility to another person, keep in mind that as long as
HTCondor is started up as root, it should be clearly understood that
whoever has the ability to edit the condor configuration files can
effectively run arbitrary programs as root.

The HTCondor administrator will be regularly updating HTCondor by following
these instructions or by using the system-specific installation methods below.
The administrator
will also customize policies of the HTCondor submit and execute nodes.
This person will also receive information from HTCondor if 
something goes wrong with the pool, 
as described in the documentation of the \Macro{CONDOR\_ADMIN}
configuration variable.

\item[5. Will you have a Unix user named condor, and will its home
directory be shared?]

\index{Unix user!condor}
To simplify installation of HTCondor,
create a Unix user named condor on all machines in the pool.
The HTCondor daemons will create files
(such as the log files) owned by this user,
and the home directory can be used to specify the location of files
and directories needed by HTCondor.  The home directory of this user can
either be shared among all machines in your pool, or could be a
separate home directory on the local partition of each machine.  Both
approaches have advantages and disadvantages.  Having the directories
centralized can make administration easier, but also concentrates the
resource usage such that you potentially need a lot of space for a
single shared home directory.  See the section below on
machine-specific directories for more details.

Note that the user condor must not be an account into which a person
can log in.
If a person can log in as user condor, 
it permits a major security breach,
in that the user condor could submit jobs that run as any other user,
providing complete access to the user's data by the jobs. 
A standard way of not allowing log in to an account on Unix platforms
is to enter an invalid shell in the password file.

If you choose not to create a user named condor,
then you must specify either via the
\index{environment variables!CONDOR\_IDS@\texttt{CONDOR\_IDS}}
\index{CONDOR\_IDS@\texttt{CONDOR\_IDS}!environment variable}
\Env{CONDOR\_IDS} environment variable or the \Macro{CONDOR\_IDS}
config file setting which uid.gid pair should be used for
the ownership of various HTCondor files.  
See section~\ref{sec:uids} on UIDs in HTCondor on
page~\pageref{sec:uids} in the Administrator's Manual for details.

\item[6. Where should the machine-specific directories for
HTCondor go?]

HTCondor needs a few directories that are unique on every machine in
your pool.  These are 
\File{spool}, 
\File{log}, and 
\File{execute}.  Generally, all
three are subdirectories of a single machine specific directory called
the local directory (specified by the \Macro{LOCAL\_DIR} macro
in the configuration file).
Each should be owned by the user that HTCondor is to be run as.
\index{owner!of directories}

If you have a Unix user named \Login{condor} with a local home directory on each
machine, the \MacroNI{LOCAL\_DIR} could just be user \Login{condor}'s home
directory (\MacroNI{LOCAL\_DIR} = \MacroUNI{TILDE} in the 
configuration file).
If this user's home directory is shared among all machines in your
pool, you would want to create a directory for each host (named by
host name) for the local directory (for example, \MacroNI{LOCAL\_DIR} =
\MacroUNI{TILDE}/hosts/\MacroUNI{HOSTNAME}).  If you do not
have a \Login{condor} account on your machines, you can put these directories
wherever you'd like.
However, where to place the directories will require some
thought, as each one has its own resource needs:

\begin{description}
\index{Unix directory!\File{execute}}
\index{disk space requirement!\File{execute} directory}
\item[\File{execute}] This is the directory that acts as the current working
directory for any HTCondor jobs that run on a given execute machine.
The binary for the remote job is copied into this directory, so
there
must be enough space for it.  (HTCondor will not send a job to a
machine that does not have enough disk space to hold the initial
binary).  In addition, if the remote job dumps core for some reason,
it is first dumped to the execute directory before it is sent back to
the submit machine.  So, put the execute directory on
a partition with enough space to hold a possible core file from the
jobs submitted to your pool.

\index{Unix directory!\File{spool}}
\index{disk space requirement!\File{spool} directory}
\item[\File{spool}] The \File{spool} directory holds the job queue
and history files,
and the checkpoint files for all jobs submitted from a given machine.
As a result, disk space requirements for the \File{spool} directory
can be quite large,
particularly if users are submitting jobs with very large
executables or image sizes.
By using a checkpoint server
(see section~\ref{sec:Ckpt-Server} on Installing a Checkpoint Server on
page~\pageref{sec:Ckpt-Server} for details),
you can ease the disk
space requirements, since all checkpoint files are stored on the
server instead of the spool directories for each machine.  However,
the initial checkpoint files (the executables for all the clusters you
submit) are still stored in the spool directory, so you will need
%
% how much?!?
%
some space, even with a checkpoint server. The amount of space will 
depend on how many executables, and what size they are, that need to be stored
in the spool directory.

\index{Unix directory!\File{log}}
\index{disk space requirement!\File{log} directory}
\item[\File{log}] Each HTCondor daemon writes its own log file,
and each log file is placed
in the \File{log} directory.  You can specify what size you want these files
to grow to before they are rotated,
%
% rotated?  Maybe this is talking about wrapping around to
% overwrite the oldest entries first
%
so the disk space requirements of
the directory are configurable.
The larger the log files, the more
historical information they will hold if there is a problem, but the
more disk space they use up.  If you have a network file system
installed at your pool, you might want to place the log directories in
a shared location (such as \File{/usr/local/condor/logs/\$(HOSTNAME)}),
so that you can view the log files from all your machines in a single
location.  However, if you take this approach, you will have to
specify a local partition for the \File{lock} directory (see below).

\index{Unix directory!\File{lock}}
\item[\File{lock}] HTCondor uses a small number of lock files to synchronize
access to certain files that are shared between multiple daemons.
Because of problems encountered with file locking and network
file systems (particularly NFS), these lock files should be placed on a
local partition on each machine.  By default, they are placed in
the \File{log} directory.  If you place your \File{log}
directory on a network file system partition,
specify a local partition for the
lock files with the \Macro{LOCK} parameter in the configuration file (such as
\File{/var/lock/condor}).

\end{description}

\index{disk space requirement!HTCondor files}
Generally speaking, it is recommended that you do not put these directories
(except \File{lock}) on the same partition as \File{/var},
since if the partition
fills up, you will fill up \File{/var} as well. 
This will cause lots of
problems for your machines.  Ideally, you will have a separate partition
for the HTCondor directories. Then, the only consequence of filling up
the directories
will be HTCondor's malfunction, not your whole machine.

\item[7. Where should the parts of the HTCondor system be installed?]

	\begin{itemize}
	\item Configuration Files
	\item Release directory
		\begin{itemize}
		\item User Binaries
		\item System Binaries 
		\item \File{lib} Directory
	  	\item \File{etc} Directory
		\end{itemize}
	\item Documentation
	\end{itemize}

\label{sec:Config-File-Locations}
\begin{description}
\item[Configuration Files] There are a number of configuration files
that allow you
different levels of control over how HTCondor is configured at each
machine in your pool.  
The global configuration file is shared by all machines in the pool.
For ease of administration, this file should be located on a shared
file system, if possible.
In addition, there is a local
configuration file for each machine, where you can override settings in the
global file.  This allows you to have different daemons running,
different policies for when to start and stop HTCondor jobs, and so on.
You can also have configuration files specific to each platform in your pool.
See
section~\ref{sec:Multiple-Platforms} on
page~\pageref{sec:Multiple-Platforms} about Configuring HTCondor for
Multiple Platforms for details.

\index{configuration files!location}
In general, there are a number of places that HTCondor will look to find
its configuration files.  The first file it looks for is the global configuration
file.  These locations are searched in order until a configuration file is
found.  If none contain a valid configuration file, HTCondor will print an
error message and exit:
\begin{enumerate}
   \item File specified in the \Env{CONDOR\_CONFIG} environment variable
   \item \File{\$(HOME)/.condor/condor\_config}
   \item \File{/etc/condor/condor\_config}
   \item \File{/usr/local/etc/condor\_config}
   \item \File{\Tilde condor/condor\_config}
   \item Files in the directory specified by the configuration variable 
\Macro{LOCAL\_CONFIG\_DIR}.
\end{enumerate}

If you specify a file in the \Env{CONDOR\_CONFIG} environment variable
and there's a problem reading that file, HTCondor will print an error
message and exit right away, instead of continuing to search the other
options.
However, if no \Env{CONDOR\_CONFIG} environment variable is set,
HTCondor will search through the other options.

Next, HTCondor tries to load the local configuration file(s).
The only way to specify the local configuration file(s) is in the global configuration
file, with the \Macro{LOCAL\_CONFIG\_FILE} macro.  If that macro is not
set, no local configuration file is used.  This macro can be a list of files
or a single file.

If configured, HTCondor also uses the \Macro{LOCAL\_CONFIG\_DIR} setting get a
directory where a list of configuration files is held.  The files are searched
for configuration settings.

\item[Release Directory]

Every binary distribution contains a contains
five subdirectories: \File{bin}, \File{etc}, \File{lib}, \File{sbin},
and \File{libexec}. Wherever you
choose to install these five directories we call the release directory
(specified by the \Macro{RELEASE\_DIR} macro in the configuration file).
Each
release directory contains platform-dependent binaries and libraries,
so you will need to install a separate one for each kind of machine in
your pool.  For ease of administration, these directories should be
located on a shared file system, if possible.

\begin{itemize}
     \item User Binaries:

     All of the files in the \File{bin} directory are programs the end
     HTCondor users should expect to have in their path.  You could
     either put them in a well known location (such as
     \File{/usr/local/condor/bin}) which you have HTCondor users add to
     their \Env{PATH} environment variable, or copy those files
     directly into a well known place already in the user's PATHs (such as
     \File{/usr/local/bin}).  With the above examples, you could also
     leave the binaries in \File{/usr/local/condor/bin} and put in
     soft links from \File{/usr/local/bin} to point to each program.

     \item System Binaries:

     All of the files in the \File{sbin} directory are HTCondor daemons and
     agents, or programs that only the HTCondor administrator would need
     to run.  Therefore, add these programs only
     to the \Env{PATH} of the HTCondor administrator.

     \item Private HTCondor Binaries:

     All of the files in the \File{libexec} directory are HTCondor
     programs that should never be run by hand, but are only used
     internally by HTCondor. 

     \item \File{lib} Directory:

     The files in the \File{lib} directory are the HTCondor libraries that
     must be linked in with user jobs for all of HTCondor's
     checkpointing and migration features to be used.  \File{lib} also
     contains scripts used by the \Condor{compile} program to help
     re-link jobs with the HTCondor libraries.  These files should be
     placed in a location that is world-readable, but they do not need
     to be placed in anyone's \Env{PATH}.  The \Condor{compile} script checks
     the configuration file for the location of the \File{lib} directory.

     \item \File{etc} Directory:

     \File{etc} contains an \File{examples} subdirectory which holds various
     example configuration files and other files used for installing HTCondor.
     \File{etc} is the recommended location to keep the master copy of your
     configuration files.  You can put in soft links from one of the places
     mentioned above that HTCondor checks automatically to find its
     global configuration file. 
\end{itemize}

\item[Documentation]

The documentation provided with HTCondor is currently available in
     HTML, Postscript and PDF (Adobe Acrobat).  It can be locally installed
     wherever is customary at your site.  You can also find the HTCondor
     documentation on the web at:
     \URL{http://www.cs.wisc.edu/condor/manual}.

\end{description}

\item[7. Am I using AFS?]

If you are using AFS at your site, be sure to read the
section~\ref{sec:HTCondor-AFS} on page~\pageref{sec:HTCondor-AFS} in the
manual.
HTCondor does not currently have a way to authenticate itself to AFS.
A solution is not ready for
\VersionNotice.
This implies that you are probably not going to want
to have the \Macro{LOCAL\_DIR} for HTCondor on AFS.
However, you can
(and probably should) have the HTCondor \MacroNI{RELEASE\_DIR} on AFS, so
that you can share one copy of those files and upgrade them in a
centralized location.  You will also have to do something special if
you submit jobs to HTCondor from a directory on AFS.  Again, read manual
section~\ref{sec:HTCondor-AFS} for all the details.

\item[8. Do I have enough disk space for HTCondor?]

\index{disk space requirement!all versions}
The compressed downloads of HTCondor currently range from a low of about 12
Mbytes for 32-bit Debian 6.0 GNU/Linux to about 94 Mbytes for Windows.  The
compressed source code takes approximately 18 Mbytes.

In addition, you will need a lot of disk space in the local directory
of any machines that are submitting jobs to HTCondor.  See question 6
above for details on this.

\end{description}

%%%%%%%%%%%%%%%%%%%%%%%%%%%%%%%%%%%%%%%%%%%%%%%%%%%%%%%%%%%%%%%%%%%%%%
\subsection{\label{sec:new-install-procedure}
Newer Unix Installation Procedure}
%%%%%%%%%%%%%%%%%%%%%%%%%%%%%%%%%%%%%%%%%%%%%%%%%%%%%%%%%%%%%%%%%%%%%%

\index{installation!with \Condor{configure}}
\index{condor\_configure command}
The Perl script \Condor{configure} installs HTCondor.
Command-line arguments specify all needed information to this
script.  The script can be executed multiple times, to modify or further
set the configuration.  \Condor{configure} has been tested using Perl 5.003.
Use this or a more recent version of Perl.

After download, all the files are in a compressed, tar format.
They need to be untarred, as
\begin{verbatim}
  tar xzf completename.tar.gz
\end{verbatim}
After untarring, the directory will have the Perl scripts
\Condor{configure} and \Condor{install}, as well as 
\File{bin}, \File{etc}, \File{examples}, \File{include},
\File{lib}, \File{libexec}, \File{man}, \File{sbin},
\File{sql} and \File{src} subdirectories.

\Condor{configure} and \Condor{install} are the same program, but have
different default behaviors.  \Condor{install} is identical to
running 
\begin{verbatim}
  condor_configure --install=.
\end{verbatim}
\Condor{configure} and \Condor{install} work on the named directories.
As the names imply, \Condor{install} is used to
install HTCondor, whereas \Condor{configure} is used to modify the
configuration of an existing HTCondor install.

\Condor{configure} and \Condor{install} are completely command-line
driven and are not interactive.  Several command-line arguments are
always needed with \Condor{configure} and \Condor{install}.
The argument
\begin{verbatim}
  --install=/path/to/release
\end{verbatim}
specifies the path to the HTCondor release directories.
The default command-line argument for \Condor{install} is 
\begin{verbatim}
  --install=.
\end{verbatim}
The argument
\begin{verbatim}
  --install-dir=<directory>
\end{verbatim}
or
\begin{verbatim}
  --prefix=<directory>
\end{verbatim}
specifies the path to the install directory.

The argument
\begin{verbatim}
  --local-dir=<directory>
\end{verbatim}
specifies the path to the local directory.

The \verb@--@\Opt{type} option to \Condor{configure}
specifies one or more of the roles that a machine may take on
within the HTCondor pool: central manager, submit or execute.
These options are given in a comma separated list.
So, if a machine is both a submit and execute
machine, 
the proper command-line option is
\begin{verbatim}
  --type=submit,execute
\end{verbatim}

Install HTCondor on the central manager machine first.  If HTCondor
will run as root in this pool (Item 3 above), run \Condor{install} 
as root, and it will install and set the file permissions correctly.  
On the central manager machine, run \Condor{install} as follows.
\begin{verbatim}
% condor_install --prefix=~condor \
	--local-dir=/scratch/condor --type=manager
\end{verbatim}

To update the above HTCondor installation, for example, to also be
submit machine:
\begin{verbatim}
% condor_configure --prefix=~condor \
	--local-dir=/scratch/condor --type=manager,submit
\end{verbatim}

As in the above example, the central manager can also be a submit
point or an execute machine, but this is only recommended for very
small pools.  If this is the case, the 
\verb@--@\Opt{type}
option changes to
\Expr{manager,execute} or \Expr{manager,submit}  or 
\Expr{manager,submit,execute}.

After the central manager is installed, the execute and submit machines
should then be configured.  Decisions about whether to run HTCondor as root
should be consistent throughout the pool. For each machine in the pool,
run

\begin{verbatim}
% condor_install --prefix=~condor \
	--local-dir=/scratch/condor --type=execute,submit
\end{verbatim}

See the \Condor{configure} manual page in
section~\ref{man-condor-configure} on
page~\pageref{man-condor-configure} for details.


%%%%%%%%%%%%%%%%%%%%%%%%%%%%%%%%%%%%%%%%%%%%%%%%%%%%%%%%%%%%%%%%%%%%%%
\subsection{\label{installed-now-what}
Starting HTCondor Under Unix After Installation}
%%%%%%%%%%%%%%%%%%%%%%%%%%%%%%%%%%%%%%%%%%%%%%%%%%%%%%%%%%%%%%%%%%%%%%
\index{starting HTCondor!Unix platforms}

Now that HTCondor has been installed on the machine(s), there are a few
things to check before starting up HTCondor.

\begin{enumerate}
\item Read through the \Release{etc/condor\_config} file.  There are a
    lot of possible settings and you should at least take a look at
    the first two main sections to make sure everything looks okay.
    In particular, you might want to set up security for
    HTCondor.  See the section~\ref{sec:Config-Security} on
    page~\pageref{sec:Config-Security} to learn how to do this.

\item For Linux platforms, run the \Condor{kbdd} to monitor keyboard
    and mouse activity on all machines within the pool that will
    run a \Condor{startd}; these are machines that execute jobs.
    To do this, the subsystem \Expr{KBDD} will need to be added to
    the \MacroNI{DAEMON\_LIST} configuration variable definition.

    For Unix platforms other than Linux,
    HTCondor can monitor the activity of your mouse and keyboard,
    provided that you tell it where to look.  You do this with the
    \Macro{CONSOLE\_DEVICES} entry in the \condor{startd} section of
    the configuration file.  On most platforms, reasonable
    defaults are provided.
    For example, the default device for the mouse
    is 'mouse', since most installations have a soft link from
    \File{/dev/mouse} that points to the right device (such as
    \File{tty00} if you have a serial mouse, \File{psaux} if you have
    a PS/2 bus mouse, etc).  If you do not have a \File{/dev/mouse}
    link, you should either create one (you will be glad you did), or
    change the \MacroNI{CONSOLE\_DEVICES} entry in HTCondor's
    configuration file.
    This entry is a comma separated list, so you can have any
    devices in \File{/dev} count as 'console devices' and activity
    will be reported in the \condor{startd}'s ClassAd as
    \AdAttr{ConsoleIdleTime}.

\item  (Linux only) HTCondor needs to be able to find the \File{utmp} file.
    According to the Linux File System Standard, this file should be
    \File{/var/run/utmp}.  If HTCondor cannot find it there, it looks in
    \File{/var/adm/utmp}.  If it still cannot find it, it gives up.  So, if
    your Linux distribution places this file somewhere else, be sure to
    put a soft link from \File{/var/run/utmp} to point to the real location.

\end{enumerate}

To start up the HTCondor daemons, execute
\Release{sbin/condor\_master}.  This is the HTCondor master, whose
only job in life is to make sure the other HTCondor daemons are running.
The master keeps track of the daemons, restarts them if they crash,
and periodically checks to see if you have installed new binaries (and
if so, restarts the affected daemons).

If you are setting up your own pool, you should start HTCondor on your
central manager machine first.  If you have done a submit-only
installation and are adding machines to an existing pool,
the start order does not
matter.

To ensure that HTCondor is running, you can run either:
\begin{verbatim}
        ps -ef | egrep condor_
\end{verbatim}
or
\begin{verbatim}
        ps -aux | egrep condor_
\end{verbatim}
depending on your flavor of Unix.  
On a central manager machine that can submit jobs as well
as execute them, there will be processes for:
\begin{itemize}
	\item \condor{master}
	\item \condor{collector}
	\item \condor{negotiator}
	\item \condor{startd}
	\item \condor{schedd}
\end{itemize}
On a central manager machine that does not submit jobs nor
execute them, there will be processes for:
\begin{itemize}
	\item \condor{master}
	\item \condor{collector}
	\item \condor{negotiator}
\end{itemize}
For a machine that only submits jobs, there will be processes for:
\begin{itemize}
	\item \condor{master}
	\item \condor{schedd}
\end{itemize}
For a machine that only executes jobs, there will be processes for:
\begin{itemize}
	\item \condor{master}
	\item \condor{startd}
\end{itemize}

Once you are sure the HTCondor daemons are running, check to make sure
that they are communicating with each other.  You can run
\Condor{status} to get a one line summary of the status of each
machine in your pool.

Once you are sure HTCondor is working properly, you should add
\Condor{master} into your startup/bootup scripts (i.e. \File{/etc/rc} ) so
that your machine runs \Condor{master} upon bootup.  \Condor{master}
will then fire up the necessary HTCondor daemons whenever your machine
is rebooted.  

If your system uses System-V style init scripts, you can look in
\Release{etc/examples/condor.boot} for a script that can be used
to start and stop HTCondor automatically by init.  Normally, you would
install this script as \File{/etc/init.d/condor} and put in soft link from
various directories (for example, \File{/etc/rc2.d}) that point back to
\File{/etc/init.d/condor}.  The exact location of these scripts and links
will vary on different platforms.

If your system uses BSD style boot scripts, you probably have an
\File{/etc/rc.local} file.  Add a line to start up
\Release{sbin/condor\_master}.


Now that the HTCondor daemons are running, there are a few things you
can and should do:

\begin{enumerate}
\item (Optional) Do a full install for the \Condor{compile} script.
    \condor{compile} assists in linking jobs with the HTCondor libraries
    to take advantage of all of HTCondor's features.  As it is currently
    installed, it will work by placing it in front of any of the
    following commands that you would normally use to link your code:
    gcc, g++, g77, cc, acc, c89, CC, f77, fort77 and ld.  If you
    complete the full install, you will be able to use
    \condor{compile} with any command whatsoever, in particular, make.
    See section~\ref{sec:full-condor-compile} on
    page~\pageref{sec:full-condor-compile} in the manual for
    directions.

\item Try building and submitting some test jobs.  See
    \File{examples/README} for details.

\item If your site uses the AFS network file system, see
section~\ref{sec:HTCondor-AFS} on page~\pageref{sec:HTCondor-AFS} in the
manual.

\item We strongly recommend that you start up HTCondor (run the
\Condor{master} daemon) as user root.  If you must start HTCondor as
some user other than root, see section~\ref{sec:Non-Root} on
page~\pageref{sec:Non-Root}.

\end{enumerate}

%%%%%%%%%%%%%%%%%%%%%%%%%%%%%%%%%%%%%%%%%%%%%%%%%%%%%%%%%%%%%%%%%%%%%%
%%%%%%%%%%%%%%%%%%%%%%%%%%%%%%%%%%%%%%%%%%%%%%%%%%%%%%%%%%%%%%%%%%%%%%
\subsection{\label{sec:Windows-Install}Installation on Windows}
%%%%%%%%%%%%%%%%%%%%%%%%%%%%%%%%%%%%%%%%%%%%%%%%%%%%%%%%%%%%%%%%%%%%%%

\index{installation!Windows|(}
\index{Windows!installation|(}
This section contains the instructions for 
installing the Windows version of HTCondor.  
The install program will set up a slightly customized configuration
file that may be further customized after the installation has completed.

Please read the copyright and disclaimer information in 
section~\ref{sec:license} on
page~\pageref{sec:license} of the manual.
Installation and
use of HTCondor is acknowledgment that you have read and agree to the
terms.

Be sure that the HTCondor tools are of the same version
as the daemons installed.
The HTCondor executable for distribution is packaged in
a single file named similar to:
\begin{verbatim}
  condor-7.9.6-122785-Windows-x86.msi
\end{verbatim}
\index{Windows!installation!initial file size}
This file is approximately 87 Mbytes in size, and it may be
removed once HTCondor is fully installed.

Before installing HTCondor, please consider joining the condor-world mailing
list.  Traffic on this list is kept to an absolute minimum.  It is only
used to announce new releases of HTCondor.
To subscribe, follow the directions given at
\URL{http://www.cs.wisc.edu/condor/mail-lists/}.

For any installation, HTCondor services are installed and run as the
Local System account.
Running the HTCondor services as any other account (such as a domain user) 
is not supported and could be problematic.
 
\subsubsection{Installation Requirements}

\begin{itemize}

\item HTCondor for Windows requires Windows 2000 SP4, Windows XP SP2, 
or a more recent version.

\item 300 megabytes of free disk space is recommended.  Significantly more 
disk space could be desired to be able to run jobs with large data files.

\item HTCondor for Windows will operate on either an NTFS or FAT file system.  However, for security purposes, NTFS is preferred.

\item HTCondor for Windows requires the Visual C++ 2008 C runtime library.
\end{itemize}

%%%%%%%%%%%%%%%%%%%%%%%%%%%%%%%%%%%%%%%%%%%%%%%%%%%%%%%%%%%%%%%%%%%%%%
\subsubsection{\label{sec:NT-Preparing-to-Install}Preparing to Install
HTCondor under Windows } 
%%%%%%%%%%%%%%%%%%%%%%%%%%%%%%%%%%%%%%%%%%%%%%%%%%%%%%%%%%%%%%%%%%%%%%

\index{Windows!installation!preparation}
Before installing the Windows version of HTCondor,
there are two major
decisions to make about the basic layout of the pool.

\begin{enumerate}
\item What machine will be the central manager?
\item Is there enough disk space for HTCondor?
\end{enumerate}

If the answers to these questions are already known,
skip to the Windows Installation Procedure section below,
section~\ref{sec:nt-install-procedure} on
page~\pageref{sec:nt-install-procedure}.
If unsure, read on.

\begin{itemize} 

%%%%%%%%%%%%%%%%%%%%%%%%%%%%%%%%%%%%%%%%%%%%%%%%%%%%%%%%%%%%%%%%%%%%%%
\item{What machine will be the central manager?}
%%%%%%%%%%%%%%%%%%%%%%%%%%%%%%%%%%%%%%%%%%%%%%%%%%%%%%%%%%%%%%%%%%%%%%

One machine in your pool must be the central manager.
This is the
centralized information repository for the HTCondor pool and is also the
machine that matches available machines with waiting
jobs.  If the central manager machine crashes, any currently active
matches in the system will keep running, but no new matches will be
made.  Moreover, most HTCondor tools will stop working.  Because of the
importance of this machine for the proper functioning of HTCondor, we
recommend installing it on a machine that is likely to stay up all the
time, or at the very least, one that will be rebooted quickly if it
does crash.  Also, because all the services will send updates (by
default every 5 minutes) to this machine, it is advisable to consider
network traffic and network layout when choosing the central
manager.

Install HTCondor on the central manager before installing
on the other machines within the pool.

%%%%%%%%%%%%%%%%%%%%%%%%%%%%%%%%%%%%%%%%%%%%%%%%%%%%%%%%%%%%%%%%%%%%%%
\item{Is there enough disk space for HTCondor?}
%%%%%%%%%%%%%%%%%%%%%%%%%%%%%%%%%%%%%%%%%%%%%%%%%%%%%%%%%%%%%%%%%%%%%%

\index{Windows!installation!required disk space}
The HTCondor release directory takes up a fair amount of space.
The size requirement for the release
directory is approximately 250 Mbytes.
HTCondor itself, however, needs space to store all of the jobs and their
input files.  If there will be large numbers of jobs,
consider installing HTCondor on a volume with a large amount
of free space.

\end{itemize}


%%%%%%%%%%%%%%%%%%%%%%%%%%%%%%%%%%%%%%%%%%%%%%%%%%%%%%%%%%%%%%%%%%%%%%
\subsubsection{\label{sec:nt-install-procedure}
Installation Procedure Using the MSI Program}
%%%%%%%%%%%%%%%%%%%%%%%%%%%%%%%%%%%%%%%%%%%%%%%%%%%%%%%%%%%%%%%%%%%%%%

% condor MUST be run as local system
% 
%  root == administrator
%  to install, must be running with administrator privileges
%  the kernel runs as == local system

Installation of HTCondor must be done by a user with administrator privileges.
After installation, the HTCondor services will be run under 
the local system account.
When HTCondor is running a user job, however, 
it will run that user job with normal user permissions.

Download HTCondor, and start the installation process by running the installer. 
The HTCondor installation is completed by answering questions 
and choosing options within the following steps.

\begin{description}
\item[If HTCondor is already installed.]

     If HTCondor has been previously installed,
     a dialog box will appear before the installation of HTCondor proceeds.
     The question asks if you wish to preserve your current
     HTCondor configuration files.
     Answer yes or no, as appropriate.
	 
     If you answer yes, your configuration files will not be changed, 
     and you will proceed to the point where the new binaries will be installed.

     If you answer no, then there will be a second question
     that asks if you want to use answers
     given during the previous installation
     as default answers.

\item[STEP 1: License Agreement.]

     The first step in installing HTCondor
     is a welcome screen and license agreement.
     You are reminded that it is best to run the installation
     when no other Windows programs are running.
     If you need to close other Windows programs, it is safe to cancel the
     installation and close them.
     You are asked to agree to the license.
     Answer yes or no.  If you should disagree with the License, the
     installation will not continue.

     Also fill in name and company information,
     or use the defaults as given.

\item[STEP 2: HTCondor Pool Configuration.]

     The HTCondor configuration needs to be set based upon
     if this is a new pool or to join an existing one.
     Choose the appropriate radio button.

     For a new pool, enter a chosen name for the pool.
     To join an existing pool, 
     enter the host name of the central manager of the pool.

\item[STEP 3: This Machine's Roles.] 

     Each machine within an HTCondor pool may either
     submit jobs or execute submitted jobs, or both
     submit and execute jobs.
     A check box determines if this machine will be a submit point for
     the pool.

     A set of radio buttons determines the ability and configuration of
     the ability to execute jobs.
     There are four choices:
     \begin{description}
     \item[Do not run jobs on this machine.]
     This machine will not execute HTCondor jobs.
     \item[Always run jobs and never suspend them.]
     \item[Run jobs when the keyboard has been idle for 15 minutes.]
     \item[Run jobs when the keyboard has been idle for 15 minutes,
and the CPU is idle.]
     \end{description}

     For testing purposes, it is often helpful to use the always run HTCondor
     jobs option. 

     For a machine that is to execute jobs and the choice is one of
the last two in the list,
HTCondor needs to further know what to do with the currently running jobs.
There are two choices:
     \begin{description}
     \item[Keep the job in memory and continue when the machine meets
the condition chosen for when to run jobs.]
     \item[Restart the job on a different machine.]
     \end{description}

     This choice involves a trade off.
     Restarting the job on a different machine is less intrusive
     on the workstation owner than leaving the job in memory for a later time.
     A suspended job left in memory will require swap space,
     which could be a scarce resource.
     Leaving a job in memory, however, has the benefit that accumulated
     run time is not lost for a partially completed job.

\item[STEP 4: The Account Domain.]

% not really used right now.  "Things that suck about NT."
% UNIX has 2 domains:  file system domain and user-ID domain
% NT has only 1:  a combination, and so going back to letter
% drives, things get screwed up.

     Enter the machine's accounting (or UID) domain.
     On this version of HTCondor for Windows, this setting is only used for user
     priorities (see section~\ref{sec:UserPrio} on
     page~\pageref{sec:UserPrio}) and to form a default e-mail address for
     the user.

\item[STEP 5: E-mail Settings.]

     Various parts of HTCondor will send e-mail to an HTCondor administrator
     if something goes wrong and requires human attention.
     Specify the e-mail address and the SMTP relay host
     of this administrator.  Please pay close attention to this e-mail,
     since it will indicate problems in the HTCondor pool.

\item[STEP 6: Java Settings.]
     In order to run jobs in the \SubmitCmd{java} universe,
     HTCondor must have the path to the jvm executable on the machine.
     The installer will search for and list the jvm path, if it finds one.
     If not, enter the path.
     To disable use of the \SubmitCmd{java} universe,
     leave the field blank.

\item[STEP 7: Host Permission Settings.]
     Machines within the HTCondor pool will need various types of 
     access permission. 
     The three categories of permission are read, write, and administrator.
     Enter the machines or domain to be given access permissions,
     or use the defaults provided.
     Wild cards and macros are permitted.

     \begin{description}
     \item[Read]
     Read access allows a machine to obtain information about
     HTCondor such as the status of machines in the pool and the job queues.
     All machines in the pool should be given read access. 
     In addition, giving read access to *.cs.wisc.edu 
     will allow the HTCondor team to obtain information about
     the HTCondor pool, in the event that debugging is needed.
     \item[Write]
     All machines in the pool should be given write access. 
     It allows the machines you specify to send information to your
     local HTCondor daemons, for example, to start an HTCondor job.
     Note that for a machine to join the HTCondor pool, 
     it must have both read and write access to all of the machines in the pool.
     \item[Administrator]
     A machine with administrator access will be allowed more
     extended permission to do things such as
     change other user's priorities, modify the job queue,
     turn HTCondor services on and off, and restart HTCondor.
     The central manager should be given administrator access
     and is the default listed.
     This setting is granted to the entire machine, 
     so care should be taken not to make this too open.
     \end{description}

     For more details on these access permissions, and others that can be
     manually changed in your configuration file, please
     see the section titled Setting Up IP/Host-Based Security in HTCondor
     in section section~\ref{sec:Host-Security}
     on page~\pageref{sec:Host-Security}.

\item[STEP 8: VM Universe Setting.]
     A radio button determines whether 
     this machine will be configured to run \SubmitCmd{vm} universe jobs 
     utilizing VMware.
     In addition to having the VMware Server installed, HTCondor also needs
     \Prog{Perl} installed.
     The resources available for \SubmitCmd{vm} universe jobs can be tuned 
     with these settings, or the defaults listed may be used.

     \begin{description}
     \item[Version]
     Use the default value, as only one version is currently supported.
     \item[Maximum Memory]
     The maximum memory that each virtual machine is permitted to use 
     on the target machine.
     \item[Maximum Number of VMs]
     The number of virtual machines that can be run in parallel 
     on the target machine.
     \item[Networking Support]
     The VMware instances can be configured to use network support.
     There are four options in the pull-down menu.
          \begin{itemize}
          \item None: No networking support.
          \item NAT: Network address translation.
          \item Bridged: Bridged mode.
          \item NAT and Bridged: Allow both methods.
          \end{itemize}
     \item[Path to Perl Executable]
     The path to the \Prog{Perl} executable.
     \end{description}

\item[STEP 9: HDFS Settings.]
     A radio button enables support for 
     the Hadoop Distributed File System (HDFS).
     When enabled, a further radio button specifies 
     either name node or data node mode.

     Running HDFS requires Java to be installed,
     and HTCondor must know where the installation is.
     Running HDFS in data node mode also requires the installation of Cygwin,
     and the path to the Cygwin directory must be added to the 
     global \Env{PATH} environment variable.

     HDFS has several configuration options that must be filled in to be used.
     \begin{description}
     \item[Primary Name Node]
     The full host name of the primary name node.
     \item[Name Node Port]
     The port that the name node is listening on.
     \item[Name Node Web Port]
     The port the name node's web interface is bound to.
     It should be different from the name node's main port.
     \end{description}

\item[STEP 10: Choose Setup Type]

\index{Windows!installation!location of files}
The next step is where the destination of the HTCondor files will be
decided.
We recommend that HTCondor be installed in the location shown as the default 
in the install choice:
\verb@C:\Condor@. This is due to several hard coded
paths in scripts and configuration files.
Clicking on the Custom choice permits changing the installation directory.

Installation on the local disk is chosen for several reasons.
The HTCondor services run as local system, and within Microsoft Windows, 
local system has no network privileges.
Therefore, for HTCondor to operate, 
HTCondor should be installed on a local hard drive,
as opposed to a network drive (file server).

The second reason for installation on the local disk is that
the Windows usage of drive letters has implications for where
HTCondor is placed.
The drive letter used must be not change, even when different users are
logged in.
Local drive letters do not change under normal operation of Windows.

While it is strongly discouraged, 
it may be possible to place HTCondor on a hard drive that is not local,
if a dependency is added to the service control manager
such that HTCondor starts after the required file services
are available.

%  !! goes in C:/condor   (default)
%  !! advice is really should go on local hard drive,
%  as opposed to a network drive (also called file server)
%  Because,
%    1. HTCondor runs as local system, and accesses to a network
%      drive can't be authenticated  -- local system has
%      no network privileges.
%    2.  it is likely that you don't have this set up:
%    (and you need it to make it work)
%    you can add a dependency in the service control manager
%    that condor should start after the file services are
%    available
%    3. drive letters are "system-wide"
%    Must have dedicated letter (for all users), that remains
%    intact for all time, or condor won't know where
%    things are and can't get access (without its "letter")

\end{description}


%%%%%%%%%%%%%%%%%%%%%%%%%%%%%%%%%%%%%%%%%%%%%%%%%%%%%%%%%%%%%%%%%%%%%%
\subsubsection{\label{sec:nt-unattended-install-procedure}
Unattended Installation Procedure Using the Included Set Up Program}
%%%%%%%%%%%%%%%%%%%%%%%%%%%%%%%%%%%%%%%%%%%%%%%%%%%%%%%%%%%%%%%%%%%%%%

\index{Windows!installation!unattended install}
This section details how to run the HTCondor for Windows installer in an
unattended batch mode.
This mode is one that occurs completely from the command prompt,
without the GUI interface.

The HTCondor for Windows installer uses the Microsoft Installer (MSI)
technology, and it can be configured for unattended installs analogous
to any other ordinary MSI installer.

The following is a sample batch file that is used to set all the
properties necessary for an unattended install.

\begin{verbatim}
@echo on
set ARGS=
set ARGS=NEWPOOL="N"
set ARGS=%ARGS% POOLNAME=""
set ARGS=%ARGS% RUNJOBS="C"
set ARGS=%ARGS% VACATEJOBS="Y"
set ARGS=%ARGS% SUBMITJOBS="Y"
set ARGS=%ARGS% CONDOREMAIL="you@yours.com"
set ARGS=%ARGS% SMTPSERVER="smtp.localhost"
set ARGS=%ARGS% HOSTALLOWREAD="*"
set ARGS=%ARGS% HOSTALLOWWRITE="*"
set ARGS=%ARGS% HOSTALLOWADMINISTRATOR="$(IP_ADDRESS)"
set ARGS=%ARGS% INSTALLDIR="C:\Condor"
set ARGS=%ARGS% POOLHOSTNAME="$(IP_ADDRESS)"
set ARGS=%ARGS% ACCOUNTINGDOMAIN="none"
set ARGS=%ARGS% JVMLOCATION="C:\Windows\system32\java.exe"
set ARGS=%ARGS% USEVMUNIVERSE="N"set ARGS=%ARGS% VMMEMORY="128"
set ARGS=%ARGS% VMMAXNUMBER="$(NUM_CPUS)"
set ARGS=%ARGS% VMNETWORKING="N"
set ARGS=%ARGS% USEHDFS="N"
set ARGS=%ARGS% NAMENODE=""
set ARGS=%ARGS% HDFSMODE="HDFS_NAMENODE"
set ARGS=%ARGS% HDFSPORT="5000"
set ARGS=%ARGS% HDFSWEBPORT="4000"
 
msiexec /qb /l* condor-install-log.txt /i condor-7.1.0-winnt50-x86.msi %ARGS%
\end{verbatim}

Each property corresponds to answers that would have been
supplied while running an interactive installer.
The following is a brief explanation of each property
as it applies to unattended installations:

\begin{description}
\item [NEWPOOL = $<$ Y \Bar\ N $>$]
determines whether the installer will create a new pool with the target
machine as the central manager.

\item [POOLNAME]
sets the name of the pool, if a new pool is to be created. Possible values
are either the name or the empty string \verb@""@.

\item [RUNJOBS = $<$ N \Bar\ A \Bar\ I \Bar\ C $>$]
determines when HTCondor will run jobs. This can be set to:
\begin{itemize}
\item Never run jobs (N)
\item Always run jobs (A)
\item Only run jobs when the keyboard and mouse are Idle (I)
\item Only run jobs when the keyboard and mouse are idle and the CPU
usage is low (C)
\end{itemize}

\item [VACATEJOBS = $<$ Y \Bar\ N $>$]
determines what HTCondor should do when it has to stop the execution of
a user job. When set to Y, HTCondor will vacate the job and start
it somewhere else if possible. When set to N, HTCondor will merely
suspend the job in memory and wait for the machine to become available
again. 

\item[SUBMITJOBS  = $<$ Y \Bar\ N $>$]
will cause the installer to configure the machine as a submit
node when set to Y. 

\item[CONDOREMAIL]
sets the e-mail address of the HTCondor administrator. Possible values are
an e-mail address or the empty string \verb@""@.

\item[HOSTALLOWREAD]
is a list of host names that are allowed to issue READ commands to
HTCondor daemons. This value should be set in accordance with the
\Macro{HOSTALLOW\_READ} setting in the configuration file, as described in
section~\ref{sec:Host-Security} on page~\pageref{sec:Host-Security}.

\item[HOSTALLOWWRITE]
is a list of host names that are allowed to issue WRITE commands to
HTCondor daemons. This value should be set in accordance with the
\Macro{HOSTALLOW\_WRITE} setting in the configuration file, as described in
section~\ref{sec:Host-Security} on page~\pageref{sec:Host-Security}.

\item[HOSTALLOWADMINISTRATOR]
is a list of host names that are allowed to issue ADMINISTRATOR commands to
HTCondor daemons. This value should be set in accordance with the
\Macro{HOSTALLOW\_ADMINISTRATOR} setting in the configuration file, 
as described in
section~\ref{sec:Host-Security} on page~\pageref{sec:Host-Security}.

\item[INSTALLDIR]
defines the path to the directory where HTCondor will be installed. 

\item[POOLHOSTNAME]
defines the host name of the pool's central manager. 

\item[ACCOUNTINGDOMAIN] 
defines the accounting (or UID) domain the target machine will be in.

\item[JVMLOCATION]
defines the path to Java virtual machine on the target machine.

\item[SMTPSERVER]
defines the host name of the SMTP server that the target machine is to
use to send e-mail.

\item [VMMEMORY]
an integer value that defines the maximum memory each VM run on the target
machine.

\item [VMMAXNUMBER]
an integer value that defines the number of VMs that can be run in parallel
on the target machine.

\item [VMNETWORKING = $<$ N \Bar\ A \Bar\ B \Bar\ C $>$]
determines if VM Universe can use networking. This can be set to:
\begin{itemize}
\item None (N)
\item NAT (A)
\item Bridged (B)
\item NAT and Bridged (C)
\end{itemize}

\item [USEVMUNIVERSE = $<$ Y \Bar\ N $>$]
will cause the installer to enable VM Universe jobs on the target machine.


\item[PERLLOCATION]
defines the path to \Prog{Perl} on the target machine. This is required in
order to use the \SubmitCmd{vm} universe.

\item[USEHDFS = $<$ Y \Bar\ N $>$]
determines if HDFS is run.

\item[HDFSMODE $<$ HDFS\_DATANODE \Bar\ HDFS\_NAMENODE $>$]
sets the mode HDFS runs in.

\item[NAMENODE]
sets the host name of the primary name node.

\item[HDFSPORT]
sets the port number that the primary name node listens to.

\item [HDFSWEBPORT]
sets the port number that the name node web interface is bound to.
\end {description}

After defining each of these properties for the MSI installer, the
installer can be started with the \Prog{msiexec} command. The following
command starts the installer in unattended mode, and it dumps a journal of
the installer's progress to a log file:
\footnotesize
\begin{verbatim}
msiexec /qb /lxv* condor-install-log.txt /i condor-7.2.2-winnt50-x86.msi [property=value] ... 
\end{verbatim}
\normalsize

More information on the features of \Prog{msiexec}
can be found at Microsoft's website at
\URL{http://www.microsoft.com/resources/documentation/windows/xp/all/proddocs/en-us/msiexec.mspx}.

%%%%%%%%%%%%%%%%%%%%%%%%%%%%%%%%%%%%%%%%%%%%%%%%%%%%%%%%%%%%%%%%%%%%%%
\subsubsection{\label{sec:NT-Manual-Install}Manual Installation HTCondor on Windows}
%%%%%%%%%%%%%%%%%%%%%%%%%%%%%%%%%%%%%%%%%%%%%%%%%%%%%%%%%%%%%%%%%%%%%%

\index{Windows!manual install}
If you are to install HTCondor on many different machines, you may wish
to use some other mechanism to install HTCondor on additional machines
rather than running the Setup program described above on each machine.

\Warn This is for advanced users only!  All others should use the Setup program described above. 

Here is a brief overview of how to install HTCondor manually without using the provided GUI-based setup program:

\begin{description}
\item [The Service]
The service that HTCondor will install is called "Condor".  The Startup
Type is Automatic.  The service should log on as System Account, but
\Bold{do not enable} "Allow Service to Interact with Desktop".  The
program that is run is \Condor{master.exe}.

The HTCondor service can be installed and removed using the
\File{sc.exe} tool, which is included in Windows XP and Windows 2003
Server. The tool is also available as part of the Windows 2000
Resource Kit.

Installation can be done as follows:
\begin{verbatim}
sc create Condor binpath= c:\condor\bin\condor_master.exe
\end{verbatim}

To remove the service, use:
\begin{verbatim}
sc delete Condor
\end{verbatim}

\item [The Registry]
HTCondor uses a few registry entries in its operation.  The key that HTCondor
uses is HKEY\_LOCAL\_MACHINE/Software/Condor.  The values that HTCondor puts
in this registry key serve two purposes.
\begin{enumerate}
\item The values of CONDOR\_CONFIG and RELEASE\_DIR are used for HTCondor
to start its service.

CONDOR\_CONFIG should point to the \File{condor\_config} file.  In this version
of HTCondor, it \Bold{must} reside on the local disk.

RELEASE\_DIR should point to the directory where HTCondor is installed.  This
is typically
\verb@C:\Condor@, and again, this \Bold{must} reside on the
local disk.

\item The other purpose is storing the entries from the last installation
so that they can be used for the next one.
\end{enumerate}

\item [The File System]
The files that are needed for HTCondor to operate are identical to the Unix
version of HTCondor, except that executable files end in \File{.exe}.  For
example the on Unix one of the files is \File{condor\_master} and on HTCondor
the corresponding file is \File{condor\_master.exe}.

These files currently must reside on the local disk for a variety of reasons.
Advanced Windows users might be able to put the files on remote resources.
The main concern is twofold.  First, the files must be there when the service
is started.  Second, the files must always be in the same spot (including
drive letter), no matter who is logged into the machine.  

Note also that when installing manually, you will need to create the
directories that HTCondor will expect to be present given your
configuration. This normally is simply a matter of creating the
\File{log}, \File{spool}, and \File{execute} directories.

\end{description}


%%%%%%%%%%%%%%%%%%%%%%%%%%%%%%%%%%%%%%%%%%%%%%%%%%%%%%%%%%%%%%%%%%%%%%
\subsubsection{\label{nt-installed-now-what}
Starting HTCondor Under Windows After Installation}
%%%%%%%%%%%%%%%%%%%%%%%%%%%%%%%%%%%%%%%%%%%%%%%%%%%%%%%%%%%%%%%%%%%%%%
\index{Windows!starting the HTCondor service}
\index{starting HTCondor!Windows platforms}

After the installation of HTCondor is completed, the HTCondor service
must be started.  If you used the GUI-based setup program to install
HTCondor, the HTCondor service should already be started.  If you installed
manually, HTCondor must
be started by hand, or you can simply reboot. \Note The HTCondor service
will start automatically whenever you reboot your machine.

To start HTCondor by hand:
\begin{enumerate}
\item From the Start menu, choose Settings.
\item From the Settings menu, choose Control Panel.
\item From the Control Panel, choose Services.
\item From Services, choose Condor, and Start.
\end{enumerate}

Or, alternatively you can enter the following command from a command prompt:
\begin{verbatim}
         net start condor
\end{verbatim}

\index{Windows!HTCondor daemon names}
Run the Task Manager (Control-Shift-Escape) to check that HTCondor
services are running.  The following tasks should
be running:  
\begin{itemize}
\item \Condor{master.exe}
\item \Condor{negotiator.exe}, if this machine is a central manager.
\item \Condor{collector.exe}, if this machine is a central manager.
\item \Condor{startd.exe}, if you indicated that this HTCondor node should start jobs
\item \Condor{schedd.exe}, if you indicated that this HTCondor node should submit jobs
to the HTCondor pool.
\end{itemize}

Also, you should now be able to open up a new cmd (DOS prompt) window, and
the HTCondor bin directory should be in your path, so you can issue the normal
HTCondor commands, such as \Condor{q} and \Condor{status}.

\index{installation!Windows|)}
\index{Windows!installation|)}

%%%%%%%%%%%%%%%%%%%%%%%%%%%%%%%%%%%%%%%%%%%%%%%%%%%%%%%%%%%%%%%%%%%%%%
\subsubsection{\label{nt-running-now-what}
HTCondor is Running Under Windows ... Now What?}
%%%%%%%%%%%%%%%%%%%%%%%%%%%%%%%%%%%%%%%%%%%%%%%%%%%%%%%%%%%%%%%%%%%%%%

Once HTCondor services are running, try submitting test jobs.
Example 2 within section~\ref{sec:sample-submit-files} 
on page~\pageref{sec:sample-submit-files} presents a vanilla
universe job.

%%%%%%%%%%%%%%%%%%%%%%%%%%%%%%%%%%%%%%%%%%%%%%%%%%%%%%%%%%%%%%%%%%%%%%

%%%%%%%%%%%%%%%%%%%%%%%%%%%%%%%%%%%%%%%%%%%%%%%%%%%%%%%%%%%%%%%%%%%%%%
\subsection{\label{sec:install-rpms} RPMs}
%%%%%%%%%%%%%%%%%%%%%%%%%%%%%%%%%%%%%%%%%%%%%%%%%%%%%%%%%%%%%%%%%%%%%%

\index{installation!using Red Hat RPMs}
\index{RPM installation on Red Hat}
RPMs are available in HTCondor \VersionNotice.
We provide a Yum repository, as well as 
installation and configuration in one easy step.
This RPM installation is currently available for Red Hat-compatible
systems only.
As of HTCondor version 7.5.1, 
the HTCondor RPM installs into FHS locations.

Yum repositories are at
 \URL{http://research.cs.wisc.edu/htcondor/yum/} .
The repositories are named to distinguish stable releases from
development releases and by Red Hat version number. 
The 6 repositories are:
\begin{itemize}
  \item \File{condor-stable-rhel4.repo}
  \item \File{condor-stable-rhel5.repo}
  \item \File{condor-stable-rhel6.repo}
  \item \File{condor-development-rhel4.repo}
  \item \File{condor-development-rhel5.repo}
  \item \File{condor-development-rhel6.repo}
\end{itemize}

For HTCondor to work properly under RHEL5, 
the Red Hat Network channel called \AdStr{RHEL Virtualization} 
must be explicitly enabled.

Here is an ordered set of steps that get HTCondor running using the RPM.
\begin{enumerate}
\item The HTCondor package will automatically add a \Login{condor} user/group,
if it does not exist already.
Sites wishing to control the attributes of this user/group 
should add the \Login{condor} user/group manually before installation.

\item
\item Download and install the meta-data that describes 
the appropriate YUM repository. 
This example is for the stable series, on RHEL 5. 
\footnotesize
\begin{verbatim}
  cd /etc/yum.repos.d
  wget http://research.cs.wisc.edu/htcondor/yum/repo.d/condor-stable-rhel5.repo
\end{verbatim}
\normalsize
Note that this step need be done only once;
do not get the same repository more than once.

\item Install HTCondor.
For 32-bit machines:
\begin{verbatim}
  yum install condor
\end{verbatim}
For 64-bit machines:
\begin{verbatim}
  yum install condor.x86_64
\end{verbatim}

\item As needed, edit the HTCondor configuration files to customize.
The configuration files are in the directory \File{/etc/condor/} .
Do \emph{not} use \Condor{configure} or \Condor{install} for configuration.
The installation will be able to find configuration files without
additional administrative intervention,
as the configuration files are placed in \File{/etc},
and HTCondor searches this directory.

\item Start HTCondor daemons:
\begin{verbatim}
  /sbin/service condor start
\end{verbatim}

\end{enumerate}

% Alain thinks that upgrades DO work.
%RPM upgrade (\Opt{-u} option) does not currently
%work for HTCondor \VersionNotice.

%%%%%%%%%%%%%%%%%%%%%%%%%%%%%%%%%%%%%%%%%%%%%%%%%%%%%%%%%%%%%%%%%%%%%%
\subsection{\label{sec:install-debs} Debian Packages}
%%%%%%%%%%%%%%%%%%%%%%%%%%%%%%%%%%%%%%%%%%%%%%%%%%%%%%%%%%%%%%%%%%%%%%
\index{installation!using Debian packages}
\index{Debian installation with Debian packages}

Debian packages are available in HTCondor \VersionNotice.
We provide an APT repository, as well as 
installation and configuration in one easy step.
These Debian packages of HTCondor are currently available for 
Debian 6 (squeeze) and Debian 7 (wheezy).
As of HTCondor version 7.5.1, 
the HTCondor Debian package installs into FHS locations.

The HTCondor APT repositories are specified at
 \URL{http://research.cs.wisc.edu/htcondor/debian/} .
See this web page for repository information.

Here is an ordered set of steps that get HTCondor running.
\begin{enumerate}
\item The HTCondor package will automatically add a \Login{condor} user/group,
if it does not exist already.
Sites wishing to control the attributes of this user/group 
should add the \Login{condor} user/group manually before installation.

\item If not already present,
set up access to the appropriate APT repository;
they are distinguished as stable or development release,
and by operating system. 
Ensure that the correct one of the following release and 
operating system-specific lines is in 
the file \File{/etc/apt/sources.list} .
\footnotesize
\begin{verbatim}
deb http://research.cs.wisc.edu/htcondor/debian/stable/ squeeze contrib
deb http://research.cs.wisc.edu/htcondor/debian/development/ squeeze contrib
deb http://research.cs.wisc.edu/htcondor/debian/stable/ wheezy contrib
deb http://research.cs.wisc.edu/htcondor/debian/development/ wheezy contrib
\end{verbatim}
\normalsize
Note that this step need be done only once;
do not add the same repository more than once.

\item Install and start HTCondor services:
\begin{verbatim}
  apt-get update
  apt-get install condor
\end{verbatim}

\item As needed, edit the HTCondor configuration files to customize.
The configuration files are in the directory \File{/etc/condor/} .
Do \emph{not} use \Condor{configure} or \Condor{install} for configuration.
The installation will be able to find configuration files without
additional administrative intervention,
as the configuration files are placed in \File{/etc},
and HTCondor searches this directory.

Then, if any configuration changes are made, restart HTCondor with
\begin{verbatim}
  /etc/init.d/condor restart
\end{verbatim}

\end{enumerate}

%%%%%%%%%%%%%%%%%%%%%%%%%%%%%%%%%%%%%%%%%%%%%%%%%%%%%%%%%%%%%%%%%%%%%%
\subsection{\label{sec:upgrade-directions}Upgrading - Installing a Newer
Version of HTCondor}
%%%%%%%%%%%%%%%%%%%%%%%%%%%%%%%%%%%%%%%%%%%%%%%%%%%%%%%%%%%%%%%%%%%%%%
Section~\ref{sec:Pool-Upgrade}  on page~\pageref{sec:Pool-Upgrade}
within the section on Pool Management describes 
strategies for doing an upgrade:
changing the running version of HTCondor 
from the current installation to a newer version.

