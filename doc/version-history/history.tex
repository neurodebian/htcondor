%%%%%%%%%%%%%%%%%%%%%%%%%%%%%%%%%%%%%%%%%%%%%%%%%%%%%%%%%%%%%%%%%%%%%%
\section{\label{sec:History-Intro}Introduction to HTCondor Versions}
%%%%%%%%%%%%%%%%%%%%%%%%%%%%%%%%%%%%%%%%%%%%%%%%%%%%%%%%%%%%%%%%%%%%%%

This chapter provides descriptions of what features have been added or
bugs fixed for each version of HTCondor.
The first section describes the HTCondor version numbering scheme, what
the numbers mean, and what the different \Term{release series} are.
The rest of the sections each describe a specific release series, and
all the HTCondor versions found in that series.

%%%%%%%%%%%%%%%%%%%%%%%%%%%%%%%%%%%%%%%%%%%%%%%%%%%%%%%%%%%%%%%%%%%%%%
\subsection{\label{sec:Version-Number-Scheme}
HTCondor Version Number Scheme}
%%%%%%%%%%%%%%%%%%%%%%%%%%%%%%%%%%%%%%%%%%%%%%%%%%%%%%%%%%%%%%%%%%%%%%

Starting with version 6.0.1, HTCondor adopted a new, hopefully easy to
understand version numbering scheme.
It reflects the fact that HTCondor is both a production system and a
research project.
The numbering scheme was primarily taken from the Linux kernel's
version numbering, so if you are familiar with that, it should seem
quite natural.

There will usually be two HTCondor versions available at any given time,
the \Term{stable} version, and the \Term{development} version.
Gone are the days of ``patch level 3'', ``beta2'', or any other random
words in the version string.
All versions of HTCondor now have exactly three numbers, separated by
``.''   

\begin{itemize}

\item The first number represents the major version number, and will
change very infrequently.

\item \emph{The thing that determines whether a version of HTCondor is
\Term{stable} or \Term{development} is the second digit.
Even numbers represent stable versions, while odd numbers represent
development versions.}

\item The final digit represents the minor version number, which
defines a particular version in a given release series.

\end{itemize}


%%%%%%%%%%%%%%%%%%%%%%%%%%%%%%%%%%%%%%%%%%%%%%%%%%%%%%%%%%%%%%%%%%%%%%
\subsection{\label{sec:Stable-Series}The Stable Release Series}
%%%%%%%%%%%%%%%%%%%%%%%%%%%%%%%%%%%%%%%%%%%%%%%%%%%%%%%%%%%%%%%%%%%%%%

People expecting the stable, production HTCondor system should download
the stable version, denoted with an even number in the second digit of
the version string.
Most people are encouraged to use this version.  
We will only offer our paid support for versions of HTCondor from the
stable release series.

\emph{On the stable series, new minor version releases will only
be made for bug fixes and to support new platforms.}
No new features will be added to the stable series.
People are encouraged to install new stable versions of HTCondor when
they appear, since they probably fix bugs you care about.
Hopefully, there will not be many minor version releases for any given
stable series.


%%%%%%%%%%%%%%%%%%%%%%%%%%%%%%%%%%%%%%%%%%%%%%%%%%%%%%%%%%%%%%%%%%%%%%
\subsection{\label{sec:Developement-Series}
The Development Release Series}
%%%%%%%%%%%%%%%%%%%%%%%%%%%%%%%%%%%%%%%%%%%%%%%%%%%%%%%%%%%%%%%%%%%%%%

Only people who are interested in the latest research, new features
that haven't been fully tested, etc, should download the development
version, denoted with an odd number in the second digit of the version
string.  
We will make a best effort to ensure that the development series will
work, but we make no guarantees.

On the development series, new minor version releases will probably
happen frequently.
People should not feel compelled to install new minor versions unless
they know they want features or bug fixes from the newer development
version.

\emph{Most sites will probably never want to install a development
version of HTCondor for any reason.}
Only if you know what you are doing (and like pain), or were
explicitly instructed to do so by someone on the HTCondor Team, should
you install a development version at your site.

After the feature set of the development series is satisfactory to the
HTCondor Team, we will put a code freeze in place, and from that point
forward, only bug fixes will be made to that development series.
When we have fully tested this version, we will release a new stable
series, resetting the minor version number, and start work on a new
development release from there.

%%%%%%%%%%%%%%%%%%%%%%%%%%%%%%%%%%%%%%%%%%%%%%%%%%%%%%%%%%%%%%%%%%%%%%
% The rest of this file just inputs other files which contain sections
% describing each release series in detail.
%%%%%%%%%%%%%%%%%%%%%%%%%%%%%%%%%%%%%%%%%%%%%%%%%%%%%%%%%%%%%%%%%%%%%%

% upgrade instructions are in the Pool Management section
%%%%%%%%%%%%%%%%%%%%%%%%%%%%%%%%%%%%%%%%%%%%%%%%%%%%%%%%%%%%%%%%%%%%%%
\section{\label{sec:to-8.2}Upgrading from the 8.0 series to the 8.2 series of HTCondor}
%%%%%%%%%%%%%%%%%%%%%%%%%%%%%%%%%%%%%%%%%%%%%%%%%%%%%%%%%%%%%%%%%%%%%%

\index{upgrading!items to be aware of}
Upgrading from the 8.0 series of HTCondor to the 8.2 series 
will bring new features introduced in the 8.1 series of HTCondor.
These new features include:
configuration is more powerful with new syntax and features, and
the default configuration policy does not preempt jobs,
monitoring is enhanced and now integrates with Ganglia,
automated detection and management of GPUs,
numerous scalability enhancements improve performance,
an improved Python API including support for Python 3,
new native packaging and ports are available for the latest Linux 
distributions including Red Hat 7 Beta and Debian 7,
cloud computing improvements including support for
EC2 spot instances, OpenStack, and \Condor{ssh\_to\_job} 
directly into EC2 jobs,
grid universe jobs can now target Google Compute Engine and BOINC servers,
partitionable slots now are compatible with \Condor{startd} \MacroNI{RANK}
expressions, and consumption policies permit partitionable slots 
to be split into dynamic slots at negotiation time,
improved data management including dynamic adjustment of the level of 
file transfer concurrency based on disk load 
(see section~\ref{param:FileTransferDiskLoadThrottle}), 
and experimental support to allow the execution of a job to be overlaid 
with the transfer of output files from the previous different job,
and
the new \Condor{sos} tool helps administrators manage overloaded daemons. 

Upgrading from the 8.0 series of HTCondor to the 8.2 series will
also introduce changes that administrators of sites running from an older
HTCondor version should be aware of when planning an upgrade.  
Here is a list of items that administrators should be aware of.

\begin{itemize}

\item New configuration syntax and features change:
  \begin{itemize}
  \item The interaction of comments and the line continuation character
   has changed.  See  section~\ref{sec:Other-Syntax} for the current
   interaction. 
  \item The use of a colon character (\verb@:@) instead of the
   equals sign (\verb@=@) in assigning a value to a configuration variable
   causes tools that parse configuration to output a warning.
   Therefore, any custom parsing of tool output may need to be updated to
   handle this warning.
   Previous versions of the default configuration set variable
   \MacroNI{RUNBENCHMARKS} using a colon character;
   HTCondor code explicitly suppresses the warning in this case.
  \end{itemize}

\item The default user priority factor for new users has changed 
from 1 to 1000.
Therefore, unless the accountant log is discarded,
existing users will still have a priority factor of 1,
while new users will have a priority factor of 1000.
Use \Condor{userprio} to change the priority factor of existing users
if the accountant log is maintained across the upgrade. 
\Ticket{4282}

\item For Windows platforms,
HTCondor has switched to use the newer 2012 Microsoft compiler,
which uses the Visual C++ 2012 Runtime components.
Therefore, the HTCondor MSI installer will acquire this Runtime,
if it is not already installed.

\item The meaning of \Expr{cpus=auto} when there are more 
slots than CPUs has changed within the configuration. 
In the \Expr{SLOT\_TYPE\_<N>} configuration variable,
\Expr{cpus=auto} previously resulted in 1 CPU per slot. 
Now, all slots with \Expr{cpu=auto} get an equal share of the CPUs, 
rounded down.
\Ticket{3249}

\item The DAGMan node status file formatting has changed.
The format of the DAG node status file is now New ClassAds,
and the amount of information in the file has increased.

\item Setting configuration variable
\Macro{DAGMAN\_ALWAYS\_USE\_NODE\_LOG} to \Expr{False}
or using the corresponding \Opt{-dont\_use\_default\_node\_log} option
to \Condor{submit\_dag} is no longer recommended.
Note that at strictness setting 1 (the default), setting
\MacroNI{DAGMAN\_ALWAYS\_USE\_NODE\_LOG} to \Expr{False}
will cause a fatal error. 
If the DAG must be run with \MacroNI{DAGMAN\_ALWAYS\_USE\_NODE\_LOG} 
set to \Expr{False},
a good way to deal with upgrading is to use DAGMan Halt files 
to cause all of the running DAGs to drain from the queue, 
and then do the upgrade after the DAGs have stopped.  
After the upgrade is done, 
edit the per-DAG configuration files to have 
\MacroNI{DAGMAN\_ALWAYS\_USE\_NODE\_LOG} set to \Expr{True},
or set \MacroNI{DAGMAN\_USE\_STRICT} to 0 and 
re-submit the DAGs, which will then run the Rescue DAGs.

\item If using \Expr{ENABLE\_IPV6 = True}, the configuration must
also set \Expr{ENABLE\_IPV4 = False}.
If both are enabled simultaneously,
daemons will listen on both IPv4 and IPv6, 
but will only advertise one of the two addresses.

\item Globus 5.2.2 or a more recent version is now required 
for grid universe jobs of grid-type nordugrid and cream.
Globus version 5.2.5 is included in this 8.2.0 release of HTCondor.
HTCondor will prefer to use libraries already installed in \File{/usr/lib[64]},
when present.
\Ticket{4243}

\item If referencing attribute \AdAttr{SubmittorUserPrio} in
configuration, such as in the \MacroNI{PREEMPTION\_REQUIREMENTS} expression,
you will need to change it to \AdAttr{SubmitterUserPrio} 
Note the spelling difference in the ClassAd attribute name.
\Ticket{4369}

\item HTCondor can not distinguish normal from abnormal job exit
for Nordugrid ARC grids.
Therefore, all grid-type nordugrid jobs will be recorded as 
terminating normally, with an exit code from 0 to 255.
\Ticket{4342}

\item For configuration, parameter substitution now honors per-daemon 
overrides.  This long standing bug's fix may result in subtle changes
to the way that your configuration files are processed.

\end{itemize}


%%%%%%%%%%%%%%%%%%%%%%%%%%%%%%%%%%%%%%%%%%%%%%%%%%%%%%%%%%%%%%%%%%%%%%%
\section{\label{sec:to-8.0}Upgrading from the 7.8 series to the 8.0 series of HTCondor}
%%%%%%%%%%%%%%%%%%%%%%%%%%%%%%%%%%%%%%%%%%%%%%%%%%%%%%%%%%%%%%%%%%%%%%

\index{upgrading!items to be aware of}
While upgrading from the 7.8 series of HTCondor to the 8.0 series 
will bring many
new features and improvements introduced in the 7.9 series of HTCondor,
it will
also introduce changes that administrators of sites running from an older
HTCondor version should be aware of when planning an upgrade.  
Here is a list of items that administrators should be aware of.

\begin{itemize}

\item There is an issue with DAGMan jobs upon upgrade
from HTCondor version 7.8.x or an earlier version
to version 8.0.0.
Without administrative intervention,
queued DAGMan jobs will restart from the beginning of the DAG
after the upgrade.
There will be no issue if the upgrade is 
from HTCondor version 7.8.x or an earlier version to HTCondor version 8.0.1
or later versions.

To avoid starting DAGMan jobs from the beginning after the upgrade,
the administrator should ensure that no \Condor{dagman} jobs are queued.
Do a \Condor{rm} on all \Condor{dagman} jobs and wait for Rescue DAGs
to be written before shutting down HTCondor to perform the upgrade.
Any \Condor{dagman} jobs that are on hold should be released before
being removed.
After the upgrade is complete and HTCondor has restarted,
all of these DAGMan jobs should be re-submitted.
This will cause them to read the appropriate Rescue DAGs and 
continue on.

To avoid losing work within partially-completed node jobs,
an alternative is to use the halt file feature,
as described in section~\ref{sec:DagSuspend}.
This will cause all
\Condor{dagman} jobs to eventually drain from the queue(s).
This will take longer than doing a \Condor{rm} on those jobs.
\Condor{dagman} jobs drained via the halt file method will also
have to be re-submitted after the upgrade.

\item The upgrade will change the machine ClassAd attribute
\AdAttr{CheckpointPlatform} for all machines.
This implies that any standard universe job with a checkpoint 
from before the upgrade will not resume after the upgrade.
To work around this potential difficulty, either change the 
attribute \AdAttr{CheckpointPlatform} on all machines to their previous value 
by setting the \Macro{CHECKPOINT\_PLATFORM} configuration variable,
or change the \AdAttr{LastCheckpointPlatform} attribute for all jobs
that have produced a checkpoint.
Make the change by using \Condor{qedit}.

For example, if machine attribute \AdAttr{CheckpointPlatform} changed 
from \verb;LINUX INTEL 2.6.x normal N/A; to 
\verb;LINUX INTEL 2.6.x normal N/A ssse3 sse4_1 sse4_2;,
use the following command:

\footnotesize
\begin{verbatim}
condor_qedit -constraint 'LastCheckpointPlatform == "LINUX INTEL 2.6.x normal N/A"'
    LastCheckpointPlatform "LINUX INTEL 2.6.x normal N/A ssse3 sse4_1 sse4_2"
\end{verbatim}
\normalsize

\end{itemize}


%%%      PLEASE RUN A SPELL CHECKER BEFORE COMMITTING YOUR CHANGES!
%%%      PLEASE RUN A SPELL CHECKER BEFORE COMMITTING YOUR CHANGES!
%%%      PLEASE RUN A SPELL CHECKER BEFORE COMMITTING YOUR CHANGES!
%%%      PLEASE RUN A SPELL CHECKER BEFORE COMMITTING YOUR CHANGES!
%%%      PLEASE RUN A SPELL CHECKER BEFORE COMMITTING YOUR CHANGES!

%%%%%%%%%%%%%%%%%%%%%%%%%%%%%%%%%%%%%%%%%%%%%%%%%%%%%%%%%%%%%%%%%%%%%%
\section{\label{sec:History-8-2}Stable Release Series 8.2}
%%%%%%%%%%%%%%%%%%%%%%%%%%%%%%%%%%%%%%%%%%%%%%%%%%%%%%%%%%%%%%%%%%%%%%

This is a stable release series of HTCondor.
As usual, only bug fixes (and potentially, ports to new platforms)
will be provided in future 8.2.x releases.
New features will be added in the 8.3.x development series.

The details of each version are described below.

%%%%%%%%%%%%%%%%%%%%%%%%%%%%%%%%%%%%%%%%%%%%%%%%%%%%%%%%%%%%%%%%%%%%%%
\subsection*{\label{sec:New-8-2-3}Version 8.2.3}
%%%%%%%%%%%%%%%%%%%%%%%%%%%%%%%%%%%%%%%%%%%%%%%%%%%%%%%%%%%%%%%%%%%%%%

\noindent Release Notes:

\begin{itemize}

\item HTCondor version 8.2.3 released on October 1, 2014.

\item This version of HTCondor includes a full port for 
Ubuntu 14.04 on the x86\_64 architecture.
A full port includes support for the standard universe.
\Ticket{4562}

\end{itemize}


\noindent New Features:

\begin{itemize}

\item The new configuration variable 
\Macro{RUN\_FILETRANSFER\_PLUGINS\_WITH\_ROOT} permits file transfer
plug-ins to run with \Login{root} privilege,
when HTCondor daemons are run as \Login{root},
and when set to the non-default value of \Expr{True}.
\Ticket{4561}

\item The new configuration variable \Macro{NETWORK\_HOSTNAME} sets
the host name that HTCondor uses to identify the local machine.
If \MacroNI{NETWORK\_HOSTNAME} is not set, 
then HTCondor uses the \Procedure{gethostname} function to determine 
the machine's host name.
This variable is useful if a machine has multiple network interfaces
with different host names.
\Ticket{4570}

\item Configuration variable \Macro{JOB\_ROUTER\_DEFAULTS} tolerates
the syntax of omitting the outer square brackets that would be 
required by new ClassAd syntax,
in order to facilitate appending to an existing value. 
If the value of \MacroNI{JOB\_ROUTER\_DEFAULTS} does not have
enclosing square brackets, 
the value will be parsed as if they are present.
\Ticket{4433}

\end{itemize}

\noindent Bugs Fixed:

\begin{itemize}

\item The RedHat 7 RPM contains the service file to start up 
HTCondor via \Prog{systemd} instead of via \Prog{init} scripts.
\Ticket{4534}

\item Fixed bugs that prohibited the \Condor{startd} use of \MacroNI{RANK} 
from preempting dynamic slots.
\Ticket{4580}

\item Using \Condor{compile} on programs which call \Procedure{posix\_memalign}
no longer causes a link error.
\Ticket{4486}

\item EC2 grid universe jobs which use the X.509 authentication method will
no longer crash if environment variable \Env{USER} is not set.
\Ticket{4540}

\item Fixed a rare memory leak. 
The leak occurred when IPv6 was disabled, 
but configuration variables \MacroNI{NETWORK\_INTERFACE} 
and \MacroNI{COLLECTOR\_HOST} were set to IPv6 addresses.
\Ticket{4502}

\item Fixed a bug in which \Condor{qsub} mishandled setting a memory request 
with a command line argument similar to \Opt{-l mem=2048MB}.
\Ticket{4549}

\item Fixed a bug that caused the \Condor{gridmanager} to fail to talk
to the \Condor{schedd} if the user's account was in a Windows domain.
\Ticket{4568}

\item On Windows platforms, users listed in the \Macro{QUEUE\_SUPER\_USERS} 
configuration variable are now checked in a case-insensitive way,
since user names are case-insensitive on Windows.
\Ticket{4579}

\item Fixed a bug that could prevent the \Condor{schedd} job queue log 
from rotating on Windows platforms.
\Ticket{4548}

\item Fixed a bug that caused all HTCondor daemons to leak 
a small amount of memory upon reconfiguration.
\Ticket{4582}

\item Fixed a bug that caused \Condor{config\_val} \Opt{-verbose} to sometimes append incorrect meta-knob
information to the file and line number information for a configuration variable.
\Ticket{4559}

\item Fixed a bug that sometimes prevented adding a \File{.txt} file name
extension to the log file name of an HTCondor daemon on Windows platforms.
\Ticket{4571}

\item Fixed a bug that caused \Condor{dagman} to crash if
configuration variable
\Macro{DAGMAN\_ALWAYS\_USE\_NODE\_LOG} was set to \Expr{False} and
configuration variable
\Macro{DAGMAN\_USE\_STRICT} was set to 1 or a higher value.
\Ticket{4600}

\item Fixed a bug that caused the DAG node status file (if one is specified)
to have the wrong final status for a DAG that is aborted by an
\Arg{ABORT-DAG-ON} specification.
\Ticket{4312}

\item Fixed a bug in the \Prog{batch\_gahp} that could cause it to fail
when attempting to query the status of an LSF job.
\Ticket{4592}

\end{itemize}

\noindent Known Bugs:

\begin{itemize}

\item On Windows platforms only, issuing \Condor{rm} on a 
\Condor{dagman} job does not work correctly.
The \Condor{dagman} process is immediately killed,
and it does not write a Rescue DAG or remove its node jobs.
Note that this bug has probably existed since DAGMan was first
implemented on the Windows platform.
\Ticket{4566}

\end{itemize}

%%%%%%%%%%%%%%%%%%%%%%%%%%%%%%%%%%%%%%%%%%%%%%%%%%%%%%%%%%%%%%%%%%%%%%
\subsection*{\label{sec:New-8-2-2}Version 8.2.2}
%%%%%%%%%%%%%%%%%%%%%%%%%%%%%%%%%%%%%%%%%%%%%%%%%%%%%%%%%%%%%%%%%%%%%%

\noindent Release Notes:

\begin{itemize}

\item HTCondor version 8.2.2 released on August 7, 2014.

\item This version of HTCondor includes a full port for 
Red Hat Enterprise Linux 7.0 on the x86\_64 architecture.
A full port includes support for the standard universe.
\Ticket{4511}

\item The RPM for RHEL 7 contains several subpackages for elements of HTCondor,
modernizing the RPM-based installation.
\Ticket{4518}

\end{itemize}

\noindent New Features:

\begin{itemize}

\item None.

\end{itemize}

\noindent Bugs Fixed:

\begin{itemize}

\item When using the Windows installer,
the choice of a new pool caused an invalid value in the configuration of 
\Expr{\$\$(FULL\_HOSTNAME)} to be used, 
instead of the correct value of \Expr{\$(FULL\_HOSTNAME)}.
This prevented all daemons from talking to the \Condor{collector} daemon.
\Ticket{4509}

\item Fixed a bug that only manifested on Linux 3.14 or more recent kernels, 
which caused the \Condor{collector} to respond very slowly to queries.
\Ticket{4489}

\item Fixed a Windows platform bug that caused \Condor{status} to abort
when \Macro{ENABLE\_CLASSAD\_CACHING} was set to \Expr{True}. 
\Ticket{4459}

\item Fixed a bug that prevented the detection of hyper-threaded cores
on Linux platforms.
All cores reported as full cores without hyper-threading. 
\Ticket{4458}

\item Fixed the detection of hyper-threaded cores on Mac OS X platforms.
\Ticket{4516}

\item Fixed a Windows platform bug that caused the \Condor{starter}
to abort while creating the job sandbox.
The bug presents as a minor memory leak in all versions of HTCondor 
for Windows prior to version 8.2.2 and 8.3.0.
In HTCondor version 8.2.0, this bug could sometimes
present as an abrupt \Condor{starter} exit with status -1073740940. 
\Ticket{4467}

\item Fixed a file descriptor leak in the \Condor{shared\_port}
daemon.
\Ticket{4456}

\item Fixed a bug existing on Linux platforms with newer kernels.
With cgroups enabled, the OOM killer killed the job when the job
went over its memory allocation.  
Now, the \Condor{starter} catches the OOM signal and 
places the job on hold with an appropriate message.
\Ticket{4435}

\item Fixed a bug in which the expression set by submit command 
\SubmitCmd{periodic\_remove} would not remove
jobs running on Linux machines when PID namespaces were enabled.
\Ticket{4421}

\item Fixed a Windows-specific bug:  specifying a DAG node status file
caused DAGMan to fail.
\Ticket{4361}

\item Fixed a problem in which job rank may not have always worked
as documented due to a bug in HTCondor's auto cluster mechanism.
\Ticket{4403}

\item Updated the HTCondor DRMAA library to version 1.6.2. 
This version fixes minor bugs in the functions for querying how a job exited.
\Ticket{4413}

\item \Condor{submit} no longer fails if the value of
\SubmitCmd{x509userproxy} is a relative path, 
and the value of \SubmitCmd{initialdir} is set to a directory 
that is not the current working directory of \Condor{submit}.
\Ticket{4415}

\item Fixed a bug that caused \Condor{submit\_dag} to core dump if
a non-existent DAG file was specified.
\Ticket{4423}

\item Fixed a bug that resulted in output of the string \AdStr{undefined}, 
instead of printing nothing,
when using the \Opt{\%s} format specifier to
\Condor{q} \Opt{-format}.
\Ticket{4418}

\item Fixed a bug in the \Condor{shadow} that caused the user's supplemental
groups to be unset when trying to write to the user's job event log.
This could result in the job being held with the hold reason
\Expr{"Failed to initialize user log to <path>"}.
\Ticket{4437}

\item Fixed a bug in the \Prog{cream\_gahp} that would corrupt memory when
using more than the default number of worker threads.
\Ticket{4416}

\item Fixed a bug that could cause the \Prog{cream\_gahp} to fail at
start up, because it could not locate a Globus threading library.
\Ticket{4440}

\item When a daemon checks whether a user has execute permission for a
directory, it now considers supplemental groups and POSIX ACLs in the
determination.
\Ticket{4402}

\item Fixed a bug that could cause GSI security operations to fail if
\Env{GLOBUS\_THREAD\_MODEL} was set in the environment.
\Ticket{4464}

\item Fixed a bug in \Condor{ft-gahp} that caused it to ignore the peer
version given by the \Expr{CONDOR\_VERSION} command, causing it to think that
its file transfer peer was the same version as itself.
\Ticket{4473}

\item Fixed the handling of optional authentication parameters given to
\Prog{remote\_gahp}. This is used as part of the batch grid-type when
submitting jobs to a remote system via \Prog{ssh}.
\Ticket{4434}

\item  Fixed a bug in the parsing the value set for the
\Attr{Detected<Tag>} attribute of the output of a script specified by
configuration variable \Macro{MACHINE\_RESOURCE\_INVENTORY\_<TAG>}.
If the value of \Attr{Detected<Tag>} was not a string, 
then it would not be parsed correctly. 
As a result the resource quantity would be set to 0.
\Ticket{4427}

\end{itemize}

%%%%%%%%%%%%%%%%%%%%%%%%%%%%%%%%%%%%%%%%%%%%%%%%%%%%%%%%%%%%%%%%%%%%%%
\subsection*{\label{sec:New-8-2-1}Version 8.2.1}
%%%%%%%%%%%%%%%%%%%%%%%%%%%%%%%%%%%%%%%%%%%%%%%%%%%%%%%%%%%%%%%%%%%%%%

\begin{itemize}
\item HTCondor version 8.2.1 released on July 7, 2014.

\item \Security
This release of HTCondor fixes a security-related bug described at 
\URL{http://research.cs.wisc.edu/htcondor/security/vulnerabilities/HTCONDOR-2014-0001.html}.
\Ticket{4420}

\end{itemize}

\noindent New Features:

\begin{itemize}

\item None.

\end{itemize}

\noindent Bugs Fixed:

\begin{itemize}

\item None.

\end{itemize}

%%%%%%%%%%%%%%%%%%%%%%%%%%%%%%%%%%%%%%%%%%%%%%%%%%%%%%%%%%%%%%%%%%%%%%
\subsection*{\label{sec:New-8-2-0}Version 8.2.0}
%%%%%%%%%%%%%%%%%%%%%%%%%%%%%%%%%%%%%%%%%%%%%%%%%%%%%%%%%%%%%%%%%%%%%%

\noindent Release Notes:

\begin{itemize}

\item HTCondor version 8.2.0 released on June 24, 2014.

\end{itemize}


\noindent New Features:

\begin{itemize}

\item The new configuration variable \Macro{SOCKET\_LISTEN\_BACKLOG}
controls the listen backlog setting for a daemon's command port.
The default value of 500 implements the previously hard coded value.
\Ticket{4393}

\item Streamlined the network protocol used by \Condor{submit},
resulting in faster job submission times and less \Condor{schedd} overhead, 
especially when performing a submit to a remote \Condor{schedd}.
\Ticket{3846}

\item The default value for configuration variable \Macro{CLAIM\_WORKLIFE} 
has changed from 60 minutes to 20 minutes.
\Ticket{4374}

\item The default value for configuration variable 
\Macro{NEGOTIATOR\_PRE\_JOB\_RANK} has changed to prefer to match
multi-core jobs to dynamic slots in a best-fit manner.
And, the default value for configuration variable
\Macro{PREEMPTION\_RANK} has changed to first choose the user with the
worst priority, and then choose the job of that user with the least
amount of accumulated run time. 
\Ticket {4374}

\item The default set of metrics published by the \Condor{gangliad} has been
reduced to an essential set of scheduler and negotiator metrics.
Also, the units for accumulated times have changed from seconds to hours.
\Ticket{4299}

\end{itemize}

\noindent Bugs Fixed:

\begin{itemize}

\item Fixed a bug that caused a memory leak in the \Condor{procd}
when cgroup tracking is enabled.
\Ticket{4408}

\item Fixed a bug that caused a memory leak in the \Condor{collector}
under heavy load.  This bug was introduced in HTCondor version 8.1.5.
\Ticket{4370}

\item Windows machines with more than nine dynamic slots may have
failed to start jobs due to a limit on the number of characters
in a user name.
To address this limit, the user name is shortened from
\Expr{condor-reuse-slot<N>} to \Expr{condor-slot<N>}.
\Ticket{4388}

\item Fixed a bug in which \Condor{q} failed to communicate with a
\Condor{schedd} of HTCondor version 8.1.4.
\Ticket{4384}

\item Fixed bugs introduced in HTCondor version 8.1.5 that caused communication
between the \Prog{cream\_gahp} and the remote CREAM servers to fail.
\Ticket{4392}

\item Fixed a bug introduced in HTCondor version 8.1.2 that caused grid-type
cream jobs to fail when \SubmitCmd{copy\_to\_spool} was set to \Expr{True}
in the submit description file.
\Ticket{4391}

\item When submitting a grid universe job with a grid type of batch and
setting \SubmitCmd{request\_memory}, the job would fail if the remote
batch system was HTCondor. This has been fixed.
\Ticket{4367}

\item Improved the detection of IPv4 link-local addresses.
\Ticket{4341}

\item Fixed a bug in which the HTCondor central manager may attempt to
send email to a user named \Expr{NONE}, if configuration variable
\MacroNI{CONDOR\_DEVELOPERS} is left unset.
\Ticket{4399}

\item Fixed a bug in which \Condor{user\_prio} could result in a
segmentation fault when given the \Opt{-grouporder} option.
\Ticket{4407}

\item Fixed a bug that caused frequent crashes of the \Prog{cream\_gahp}.
\Ticket{4406}

\item Fixed a bug that prevented attribute \AdAttr{SubmitterUserPrio} from
	properly functioning in \MacroNI{PREEMPTION\_REQUIREMENTS} and 
	\MacroNI{PREEMPTION\_RANK} expressions as documented in
	section~\ref{sec:Priorities-in-Negotiation-and-Preemption}.
\Ticket{4369}

\item Fixed a bug that could cause some commands sent to HTCondor daemons
to fail, especially when sent over a slow network.
This bug was introduced in HTCondor version 8.1.5.
\Ticket{4368}

\end{itemize}


%%%      PLEASE RUN A SPELL CHECKER BEFORE COMMITTING YOUR CHANGES!
%%%      PLEASE RUN A SPELL CHECKER BEFORE COMMITTING YOUR CHANGES!
%%%      PLEASE RUN A SPELL CHECKER BEFORE COMMITTING YOUR CHANGES!
%%%      PLEASE RUN A SPELL CHECKER BEFORE COMMITTING YOUR CHANGES!
%%%      PLEASE RUN A SPELL CHECKER BEFORE COMMITTING YOUR CHANGES!

%%%%%%%%%%%%%%%%%%%%%%%%%%%%%%%%%%%%%%%%%%%%%%%%%%%%%%%%%%%%%%%%%%%%%%
\section{\label{sec:History-8-1}Development Release Series 8.1}
%%%%%%%%%%%%%%%%%%%%%%%%%%%%%%%%%%%%%%%%%%%%%%%%%%%%%%%%%%%%%%%%%%%%%%

This is the development release series of HTCondor.
The details of each version are described below.


%%%%%%%%%%%%%%%%%%%%%%%%%%%%%%%%%%%%%%%%%%%%%%%%%%%%%%%%%%%%%%%%%%%%%%
\subsection*{\label{sec:New-8-1-6}Version 8.1.6}
%%%%%%%%%%%%%%%%%%%%%%%%%%%%%%%%%%%%%%%%%%%%%%%%%%%%%%%%%%%%%%%%%%%%%%

\noindent Release Notes:

\begin{itemize}

\item HTCondor version 8.1.6 released on May 22, 2014.

\end{itemize}


\noindent New Features:

\begin{itemize}

\item HTCondor can discover, schedule, and manage GPUs in an
exceedingly simple way by inserting
\begin{verbatim}
  use feature : GPUs
\end{verbatim}
in the configuration file.
The HTCondor wiki page, 
\URL{https://htcondor-wiki.cs.wisc.edu/index.cgi/wiki?p=HowToManageGpus},
describes the capabilities.

\item The grid universe can now be used to submit and manage jobs on
a BOINC server, using the new grid type \SubmitCmd{boinc}.
\Ticket{3540}

\item Configuration has been enhanced in structure and with
newly implemented semantics describing configuration.
As part of this effort, most all configuration variables have
compile-time defaults specified and incorporated into the code.
Therefore, they no longer appear in the example, distributed
configuration file.
It is only when values change that these variables will be placed
into a configuration file.
For current installations wishing to transition to the new, stripped down
configurations files, 
the new \Opt{-writeconfig} option to \Condor{config\_val} will
help to identify values different from defaults.
New configuration semantics permit
\begin{itemize}
  \item the inclusion of configuration defined elsewhere.
  See section~\ref{sec:Config-Include} for a description.
  \item metaknobs, which incorporate predefined sets of configuration
  that are commonly used.
  See section~\ref{sec:Config-Metaknobs} for a description.
  \item a simple if/else syntax for conditional specification of 
  configuration. 
  See section~\ref{sec:Config-IfElse} for a description.
\end{itemize}
\Ticket{4325}
\Ticket{3894}
\Ticket{4319}
\Ticket{4031}
\Ticket{4211}

\item When hierarchical group quotas are used, and surplus
sharing is enabled, the quotas are now correctly computed
if slot weights are also enabled.
\Ticket{4324}

\item The default priority factor set for new users is now 1000.
This was changed from a default value of 1, because a value of 1 
leaves no room to boost the priority factor.
\Ticket{4282}

\item The \Condor{schedd} may now keep open a configurable number
of job event log files.
This improves performance over the previous behavior of
open, write, close done for each event.
New configuration variables \Macro{USERLOG\_FILE\_CACHE\_MAX} and 
\Macro{USERLOG\_FILE\_CACHE\_CLEAR\_INTERVAL} specify the number
of job event log files that may be kept open at the same time
and the periodic interval of time that passes
before the set of open files are closed.
\Ticket{4040}

\item The curl file transfer plug-in can now be used to transfer output
files in addition to input files.
\Ticket{4190}

\item New python bindings allow the user access to the same 
file locking protocol as HTCondor daemons.
\Ticket{4315}

\item The DAGMan node status file formatting has changed.
The format of the DAG node status file is now New ClassAds,
and the amount of information in the file has increased.
Section~\ref{sec:DAG-node-status} has details on node status files.
\Ticket{4115}

\item The new configuration variable \Macro{STARTER\_LOG\_NAME\_APPEND}
controls the file name extension of the log used by the \Condor{starter}.
\Ticket{4244}

\item The new configuration variable
 \Macro{ENVIRONMENT\_VALUE\_FOR\_UnAssigned<name>}
is intended for use with GPUs, where \texttt{<name>} is \texttt{GPUs}. 
It defines what GPU ID to assign to slots that have no assigned GPU.
Without this, the CUDA runtime would allow slots with no assigned GPU to use
all of the GPUs.
\Ticket{4320}

\item The batch system name \texttt{HTCondor} is now published in 
each job's environment.
\Ticket{4233}

\item New configuration variables \Macro{UDP\_NETWORK\_FRAGMENT\_SIZE} and
	\Macro{UDP\_LOOPBACK\_FRAGMENT\_SIZE} added to control UDP message
	fragmentation size over the network and loopback interface, 
	respectively.
\Ticket{4321}

\item The new \Condor{pool\_job\_report} tool for Linux platforms
composes and mails a report about all jobs run in the previous
24 hours on all execute machines within the pool. 
\Ticket{4267}

\item HTCondor now publishes more I/O statistics as job ClassAd attributes.
The new attributes are
\Attr{BlockReads},
\Attr{BlockWrites},
\Attr{RecentBlockReads},
\Attr{RecentBlockWrites},
\Attr{RecentBlockReadKbytes}, and
\Attr{RecentBlockWriteKbytes}.
\Ticket{3850}

\item The new job ClassAd attribute \Attr{SpoolOnEvict} facilitates
the debugging of failed jobs.
\Ticket{4292}

\item Memory corruption mitigation is enabled by additional linker flags,
when building HTCondor from source against system-shared
libraries installed by the distribution.
\Ticket{4153}

\item An experimental new feature to overlap the transfer of job output
with the execution of a subsequent job is documented with a link from
the HTCondor wiki page, 
\URL{https://htcondor-wiki.cs.wisc.edu/index.cgi/wiki?p=ExperimentalFeatures}.
\Ticket{4291}

\item An experimental new feature to provide custom output formatting
for \Condor{q} and \Condor{status} is documented with a link from
the HTCondor wiki page, 
\URL{https://htcondor-wiki.cs.wisc.edu/index.cgi/wiki?p=ExperimentalFeatures}.
\Ticket{4241}

\end{itemize}

\noindent Bugs Fixed:

\begin{itemize}

\item The \Condor{shared\_port} daemon no longer blocks
on a very unresponsive daemon.
\Ticket{4314}

\item vm universe jobs now report attribute \Attr{RemoteUserCPU} when
run on a KVM hypervisor.
CPU usage remains unreported by VMware hypervisors.
\Ticket{4337}

\item The \Condor{gridmanager} no longer assumes that a NorduGrid ARC job
with a reported exit code greater than 128 exited abnormally via a signal.
\Ticket{4342}

\item Many tools, including \Condor{off} and \Condor{restart} interpreted
the command line argument \Opt{-defrag} incorrectly as \Opt{-debug},
since both words start with the string \AdStr{de}.
The confusion has been fixed. 
Use of \Opt{-defrag} will now produce an error message, 
since it is not a valid option for these tools.
\Ticket{3717}

\item Fixed a crash by the \Condor{gpu\_discovery} tool,
when running on a 32-bit platform or on Windows and detecting via OpenCL.
\Ticket{4339}

\end{itemize}

%%%%%%%%%%%%%%%%%%%%%%%%%%%%%%%%%%%%%%%%%%%%%%%%%%%%%%%%%%%%%%%%%%%%%%
\subsection*{\label{sec:New-8-1-5}Version 8.1.5}
%%%%%%%%%%%%%%%%%%%%%%%%%%%%%%%%%%%%%%%%%%%%%%%%%%%%%%%%%%%%%%%%%%%%%%

\noindent Release Notes:

\begin{itemize}

\item HTCondor version 8.1.5 released on April 15, 2014.

\end{itemize}


\noindent New Features:

\begin{itemize}

\item The default configuration now implements a policy 
that disables preemption.
\Ticket{4281}

\item The protocol for interaction between \Condor{q} and the 
\Condor{schedd} daemon has been rewritten.
The new protocol does not require the \Condor{schedd} to fork a child process 
and does not cause blocking; 
the result is that the \Condor{schedd} should be able to handle
many concurrent \Condor{q} requests with minimal resource usage.
\Ticket{4111}

\item The specification in configuration for the size or amount of time
that a log file may grow has changed.
An explicit size or amount of time may still be specified for any
individual log file.
However, any log files not explicitly specified have a default maximum
size specified by the new configuration variable 
\Macro{MAX\_DEFAULT\_LOG}.
\Ticket{4246}

\item The new \Condor{urlfetch} tool is enables the  acquisition of
configuration with a query to a URL.
\Ticket{4018}

\item The \Prog{cream\_gahp} and \Prog{nordugrid\_gahp} can now talk to
servers over IPv6.
\Ticket{4243}

\item The python bindings can now accept a list of \Condor{collector} hosts
in the constructor of the \texttt{Collector} object.  
This eases use of the bindings for high availability setups.
\Ticket{4245}

\item The new python binding \texttt{transaction} creates a transaction
with the \Condor{schedd},
providing a way to submit multiple clusters of jobs
or edit multiple attributes atomically.
\Ticket{4225}

\item New configuration variable \Macro{NEGOTIATOR\_MAX\_TIME\_PER\_CYCLE}
places an upper time limit on the time spent in each negotiation cycle.
\Ticket{4271}

\item The configuration variable \Macro{VALID\_SPOOL\_FILES} has been redefined
to list only files that the system administrator determines must not
be removed by \Condor{preen}.
The new configuration variable \Macro{SYSTEM\_VALID\_SPOOL\_FILES} contains 
a predetermined list of files that are known to be valid at 
the time HTCondor was built. 
\Condor{preen} will use the union of these two configuration variables 
as the set of valid files that should not be removed from the \MacroNI{SPOOL}
directory.
\Ticket{4257}

\item The new configuration variable \Macro{OFFLINE\_MACHINE\_RESOURCE\_<name>}
is used to identify a custom machine resource as offline,
so that the resource will not be allocated to any slot.
\Ticket{4177}

\item The default value of configuration variable 
\Macro{NEGOTIATOR\_USE\_WEIGHTED\_DEMAND} has been changed from 
\Expr{False} to \Expr{True}.
\Ticket{4238}

\item The new configuration variable 
\Macro{NEGOTIATOR\_TRIM\_SHUTDOWN\_THRESHOLD} can be used to avoid 
making matches to resources that are about to go away. 
It is primarily of interest to glidein pools.  
Section~\ref{param:NegotiatorTrimShutdownThreshold} details the new
configuration variable.
\Ticket{4266}

\item No user-visible changes result from reductions in the quantity of
unused memory within DaemonCore data structures.
\Ticket{4206}

\item The \Condor{negotiator} logs more information about its round robin
iteration to ease debugging.
\Ticket{3871}

\item Some communications between daemons will cause fewer network timeouts,
as the reading of commands no longer blocks while
waiting for completion of the command.
\Ticket{4237}

\end{itemize}

\noindent Bugs Fixed:

\begin{itemize}

\item Fixed a bug that affected \Condor{on}, \Condor{off}, \Condor{restart},
\Condor{reconfig}, and \Condor{set\_shutdown}. 
When multiple machines were named on the command line, 
these tools could report 
\begin{verbatim}
Can't find address for master XXXX 
\end{verbatim}
for some daemons,
even though the daemons were properly advertised to the \Condor{collector}.
\Ticket{4207}

\item Fixed a bug that could have caused the \Condor{startd} to become 
unresponsive when starting a job obtained via the Work Fetch Hook.
\Ticket{4210}

\item Fixed a bug that could have caused the \Condor{schedd} to advertise a 
stale address in the \Attr{ScheddIpAddr} attribute of its submitter ClassAds,
resulting in other daemons being unable to contact it.
The problem occurred when using both the \Condor{shared\_port} daemon and CCB,
and the value of configuration variable \Macro{CCB\_ADDRESS} was changed.
\Ticket{4250}

\item Fixed a bug introduced earlier in the 8.1 developer series that 
could cause \Condor{submit} to crash when reading 
large submit description files.
\Ticket{4260}

\item Fixed a bug that prevented a configuration variable 
from referring to itself,
when the previous value was defined by the code,
rather than within a configuration file.
\Ticket{4256}

\item The temperature attributes output by the \Condor{gpu\_discovery} tool
contained values represented in Celsius, while
the names of these attributes ended in the letter 'F,' implying Fahrenheit.
The names of these attributes have been changed to end with the letter 'C.' 
For instance \Attr{<name>DieTempF} has been changed to \Attr{<name>DieTempC}.
\Ticket{4294}

\item The \Condor{startd} no longer generates this erroneous message
when a plugin can not be run:
\begin{verbatim}
Warning: Starter pid XXX is not associated with a claim.
A slot may fail to transition to Idle.
\end{verbatim}
\Ticket{4026}

\end{itemize}

%%%%%%%%%%%%%%%%%%%%%%%%%%%%%%%%%%%%%%%%%%%%%%%%%%%%%%%%%%%%%%%%%%%%%%
\subsection*{\label{sec:New-8-1-4}Version 8.1.4}
%%%%%%%%%%%%%%%%%%%%%%%%%%%%%%%%%%%%%%%%%%%%%%%%%%%%%%%%%%%%%%%%%%%%%%

\noindent Release Notes:

\begin{itemize}

\item HTCondor version 8.1.4 released on February 27, 2014.

\item This version of HTCondor includes all bug fixes from version 8.0.6,
as well as the new full port for the Red Hat Enterprise Linux 7.0 \emph{Beta} 
release on the x86\_64 architecture.
A full port includes support for the standard universe. 

\end{itemize}


\noindent New Features:

\begin{itemize}

\item When configured to use partitionable slots,
those slots running jobs can now be preempted by the 
\Condor{negotiator} daemon based on the value of 
the machine's configuration of \MacroNI{RANK}.
\Ticket{3667}

\item Improved support for publishing monitoring information about an
HTCondor pool to \TM{Ganglia}.
Added Ganglia statistics for total job starts and total job preemptions
within a \Condor{startd}.
This allows Ganglia to graph the total job preemptions
across all \Condor{startd} daemons in a pool.
See section~\ref{sec:Config-gangliad} for configuration variable definitions,
and section~\ref{sec:monitor-ganglia} for details about monitoring
with Ganglia.
\Ticket{4151}
\Ticket{3965}

\item The grid universe can now be used to create and manage VM instances
in Google Compute Engine (GCE), using the new grid type \SubmitCmd{gce}.
\Ticket{3833}

\item As a scalability improvement for Unix platforms, 
the \Condor{shared\_port} daemon no longer forks on incoming connections.
\Ticket{4094}

\item \Condor{ssh\_to\_job} and interactive jobs no longer try to 
connect to held jobs.
They instead report the hold and the reason why the job is being held.
\Ticket{3867}

\item Improved the restart time of the \Condor{schedd} after it has crashed.
\Ticket{4169}

\item The new configuration variable \Macro{EC2\_RESOURCE\_TIMEOUT} sets
the amount of time that HTCondor will wait for an unresponsive EC2 service 
before placing the corresponding jobs on hold.
\Ticket{4113}

\item The new python binding \Procedure{refreshGSIProxy}
can refresh a remote job's GSI proxy as a part of the \texttt{Schedd} object.
\Ticket{4116}

\item By default, 
the TCP keep alive interval is automatically tuned to 5 minutes.  
This causes at least one packet to be sent on established,
but idle, TCP connections once every 5 minutes, 
and it speeds up the detection of connections that were silently dropped 
by NAT or firewall devices.
Without this, 
the \Condor{shadow} may not reliably recover from transient network failures.
This behavior is controlled by the new configuration variable
\Macro{TCP\_KEEPALIVE\_INTERVAL}.
Setting this variable to 0 restores the prior behavior.
In addition, the configuration variable \Macro{CCB\_HEARTBEAT\_INTERVAL} 
default value has been reduced to 5 minutes.
\Ticket{4122}

\item New python \Code{ClassAd} module function calls 
\Procedure{Attribute}, \Procedure{Function}, \Procedure{Literal},
\Procedure{flatten}, \Procedure{matches}, and \Procedure{symmetricMatch}
aid the composition of ClassAd expressions.
It should now be possible to build expressions directly
in python, without having to resort to string manipulation.
\Ticket{4154}

\item For those that use the Python bindings,
the \Env{LD\_LIBRARY\_PATH} environment variable no longer needs to be set.
\Ticket{4128}

\item The Python bindings are now compatible with Python 3.
\Ticket{4146}


\item Setting configuration variable
\Macro{DAGMAN\_ALWAYS\_USE\_NODE\_LOG} to \Expr{False}
or using the corresponding \Opt{-dont\_use\_default\_node\_log} option
to \Condor{submit\_dag} is no longer recommended.
It is no longer recommended to have \Condor{dagman} read the log files 
specified in the node job submit description files.
\Ticket{4091}

\item Invoking \Condor{fetchlog} with the \Arg{STARTD\_HISTORY} argument
now fetches all \Condor{startd} history by concatenating all instances 
of log files resulting from rotation to the current history log.
\Ticket{4152}

\item Several general mechanisms for specifying user-defined \Condor{startd} 
resources have been enhanced,
so that GPUs can be easily defined and used.
New to this 8.1.4 version of HTCondor is the allocation of user defined
resources (especially GPUs) with partitionable and dynamic slots.
This includes having HTCondor automatically set the environment variable
\Env{CUDA\_VISIBLE\_DEVICES} for jobs that use CUDA GPUs
and \Env{GPU\_DEVICE\_ORDINAL} for jobs that use OpenCL GPUs.

The mechanism defines configuration variables 
\Macro{MACHINE\_RESOURCE\_<name>} and 
\Macro{MACHINE\_RESOURCE\_INVENTORY\_<name>}
to specify the definition user-defined resources with a list of resource
identifiers.  
When HTCondor allocates one of these user-defined resources to a slot, 
it will also publish this assignment within the slot's ClassAd 
using the new job ClassAd attribute \Attr{Assigned<name>}.
And, it will define in the job's environment the variable
\Env{\_CONDOR\_Assigned<name>}.
The new configuration variable \Macro{ENVIRONMENT\_FOR\_Assigned<name>}
also sets further environment variables.
\Ticket{4141}
\Ticket{4148}

\item The new \Condor{gpu\_discovery} tool detects CUDA and OpenCL GPUs,
reporting them in the format needed to configure GPU resources 
using the configuration variable
\Macro{MACHINE\_RESOURCE\_INVENTORY\_GPUs}.
\Ticket{3386}

\item Two new pre-defined configuration variables are referenced with
\MacroU{DETECTED\_PHYSICAL\_CPUS} and \MacroU{DETECTED\_CPUS}.
\MacroUNI{DETECTED\_PHYSICAL\_CPUS} contains the number of 
physical (non-hyperthreaded) CPUs. 
\MacroUNI{DETECTED\_CPUS} will match the value of
either \MacroNI{DETECTED\_CORES} or \MacroNI{DETECTED\_PHYSICAL\_CPUS}, 
depending on the state of \Macro{COUNT\_HYPERTHREAD\_CPUS}.
The default value of \Macro{NUM\_CPUS} now defaults to the value
of \MacroNI{DETECTED\_CPUS}.
\Ticket{4197}

\item \Condor{q} will now show the macro-expanded job description from the attribute
\Attr{MATCH\_EXP\_JobDescription} instead of \Attr{JobDescription} if it is available.
\Ticket{4110}

\end{itemize}

\noindent Bugs Fixed:

\begin{itemize}

\item Fixed a small memory leak that was triggered by failed
file transfer attempts.
\Ticket{4134}

\item Fixed a bug that would leak one socket in each daemon,
when \Expr{NO\_DNS = True}.
\Ticket{4140}

\item Changed the way the \Condor{startd} allocates CPUs to
slots in configurations where there are more slots than CPUs.
CPUs are now distributed equally between slots that are not configured
to receive a specific number 
(using configuration variable \Macro{SLOT\_TYPE\_<N>}).
Before this change, these slots received 1 CPU each.
The new behavior matches how other slot resources are distributed.
\Ticket{3249}

\item The failure to terminate an EC2 grid universe job instance,
because the instance no longer exists at the service, 
is now considered a successful termination.  
This allows EC2 grid universe jobs to exit the queue, 
if the service purges termination records quickly.
\Ticket{4133}

\item HTCondor now interacts with EC2 services by using \Code{POST}
instead of \Code{GET},
which permits more services to accept user data with size greater than 8Kbytes.
\Ticket{4004}

\item Improved the handling of the \SubmitCmd{coresize} 
submit description file command,
by allowing values larger than 4Gbytes.
\Ticket{4155}

\item Fixed a bug that caused job arguments to not be displayed in the
default output of \Condor{q} when the submit description file used the
new syntax for job arguments.
\Ticket{2875}

\item The \Condor{startd} daemon will no longer abort when it exhausts
the supply of user-defined resources such as GPUs 
while assigning automatic resource shares to slots.
\Ticket{4176}

\end{itemize}

%%%%%%%%%%%%%%%%%%%%%%%%%%%%%%%%%%%%%%%%%%%%%%%%%%%%%%%%%%%%%%%%%%%%%%
\subsection*{\label{sec:New-8-1-3}Version 8.1.3}
%%%%%%%%%%%%%%%%%%%%%%%%%%%%%%%%%%%%%%%%%%%%%%%%%%%%%%%%%%%%%%%%%%%%%%

\noindent Release Notes:

\begin{itemize}

\item HTCondor version 8.1.3 released on December 23, 2013.
This developer release contains all bug fixes from HTCondor version 8.0.5.

\end{itemize}


\noindent New Features:

\begin{itemize}

\item The parsing of configuration has changed with respect to how
line continuation characters and comments interact.
The line continuation character no longer takes precedence over the
comment character.
\Ticket{4027}

\index{SUBSYS\_SUPER\_ADDRESS\_FILE macro@\texttt{<SUBSYS>\_SUPER\_ADDRESS\_FILE} macro}
\index{configuration macro!\texttt{SUBSYS\_SUPER\_ADDRESS\_FILE}}
\item When the super user issues a command 
or when the new \Condor{sos} tool invokes another tool,
the command can be serviced with a higher priority. 
This should be useful when attempting to get information from an
overloaded daemon, in order to diagnose or fix a problem.
Commands directed at the \Condor{schedd} or \Condor{collector} daemons 
have this ability by default.
Other DaemonCore daemons require configuration using the new 
configuration variable
\MacroB{<SUBSYS>\_SUPER\_ADDRESS\_FILE}.
\Ticket{4029}

\item The dedicated scheduler cpu usage within the \Condor{schedd} is now
throttled, so that it cannot consume all of the cpu, while starving the vanilla
scheduler.  This throttle can be adjusted by the new configuration variable
\Macro{DEDICATED\_SCHEDULER\_DELAY\_FACTOR}.  
This variable, which defaults to five,
sets the ratio of time spent not in the dedicated scheduler to the 
time scheduling parallel jobs.  
With this default of five, 
a maximum of 20\% of the scheduler's time will go to scheduling
parallel jobs.
\Ticket{4048}

\item The new \Condor{defrag} daemon ClassAd attribute 
\Attr{MeanDrainedArrived}
measures the mean time between arrivals of fully drained machines,
and the new attribute \Attr{DrainedMachines} 
measures the total numbers of fully drained machines
which have arrived during the run time of this \Condor{defrag} daemon.
\Ticket{4055}

\item The new \Opt{-defrag} option for \Condor{status} queries ClassAds
of the \Condor{defrag} daemon.
\Ticket{4039}

\item Machine ClassAd attributes \Attr{ExpectedMachineQuickDrainingCompletion}
and \Attr{ExpectedMachineGracefulDrainingCompletion} are updated with their
completion times if there are no active claims,
making these attributes more useful in setting policy for
partitionable slots. 
\Ticket{3481}

\item In a DAG, the node retry number is now available as VARS macro
(see section~\ref{dagman:VARS}).
\Ticket{4032}

\item Macro substitution both within configuration and within submit
description files has been extended to specify and use  
an optional default value if a value is not defined.
Section~\ref{sec:Config-File-Macros} has details for configuration.
\Ticket{4033}

\item The Python bindings \Code{htcondor} module has 
a new \Procedure{read\_events} method to acquire an iterator of
an HTCondor event log file.
\Ticket{4071}

\item The new \Opt{-daemons} option to \Condor{who} prints information
about the HTCondor daemons running on the specified machine,
including the daemon's PID, IP address and command port.
\Ticket{4007}

\end{itemize}

\noindent Configuration Variable and ClassAd Attribute Additions and Changes:

\begin{itemize}

\item Configuration variable \Macro{DAGMAN\_DEFAULT\_NODE\_LOG}
has been made more powerful,
so that it can be defined in HTCondor configuration files, 
instead of being useful only when defined in a per-DAG configuration file.
See section~\ref{param:DAGManDefaultNodeLog} for details.
\Ticket{3930}

\item The new configuration variable \Macro{CORE\_FILE\_NAME} is used to set
the name that DaemonCore uses to create a core file,
in the event of a daemon crash.
The default value for this configuration variable appends the daemon name,
so a crash of the \Condor{schedd} would create a core file named
\File{core.SCHEDD}.
\Ticket{4100}

\item The new configuration variable \Macro{JOB\_EXECDIR\_PERMISSIONS}
defines the permissions on a job's scratch directory. 
It defaults to setting permissions as \emph{0700}.
\Ticket{4016}

\item The following recently added machine ClassAd attributes have been renamed.
\begin{description}
\item \Attr{TotalJobStarts} became \Attr{JobStarts}.
\item \Attr{RecentTotalJobStarts} became \Attr{RecentJobStarts}.
\item \Attr{TotalPreemptions} became \Attr{JobPreemptions}.
\item \Attr{RecentPreemptions} became \Attr{RecentJobPreemptions}.
\item \Attr{TotalRankPreemptions} became \Attr{JobRankPreemptions}.
\item \Attr{RecentTotalRankPreemptions} became \Attr{RecentJobRankPreemptions}.
\item \Attr{TotalUserPrioPreemptions} became \Attr{JobUserPrioPreemptions}.
\item \Attr{RecentTotalUserPrioPreemptions} became \Attr{RecentJobUserPrioPreemptions}.
\end{description}
\Ticket{4101}

\item The new \Condor{schedd} statistics ClassAd attribute
\Attr{Autoclusters} gives the number of active autoclusters.
\Ticket{4020}

\end{itemize}

\noindent Bugs Fixed:

\begin{itemize}

\item None.

\end{itemize}

\noindent Known Bugs:

\begin{itemize}

\item None.

\end{itemize}

\noindent Additions and Changes to the Manual:

\begin{itemize}

\item None.

\end{itemize}


%%%%%%%%%%%%%%%%%%%%%%%%%%%%%%%%%%%%%%%%%%%%%%%%%%%%%%%%%%%%%%%%%%%%%%
\subsection*{\label{sec:New-8-1-2}Version 8.1.2}
%%%%%%%%%%%%%%%%%%%%%%%%%%%%%%%%%%%%%%%%%%%%%%%%%%%%%%%%%%%%%%%%%%%%%%

\noindent Release Notes:

\begin{itemize}

\item HTCondor version 8.1.2 released on October 31, 2013.
This 8.1.2 release contains all bug fixes from HTCondor version 8.0.4.

\end{itemize}


\noindent New Features:

\begin{itemize}

\item \Condor{config\_val} now supports \Opt{-dump} and \Opt{-verbose}
options to query configuration remotely from daemons.
\Ticket{3894}

\item The \Condor{chirp} protocol and command line tool has been
enhanced to support lower-cost, delayed updates to the job
ClassAd residing in the \Condor{schedd}; updates occur as other communications
take place, eliminating the overhead of a separate update.
These two new Chirp commands,
\Opt{set\_job\_attr\_delayed} and \Opt{get\_job\_attr\_delayed} allow the job
to send lightweight notification for events such as progress
monitoring, which need not be durable.
\Ticket{3353}

\item \Condor{history} has been enhanced to support
remote history using new \Opt{-pool} and \Opt{-name} options.
\Ticket{3897}

\item Matchmaking in the \Condor{negotiator} may be made aware of resources
available for partitionable slots.
This permits multiple jobs to be matched against a partitionable slot
during a single negotiation cycle.
The new policies discussed in Section~\ref{sec:consumption-policy}
are set using new configuration variables and are known as consumption policies.
\Ticket{3435}

\item Definition syntax for the authorization configuration variables
\Macro{ALLOW\_*} and \Macro{DENY\_*} has been expanded to permit
the specification of Unix netgroups.
See section~\ref{sec:Security-Authorization} for the syntax.
\Ticket{3859}

\item Definition syntax for the configuration variable
\Macro{QUEUE\_SUPER\_USERS} has been expanded to accept a specification
of Unix user groups.
See section~\ref{param:QueueSuperUsers} for the syntax.
\Ticket{3859}

\item To ensure that a grid universe job running at an EC2 service
terminates, 
HTCondor now checks after a fixed time interval 
that the job actually has terminated,
instead of relying on the service's potentially unreliable 
job shut down indication.
If the job has not terminated after a total of four checks,
the job is placed on hold; it does not leave the queue marked as completed.
\Ticket{3438}

\item Email alerts about file transfers taking longer than
\Macro{MAX\_TRANSFER\_QUEUE\_AGE} are now grouped together
to reduce the number of email messages that are sent.

\item Floating point values in Old ClassAds are now printed in a more
human-readable format, while retaining 64-bit double precision.
In previous versions, these values were always printed in scientific
notation.
\Ticket{3928}

\item \Condor{ssh\_to\_job} now works with grid universe jobs
which use EC2 resources.
\Ticket{1548}

\item Machine ClassAd attributes \Attr{Disk} and \Attr{TotalDisk} 
are now published as 64-bit integers,
rather than being capped at the maximum value of a 32-bit integer.
\Ticket{1784}

\item In an effort to improve scalability under heavy load, the tuning
configuration variable \Macro{MAX\_REAPS\_PER\_CYCLE} is exposed,
as defined at section~\ref{param:MaxReapsPerCycle}.
The default for this variable changed from 1 to 0.
\Ticket{3992}

\item To reduce the overwhelming quantity of per-user \Condor{schedd} 
statistics that are generated when configuration variables 
\MacroNI{SCHEDD\_COLLECT\_STATS\_FOR\_<Name>} or 
\MacroNI{SCHEDD\_COLLECT\_STATS\_BY\_<Name>} are used, 
the statistics are now published at verbosity level 2,
instead of verbosity level 1.
\Ticket{3980}

\item The Python bindings now include the \Code{Negotiator} class to
manage users and their priorities.
\Ticket{3893}

\item The Python bindings now provide automatic conversions from 
dictionaries to ClassAds,
so they can accept a dictionary directly as an argument,
rather than constructing a ClassAd from the dictionary.
\Ticket{3892}

\item The Python bindings \Code{ClassAd} module has 
\Procedure{quote} and \Procedure{unquote} 
methods to help create string literals. 
\Ticket{3900}

\item The Python bindings \Code{ClassAd} module has new
methods \Procedure{parseAds} and \Procedure{parseOldAds} 
that implement an iterator over ClassAds, in the New ClassAd and 
Old ClassAd format. 
\Ticket{3918}

\item The ordering of adding attributes to the machine ClassAd has been
changed, such that the attributes \Attr{Draining}, \Attr{DrainingRequestId},
and \Attr{LastDrainStartTime} are now added before the job retirement
is calculated.
This allows a decision about preemption to be made based on if
a machine is currently draining. 
\Ticket{3901}

\end{itemize}

\noindent Bugs Fixed:

\begin{itemize}

\item When \Macro{USE\_PID\_NAMESPACES} is \Expr{True}, 
the soft kill signal is now successfully sent to the job.
Previously, a \Condor{rm}
command of such a job would not remove the job until the
killing timeout had expired.
\Ticket{3981}

\item If a standard universe job exited without producing any
checkpoints and no checkpoint server was used, 
two spurious error messages would be logged to the \File{SchedLog},
as it tried to remove the old checkpoint images from the
non-existent checkpoint server.  
These error messages are no longer logged.
\Ticket{3919}

\item When configuration variable \Macro{STARTER\_RLIMIT\_AS} is set 
to its default value of 0, it means that there is no limit.  
This value was logged as a limit of 0Mb, leading to confusion.
Now, no message is logged in this default case.
\Ticket{3914}

\item Improved how the \Condor{schedd} notifies the \Condor{shadow}
and \Condor{gridmanager} about modifications to job ClassAds made using
\Condor{qedit}.
\Ticket{3909}

\item Grid universe jobs now use the correct executable file when
\SubmitCmd{copy\_to\_spool} is set to \Expr{True}.
Previously, the executable file named in the submit description file 
would be copied to the remote server, 
rather than the copy of the executable file stored in the spool directory.
\Ticket{3589}

\item The example configuration provided within files 
\File{condor\_config.generic} and \File{condor\_config.generic.redhat} 
has been updated to fix an inadequate expression defining 
\MacroNI{NEGOTIATOR\_POST\_JOB\_RANK} when the \Condor{startd} is 
configured to not run benchmarks, as \Attr{Kflops} would not be defined.
\Ticket{3589}

\item Fixed a Python binding crash due to a segmentation fault,
when evaluating an expression tree with an undefined reference.
The fix allows the user to define the \Code{ClassAd} scope 
within which an expression tree is evaluated.
\Ticket{3910}

\item The Python bindings now include a correct conversion of
\Code{absTime} and \Code{relTime} ClassAd literals to the 
corresponding Python types.
\Ticket{3911}

\end{itemize}


%%%%%%%%%%%%%%%%%%%%%%%%%%%%%%%%%%%%%%%%%%%%%%%%%%%%%%%%%%%%%%%%%%%%%%
\subsection*{\label{sec:New-8-1-1}Version 8.1.1}
%%%%%%%%%%%%%%%%%%%%%%%%%%%%%%%%%%%%%%%%%%%%%%%%%%%%%%%%%%%%%%%%%%%%%%

\noindent Release Notes:

\begin{itemize}

\item HTCondor version 8.1.1 released on September 17, 2013.
This release contains all bug fixes from the stable release version 8.0.2.

\end{itemize}


\noindent New Features:

\begin{itemize}

\item Reduced the number of calls to the service when managing EC2 jobs. This
should increase the number of EC2 jobs HTCondor can manage on a given service
without overloading it.
\Ticket{3683}

\item When configuration variable \Macro{USE\_SHARED\_PORT} is \Expr{True},
\Macro{SHARED\_PORT} will now be automatically added to \Macro{DAEMON\_LIST}.
To disable this new behavior, set the new configuration variable:
\begin{verbatim}
  AUTO_INSERT_SHARED_PORT_IN_DAEMON_LIST = False
\end{verbatim}
\Ticket{3799}

\item Floating point values in ClassAds are now printed as 64-bit
double precision values when sent over the network, written to disk, and
displayed using the \Opt{-long} or \Opt{-autoformat} options of
\Condor{status} and \Condor{q}.
\Ticket{3363}

\item In the Pegasus/DAGMan workflow metrics,
as documented in section ~\ref{sec:DAGMetrics},
the two metrics
\Expr{dagman\_id} and \Expr{parent\_dagman\_id} are now reported
as the \Attr{ClusterId} of the \Condor{dagman} job.  This
eliminates any privacy concerns with reporting the \Condor{schedd} 
daemon's address,
which includes the submit machine's IP address.

\item The python bindings now can perform the equivalent of 
\Condor{ping} as a part of the \texttt{SecMan} object.
\Ticket{3857}

\item The \Condor{gridmanager} and \Condor{ft-gahp} now create
dynamic security session for performing file transfers.
Previously, the security configuration had to be set in a special
way for file transfers with the \Condor{ft-gahp} to work.
\Ticket{3536}

\end{itemize}

\noindent Configuration Variable and ClassAd Attribute Additions and Changes:

\begin{itemize}

\item The new configuration variable \Macro{USE\_RESOURCE\_REQUEST\_COUNTS}
is a boolean value that defaults to \Expr{True}, 
reducing the latency of negotiation 
when there are many jobs next to each other in the queue 
with the same auto cluster, and many matches are being made.
\Ticket{3585}

\item Four new machine ClassAd attributes are advertised.
\Attr{TotalJobStarts} is the total number of jobs started by 
this \Condor{startd} daemon since it booted. 
\Attr{RecentTotalJobStarts} is the number of jobs started in the
last twenty minutes.  
Similarly, \Attr{TotalPreemptions} is
the number of jobs preempted since the \Condor{startd} daemon started,
and \Attr{RecentTotalPreemptions} is 
the number in the last 20 minutes.
\Ticket{3712}

\item \Macro{FILE\_TRANSFER\_DISK\_LOAD\_THROTTLE} now accepts tabs in addition to spaces as delimiters.
\Ticket{3798}

\item Configuration variable \Macro{VALID\_SPOOL\_FILES} has been expanded
to accept a single asterisk wild card character in each listed file name.
\Ticket{3764}

\item The new configuration variable \Macro{GAHP\_DEBUG\_HIDE\_SENSITIVE\_DATA}
is a boolean value that defaults to hiding sensitive data 
such as security keys and passwords
when communication with a GAHP server is written to a daemon log.
\Ticket{3536}

\item The default value of configuration variable 
\Macro{ENABLE\_CLASSAD\_CACHING} has changed to \Expr{True} for all
daemons other than the \Condor{shadow}, \Condor{starter}, and \Condor{master}.
\Ticket{3441}

\end{itemize}

\noindent Bugs Fixed:

\begin{itemize}

\item The \Condor{gridmanager} now does proper failure recovery when
submitting EC2 grid universe jobs to services that do not support 
the EC2 ClientToken parameter.
Previously, if there was a failure when submitting jobs to OpenStack
or Eucalyptus, the jobs could be submitted twice.
\Ticket{3682}

\item Fixed the printing of nested ClassAds, so that the nested ClassAds
can be read back properly.
\Ticket{3772}

\item Fixed a bug between the \Condor{gridmanager} and \Condor{ft-gahp}
that caused file transfers to fail if one of the two daemons was older
than version 8.1.0.
\Ticket{3856}

\item Fixed a bug that caused substitution in configuration variable
evaluation to ignore per-daemon overrides. 
This is a long standing bug that may result in subtle changes
to the way your configuration files are processed.
An example of how substitution works with the per-daemon overrides
is in section \ref{sec:Config-File-Macros}.
\Ticket{3822}

\item Fixed a bug that caused the command
\begin{verbatim}
  condor_submit -
\end{verbatim}
to be interpreted as an interactive submit,
rather than a request to read input from \File{stdin}.
\Condor{qsub} was also modified to be immune to this bug,
such that it will still work with other versions of HTCondor containing
the bug.
\Ticket{3902}

\end{itemize}

\noindent Known Bugs:

\begin{itemize}
\item DAGMan recovery mode does not work for Pegasus-generated sub-DAGs.
For sub-DAGs, doing \Condor{hold} or \Condor{release} on
the \Condor{dagman} job, or stopping and re-starting the
\Condor{schedd} with the DAGMan
job in the queue will result in failure of the DAG.  This can be
avoided by doing a \Condor{rm} of the DAGMan job, which produces a Rescue
DAG, and re-submitting the DAG; the Rescue DAG is automatically run.
This bug was introduced in HTCondor version 8.0.1, and it also appears
in versions 8.0.2, 8.1.0, and 8.1.1.
\Ticket{3882}

\end{itemize}

\noindent Additions and Changes to the Manual:

\begin{itemize}

\item None.

\end{itemize}


%%%%%%%%%%%%%%%%%%%%%%%%%%%%%%%%%%%%%%%%%%%%%%%%%%%%%%%%%%%%%%%%%%%%%%
\subsection*{\label{sec:New-8-1-0}Version 8.1.0}
%%%%%%%%%%%%%%%%%%%%%%%%%%%%%%%%%%%%%%%%%%%%%%%%%%%%%%%%%%%%%%%%%%%%%%

\noindent Release Notes:

\begin{itemize}

\item HTCondor version 8.1.0 released on August 5, 2013.
This release contains all bug fixes from the stable release version 8.0.1.

\end{itemize}


\noindent New Features:

\begin{itemize}

\item Added support for publishing information about an HTCondor pool 
to \TM{Ganglia}.
See section~\ref{sec:Config-gangliad} on 
page~\pageref{sec:Config-gangliad} for configuration variable details.
\Ticket{3515}

\item Improved the performance of the \Condor{collector} daemon when running
at sites that do not observe daylight savings time.
\Ticket{2898}

\item \Condor{q}, \Condor{rm}, \Condor{status} and \Condor{qedit} are now 
more consistent in the way they handle the \Opt{-constraint} option.
\Ticket{1156}

\item The new \Condor{dagman\_metrics\_reporter} executable
with manual page at ~\pageref{man-condor-dagman-metrics-reporter},
reports metrics for DAGMan workflows running under Pegasus.  \Condor{dagman}
now generates an output file of the relevant metrics,
as described at ~\pageref{sec:DAGMetrics}.
\Ticket{3532}

\end{itemize}

\noindent Configuration Variable and ClassAd Attribute Additions and Changes:

\begin{itemize}

\item The default value of configuration variable
\Macro{COLLECTOR\_MAX\_FILE\_DESCRIPTORS} has changed to 10240,
and the default value of configuration variable 
\Macro{SCHEDD\_MAX\_FILE\_DESCRIPTORS} has changed to 4096.
This increases the scalability of the default configuration.
\Ticket{3626}

\item The new configuration variable
\Macro{FILE\_TRANSFER\_DISK\_LOAD\_THROTTLE} enables dynamic
adjustment of the level of file transfer concurrency in order to
keep the disk load generated by transfers below a specified level.
Supporting this new feature are configuration variables
\Macro{FILE\_TRANSFER\_DISK\_LOAD\_THROTTLE\_WAIT\_BETWEEN\_INCREMENTS},
\Macro{FILE\_TRANSFER\_DISK\_LOAD\_THROTTLE\_SHORT\_HORIZON}, and
\Macro{FILE\_TRANSFER\_DISK\_LOAD\_THROTTLE\_LONG\_HORIZON}.
\Ticket{3613}

\item The following new \Condor{schedd} ClassAd attributes are for
monitoring file transfer activity:
\AdAttr{TransferQueueMBWaitingToDownload},
\AdAttr{TransferQueueMBWaitingToUpload},
\AdAttr{FileTransferDiskThrottleLevel},
\AdAttr{FileTransferDiskThrottleHigh}, and
\AdAttr{FileTransferDiskThrottleLow}.
\Ticket{3613}

\item The default value for the configuration variable
\Macro{PASSWD\_CACHE\_REFRESH} has been changed from 300 seconds to
72000 seconds (20 hours).
\Ticket{3723}

\item The new configuration variables
\Macro{DAGMAN\_PEGASUS\_REPORT\_METRICS} and
\Macro{DAGMAN\_PEGASUS\_REPORT\_TIMEOUT}
set defaults used by the new \Condor{dagman\_metrics\_reporter} executable,
which reports metrics for DAGMan jobs running under Pegasus.
\Ticket{3532}

\end{itemize}

\noindent Bugs Fixed:

\begin{itemize}

\item HTCondor version 8.0.0 had an unintended change in the Chirp 
wire protocol.
This change caused \Condor{chirp} with the \Opt{put} option
to fail when the execute node
was running HTCondor version 7.8.x or earlier versions. 
HTCondor 8.0.1 and later
versions will now send the original wire protocol, and accept either the
original protocol, or the variant that HTCondor version 8.0.0 sends.
\Ticket{3735}

\item Fixed a bug that could cause the daemons to crash on Unix platforms,
if the operating system reported that a job owner's account 
did not exist, for example due to a temporary NIS or LDAP failure.
\Ticket{3723}

\item Fixed a bug that resulted in a misleading error message when
\Condor{status} with the \Opt{-constraint} option specified a constraint 
that could not be parsed.
\Ticket{1319}

\item Fixed a typo in the output of \Condor{q},
where a period was erroneously present within a heading.
\Ticket{3703}

\end{itemize}

\noindent Known Bugs:

\begin{itemize}

\item None.

\end{itemize}

\noindent Additions and Changes to the Manual:

\begin{itemize}

\item None.

\end{itemize}



%%%      PLEASE RUN A SPELL CHECKER BEFORE COMMITTING YOUR CHANGES!
%%%      PLEASE RUN A SPELL CHECKER BEFORE COMMITTING YOUR CHANGES!
%%%      PLEASE RUN A SPELL CHECKER BEFORE COMMITTING YOUR CHANGES!
%%%      PLEASE RUN A SPELL CHECKER BEFORE COMMITTING YOUR CHANGES!
%%%      PLEASE RUN A SPELL CHECKER BEFORE COMMITTING YOUR CHANGES!

%%%%%%%%%%%%%%%%%%%%%%%%%%%%%%%%%%%%%%%%%%%%%%%%%%%%%%%%%%%%%%%%%%%%%%
\section{\label{sec:History-8-0}Stable Release Series 8.0}
%%%%%%%%%%%%%%%%%%%%%%%%%%%%%%%%%%%%%%%%%%%%%%%%%%%%%%%%%%%%%%%%%%%%%%

This is a stable release series of HTCondor.
As usual, only bug fixes (and potentially, ports to new platforms)
will be provided in future 8.0.x releases.
New features will be added in the 8.1.x development series.

The details of each version are described below.

%%%%%%%%%%%%%%%%%%%%%%%%%%%%%%%%%%%%%%%%%%%%%%%%%%%%%%%%%%%%%%%%%%%%%%
\subsection*{\label{sec:New-8-0-7}Version 8.0.7}
%%%%%%%%%%%%%%%%%%%%%%%%%%%%%%%%%%%%%%%%%%%%%%%%%%%%%%%%%%%%%%%%%%%%%%

\noindent Release Notes:

\begin{itemize}

\item HTCondor version 8.0.7 released on June 5, 2014.

\end{itemize}


\noindent New Features:

\begin{itemize}

\item The new configuration variable \Macro{ALWAYS\_USE\_LOCAL\_CKPT\_SERVER}
forces all standard universe job checkpoints to be read from a checkpoint
server running on the same machine where the job runs.  
This is only useful
when all the checkpoint servers are using a common shared file system.
\Ticket{4265}

\item The new configuration variable \Macro{JOB\_EXECDIR\_PERMISSIONS}
defines the permissions on a job's scratch directory. 
It defaults to setting permissions as \emph{0700}.
\Ticket{4208}

\item An experimental new feature allows the specification of 
\Prog{icc} on the \Condor{compile} command line,
instead of \Prog{cc} or \Prog{gcc}, to build standard universe programs 
with the Intel C compiler. 
\Ticket{4234}

\end{itemize}

\noindent Bugs Fixed:

\begin{itemize}

\item Fixed a bug  which caused the memory used by the \Condor{procd}
grow arbitrarily in size when cgroup-based tracking was in effect.
\Ticket{4343}

\item Fixed a bug in which any job running on an execute node with a
process name that had a space character caused the 
\Condor{procd} to log a short read error repeatedly to the \File{ProcLog}.
\Ticket{4235}

\item Fixed a bug which would cause the \Condor{schedd} to
repeatedly crash.
This occurred when parallel universe jobs were running
as the \Condor{schedd} was restarted, 
and if the \Condor{startd} daemons in use by running jobs were 
also shut down while the \Condor{schedd} was not running.
\Ticket{4305}

\item Fixed the parsing of the \Condor{q} command line,
such that it looks for the entire option \Opt{-stream-results},
instead of only the portion \Opt{-stream}.
The code now matches what was previously documented in the manual page
and in the usage output.
\Ticket{4205}

\item Fixed a bug in the grid universe that could allow a removed job to
leave the queue without cleaning up all job state on the remote server.
\Ticket{4216}

\item Fixed the printing of nested ClassAds, so that the nested ClassAds
can be read back properly.
\Ticket{3772}

\item Fixed a bug in the \Condor{starter} that could cause a crash if
the job is evicted while a \Condor{ssh\_to\_job} command is active.
\Ticket{4251}

\item HTCondor executables are now linked on RHEL 5 platforms using
both GNU and System V hash styles to support older Linux distributions,
such as SUSE Enterprise Linux 10.
\Ticket{4278}

\item Fixed a bug in which \Condor{dagman} did not correctly update
the node status file at the end of the run for any DAGs containing
a FINAL node.
\Ticket{4209}

\item Fixed a bug in which \Condor{dagman} sometimes printed incorrect
node status counts to the node status file.  Note that this fix also
included changing one of the strings used as a node status from
\Expr{"STATUS\_UNREADY"} to \Expr{"STATUS\_NOT\_READY"}.
\Ticket{4248}

\item Fixed a bug that caused the \Prog{nordugrid\_gahp} to crash if
the NorduGrid ARC server returned an unexpected response during job
submission.
\Ticket{4318}

\item Fixed a bug that caused the \Condor{startd} to crash when
\Attr{NUM\_SLOTS} was set to 0 and a \Condor{reconfig} command was
issued.
\Ticket{4327}

\item Under limited and unusual error conditions,
the \Condor{schedd} and the \Condor{shadow} could treat a failed 
file transfer as if it had succeeded.
As a result, the \Condor{schedd} could try to run a job whose input
files were not available, 
or a job could leave the queue as completed,
despite its output files not being transferred back from the executing
machine.
\Ticket{4328}

\item The \Condor{shadow} no longer performs a DNS query while holding
the lock for the job event logs.
\Ticket{4377}

\end{itemize}

%%%%%%%%%%%%%%%%%%%%%%%%%%%%%%%%%%%%%%%%%%%%%%%%%%%%%%%%%%%%%%%%%%%%%%
\subsection*{\label{sec:New-8-0-6}Version 8.0.6}
%%%%%%%%%%%%%%%%%%%%%%%%%%%%%%%%%%%%%%%%%%%%%%%%%%%%%%%%%%%%%%%%%%%%%%

\noindent Release Notes:

\begin{itemize}

\item HTCondor version 8.0.6 released on February 11, 2014.

\end{itemize}


\noindent New Features:

\begin{itemize}

\item A full port, which includes support for the standard universe,
is available for the Red Hat Enterprise Linux 7.0 \emph{Beta} release
on the x86\_64 architecture.
\Ticket{3629}

\item HTCondor now forces proxies that it delegates 
to be a minimum of 1024 bits.
\Ticket{4168}

\end{itemize}

\noindent Bugs Fixed:

\begin{itemize}

\item Fixed a rare bug in the \Condor{negotiator} where 
completely disabled preemption achieved with
\Expr{PREEMPTION\_REQUIREMENTS = False}
might still cause a newly started job to be preempted.  
This bug was much more likely when
configuration variable \Macro{NEGOTIATOR\_CYCLE\_DELAY} was set lower 
than the default value.
\Ticket{4185}

\item Fixed a bug in the \Condor{schedd} which would cause it
to crash when running remotely submitted parallel universe jobs.
\Ticket{4163}

\item Fixed a bug with concurrency limits and partitionable slots,
in which the partitionable slot would consume a concurrency token,
even though the slot never runs jobs.  This would cause fewer jobs
than desired to run.
\Ticket{4145}

\item Fixed a bug in which the Job Exit work fetch hook was prematurely killed.
\Ticket{3669}

\item When the Windows installer was told to use VMware,
it configured a requirement for a 
\File{condor\_vmware\_local\_settings} file that it did not provide.
This caused the \Condor{vm-gahp} to fail to start,
unless the user created this file.
This configuration has been removed, so that the file is no longer required. 
\Ticket{4109}

\item Fixed a crash of the \Condor{shadow}, triggered when a disconnect
from the \Condor{starter} occurs just as the job terminates.
\Ticket{4127}

\item Modified the output of \Condor{q},
such that when invoked with options \Opt{-userlog} and \Opt{-af},
a blank line and a totals summary line are not displayed.
\Ticket{4045}

\item For hierarchical group quotas, 
fixed an incorrect count of the number of jobs when applying a quota.
\Ticket{4117}

\item Fixed the HTCondor module \Procedure{send\_command} python binding;
an incorrect argument was only the first character of the daemon's name,
instead of the full name. 
This affected the ability to turn off specific daemons,
as sent to the \Condor{master}.
\Ticket{4160}

\item Fixed the transfer of files larger than 4Gbytes,
such that they no longer stop before the transfer is completed.
This bug presented itself on all Windows systems and on all
platforms running on a 32-bit architecture.
\Ticket{4150}

\item Fixed a bug that caused \Condor{submit\_dag} to crash on very
large DAG input files, such as those larger than 2 Gbytes.
The new configuration variable \MacroNI{DAGMAN\_USE\_OLD\_DAG\_READER},
as detailed in section~\ref{param:DAGmanUseOldDagReader},
allows disabling the new file reader code put in place to fix the bug.
\Ticket{4171}

\item Fixed a bug that caused attempts to set the CPU affinity 
on Windows platforms to be quietly ignored.
\Ticket{4131}

\end{itemize}

%%%%%%%%%%%%%%%%%%%%%%%%%%%%%%%%%%%%%%%%%%%%%%%%%%%%%%%%%%%%%%%%%%%%%%
\subsection*{\label{sec:New-8-0-5}Version 8.0.5}
%%%%%%%%%%%%%%%%%%%%%%%%%%%%%%%%%%%%%%%%%%%%%%%%%%%%%%%%%%%%%%%%%%%%%%

\noindent Release Notes:

\begin{itemize}

\item HTCondor version 8.0.5 released on December 12, 2013.

\item Helper script \Condor{ckpt\_probe} has been missing
from HTCondor releases since version 7.8.1, and it is once again 
in this release.
As a result, the machine ClassAd attribute \Attr{CheckpointPlatform}
will change for standard universe jobs upon upgrade to HTCondor version 8.0.5.
This will prevent standard universe jobs started before the upgrade
from continuing, because the attribute change eliminates a match 
with upgraded machines. 
To work around this issue,
change the \Attr{LastCheckpointPlatform} attribute to be current,
such that all jobs that have produced a checkpoint will qualify to 
continue on the upgraded machines.
Make the change by using \Condor{qedit}. 
For example, use
\footnotesize
\begin{verbatim}
condor_qedit -constraint 'LastCheckpointPlatform == "LINUX INTEL 2.6.x normal N/A"'
    LastCheckpointPlatform "LINUX INTEL 2.6.x normal N/A ssse3 sse4_1 sse4_2"
\end{verbatim}
\normalsize
where \Attr{CheckpointPlatform} after upgrade shows as
\Expr{LINUX INTEL 2.6.x normal N/A ssse3 sse4\_1 sse4\_2}.
\Ticket{4025}

\end{itemize}


\noindent New Features:

\begin{itemize}

\item None.

\end{itemize}

\noindent Bugs Fixed:

\begin{itemize}

\item Fixed a bug that resulted in incorrect values for ClassAd attributes
\Attr{RemoteGroupQuota} and \Attr{SubmitterGroupQuota}.
These attributes are commonly used in an expression that defines
\Macro{PREEMPTION\_REQUIREMENTS}.
\Ticket{4093}

\item Fixed a permissions bug that prevented Java universe jobs
from running.
\Ticket{4087}

\item Fixed a bug in which job output files may not have been properly truncated
at start up on Windows platforms. 
\Ticket{4097}

\item Fixed bug with \Condor{submit} 
when invoked with the \Opt{-interactive} option and cgroups where enabled.
The shell was terminated immediately after it started.
\Ticket{4028}

\item Prevent illegal values from being written into the job queue
or accounting recovery logs, in order to minimize the chance of errors when
the \Condor{schedd} or \Condor{negotiator} are restarted. Also
have \Condor{qedit} validate attribute names and values.
\Ticket{3616}

\item The \Condor{starter} now writes to the log file
\File{StarterLog.slotX} when running work fetch jobs, mirroring the
behavior when running jobs that come from a \Condor{schedd}.
Previously, log file \File{StarterLog} was used for all work fetch
jobs.
\Ticket{3091}

\item Fixed a Python binding crash due to a segmentation fault, when evaluating an expression tree with an undefined reference.
\Ticket{3910}

\item Fixed the cURL file transfer plugin such that it now works
on Windows platforms.
\Ticket{3979}
\Ticket{4012}

\item Fixed a bug that caused cream grid universe jobs to fail, 
if the submit description file contained 
submit commands of \SubmitCmd{environment} or \SubmitCmd{cream\_attributes}.
\Ticket{4037}

\item Fixed a bug that could cause the \Condor{schedd} to crash when
starting a local universe job.
\Ticket{4088}

\item Fixed a bug that caused \File{stdout} and \File{stderr} of
nordugrid grid universe jobs to be lost when the remote 
NorduGrid ARC server was using HTCondor as its local batch scheduler.
\Ticket{4017}

\item \Condor{status} with the \Opt{-t} option now consistently 
specifies the \Opt{-total} option. 
The \Opt{-target} option will now be distinguished, as it requires
at least \Opt{-targ} in its specification.
\Ticket{4096}

\item Restored the omitted HTCondor Perl module.
\Ticket{4098}

\item For RPM installations, the post-install script now
checks to make sure that the
SELinux \Code{unconfined\_execmem\_exec\_t type} exists before trying to
add it to the \File{/usr/sbin/condor\_startd} file context.
\Ticket{4034}

\item Fixed a bug in which both daemons and tools may have crashed due to
the inability to resolve a host name defined by 
configuration variable \Macro{COLLECTOR\_HOST} to an IP address.
\Ticket{3946}

\item Changed the algorithm for calculating the value of attribute
\Attr{RecentDaemonCoreDutyCycle} such
that the published value will never be negative.
\Ticket{4052}

\item For VMware vm universe jobs,
the \Condor{vm-gahp} removed from the vm description file any cdrom, 
floppy drive, serial or parallel devices.
Now, only devices that do not refer to image files are removed,
as the devices may be useful to the virtual machine.
\Ticket{4002}

\end{itemize}


%%%%%%%%%%%%%%%%%%%%%%%%%%%%%%%%%%%%%%%%%%%%%%%%%%%%%%%%%%%%%%%%%%%%%%
\subsection*{\label{sec:New-8-0-4}Version 8.0.4}
%%%%%%%%%%%%%%%%%%%%%%%%%%%%%%%%%%%%%%%%%%%%%%%%%%%%%%%%%%%%%%%%%%%%%%

\noindent Release Notes:

\begin{itemize}

\item HTCondor version 8.0.4 released on October 24, 2013.

\item A clipped version of HTCondor is now provided for Ubuntu 12.04 
on the x86\_64 architecture.
\Ticket{3972}

\end{itemize}


\noindent New Features:

\begin{itemize}

\item None.

\end{itemize}

\noindent Configuration Variable and ClassAd Attribute Additions and Changes:

\begin{itemize}

\item The new configuration variable \Macro{DYNAMIC\_RUN\_ACCOUNT\_LOCAL\_GROUP}
permits an administrator of a Windows machine to specify a local group other 
than the default \Expr{Users} for the \Expr{condor-reuse-slot<X>} run account.
\Ticket{3998}

\end{itemize}

\noindent Bugs Fixed:

\begin{itemize}

\item Added a retry to work around problems with slow
DHCP servers.  The result of the problem was that daemons
would not be able to determine their own host names,
and, among other problems, machine ClassAds would appear
in the output of \Condor{status} with blank host names.
\Ticket{3956}

\item A spurious warning in the log of the \Condor{startd} 
about undefined \MacroNI{PREEMPT\_VANILLA} expressions
is no longer generated.
\Ticket{3984}

\item EC2 grid universe jobs may provide data in files.
File contents of null-terminated strings worked correctly,
but binary data did not.
This bug fix makes sure that both kinds of data are read correctly.
\Ticket{3924}

\item EC2 grid universe jobs on OpenNebula work better for user data
specified with \SubmitCmd{ec2\_user\_data},
where the size of the data is larger than 2Kbytes.
\Ticket{3923}

\item Fixed a bug preventing HTTP file transfers from following redirects.
HTTP file transfers also now fail if the file does not exist,
but the server does exist.
\Ticket{3904}

\item Fixed a rare bug in which the \Condor{schedd} would exit
with an ERROR message pertaining to \Procedure{select},
when running with a very heavy load and many \Condor{shadow} daemons
time out.
\Ticket{3947}

\item Fixed a bug in \Condor{ssh\_to\_job}
that caused it to kill the job when the ssh session exited,
if cgroups were enabled.
\Ticket{3921}

\item If a standard universe job exited without producing any checkpoints 
and no checkpoint server was used, 
two spurious error messages would be logged to the \File{SchedLog}, 
as it tried to remove the old checkpoint images from the 
non-existent checkpoint server. 
These error messages are no longer logged.
\Ticket{3919}

\item The configuration file for Windows erroneously had two entries
for configuration variable \Macro{JAVA\_CLASSPATH\_SEPARATOR},
with the second entry specifying the
separator used on Unix platforms, which overrode the first entry. 
The Unix separator no longer affects this configuration file
for Windows platforms.
\Ticket{3957}

\item The NorduGrid GAHP no longer consumes all of the CPU when run
with threaded Globus libraries.
\Ticket{3958}

\item Fixed a bug that could cause ClassAd function \Procedure{quantize} to not
evaluate properly when its second argument was a list and ClassAd caching
was enabled.
\Ticket{3967}

\item Fixed a bug that caused the configuration variable setting 
\Expr{STARTD\_CRON\_AUTOPUBLISH = If\_Changed} to not work correctly.  
Updates were incorrectly sent to the \Condor{collector}
in many cases when no attribute value had changed.
\Ticket{3983}

\item Fixed a bug that could cause \Condor{q} \Opt{-analyze} or 
\Opt{-better-analyze}
to sometimes crash on an Illegal Instruction, 
when the \Attr{Requirements} expression contained a function.
\Ticket{3985}

\item The configuration variables \Macro{UPDATE\_COLLECTOR\_WITH\_TCP}
and \Macro{TCP\_UPDATE\_COLLECTORS} are now respected when forwarding
ClassAds to an HTCondorView server and setting \Macro{CONDOR\_VIEW\_HOST}.
\Ticket{3986}

\item \Condor{who} now prints an error message when passed an invalid argument.
\Ticket{3987}

\item Changed the script that finds the VMware tools to use the standard
environment variables to find \File{Program Files}, instead of using a
hard coded path.
This fixes installation for both 64-bit and non-English language Windows 
platforms.
\Ticket{272}

\end{itemize}

\noindent Known Bugs:

\begin{itemize}

\item None.

\end{itemize}

\noindent Additions and Changes to the Manual:

\begin{itemize}

\item None.

\end{itemize}


%%%%%%%%%%%%%%%%%%%%%%%%%%%%%%%%%%%%%%%%%%%%%%%%%%%%%%%%%%%%%%%%%%%%%%
\subsection*{\label{sec:New-8-0-3}Version 8.0.3}
%%%%%%%%%%%%%%%%%%%%%%%%%%%%%%%%%%%%%%%%%%%%%%%%%%%%%%%%%%%%%%%%%%%%%%

\noindent Release Notes:

\begin{itemize}

\item HTCondor version 8.0.3 released on September 23, 2013.

\end{itemize}


\noindent New Features:

\begin{itemize}

\item None.

\end{itemize}

\noindent Configuration Variable and ClassAd Attribute Additions and Changes:

\begin{itemize}

\item None.

\end{itemize}

\noindent Bugs Fixed:

\begin{itemize}

\item When expressions for \Macro{SUSPEND},
\Macro{CONTINUE}, \Macro{PREEMPT}, and \Macro{KILL}
were evaluated by the \Condor{startd},
a resulting value of
\Expr{UNDEFINED} or \Expr{ERROR} caused an exception in the \Condor{startd}.
These values are now treated as \Expr{False},
eliminating the exception.
This fix addresses CVE-2013-4255.
\Ticket{3869}

\item Fixed a bug that prevented the use of simple host names to identify
machines when using the command-line tools. Before the fix,
\Expr{condor\_status bar.foo.org} would show machine ClassAds, but
\Expr{condor\_status bar} would not.
Both now show machine ClassAds.
\Ticket{3694}

\item Fixed a bug with \Condor{ssh\_to\_job} that occurs when
\Macro{USE\_PID\_NAMESPACES} is enabled.
The bug presented itself as \Condor{ssh\_to\_job}
running in a private pid namespace, 
and the user running \Condor{ssh\_to\_job} not being able to see 
their job with \Prog{ps} or \Prog{gdb}.
As a fix, \Condor{ssh\_to\_job} now runs in the global namespace, 
so that it can see the processes in the user's job.
\Ticket{3872}

\item Fixed a performance problem with the \Condor{qedit} command 
that would cause the \Condor{schedd} to run very slowly when 
\Condor{qedit} is run on a large number of jobs.  
\Condor{qedit} no longer writes an event to the job event log. 
\Ticket{3827}

\item Fixed a problem where the \Code{classad} python module would return
incorrect results when ClassAd caching is enabled.
\Ticket{3879}

\item DAGMan's updating of its job ClassAd with DAG status attributes no
longer causes extra events to be written to the job event log and event log.
\Ticket{3863}

\item Fixed a bug that caused the command
\begin{verbatim}
  condor_submit -
\end{verbatim}
to be interpreted as an interactive submit,
rather than a request to read input from \File{stdin}.
\Condor{qsub} was also modified to be immune to this bug,
such that it will still work with other versions of HTCondor containing
the bug.
\Ticket{3902}

\item Value 032 of the job ClassAd attribute \Attr{HoldReasonCode}
was being used for two different reasons.
Now, value 032 identifies that the maximum total input file 
transfer size was exceeded. 
Value 034 identifies that memory usage exceeds a memory limit.
\Ticket{3858}

\item The values of the job ClassAd attributes \Attr{RemoteSysCpu} 
and \Attr{RemoteUserCpu} are sometimes impossibly large.
This bug rarely occurs and is not well understood.
Code changes attempt to fix this problem.
\Ticket{3814}

\item Fixed a bug that caused DAG recovery mode to fail on
Pegasus-generated sub-DAGs.  (Recovery mode is invoked, for example,
when a \Condor{dagman} is held and released, or when a schedd is
restarted, as happens on a version upgrade.)
\Ticket{3882}

\end{itemize}

\noindent Known Bugs:

\begin{itemize}

\item None.

\end{itemize}

\noindent Additions and Changes to the Manual:

\begin{itemize}

\item None.

\end{itemize}


%%%%%%%%%%%%%%%%%%%%%%%%%%%%%%%%%%%%%%%%%%%%%%%%%%%%%%%%%%%%%%%%%%%%%%
\subsection*{\label{sec:New-8-0-2}Version 8.0.2}
%%%%%%%%%%%%%%%%%%%%%%%%%%%%%%%%%%%%%%%%%%%%%%%%%%%%%%%%%%%%%%%%%%%%%%

\noindent Release Notes:

\begin{itemize}

%\item HTCondor version 8.0.2 not yet released.
\item HTCondor version 8.0.2 released on August 22, 2013.

\item Debian 5 is past its end of life. 
Starting with this release, we no longer provide native packages or
tarballs for Debian 5.
\Ticket{3852}

\end{itemize}


\noindent New Features:

\begin{itemize}

\item None.

\end{itemize}

\noindent Configuration Variable and ClassAd Attribute Additions and Changes:

\begin{itemize}

\item The default value of \Macro{ENABLE\_DEPRECATION\_WARNINGS} 
has been changed to \Expr{False}.
\Ticket{3848}

\end{itemize}

\noindent Bugs Fixed:

\begin{itemize}

\item Implemented a workaround to avoid triggering a Linux kernel defect 
when using cgroups and suspending jobs.
\Ticket{3847}

\item Fixed a python bindings problem of missing converters by providing
\Code{pyclassad} as a shared library.
\Ticket{3780}

\item Fixed a file permission bug introduced in HTCondor version 7.9.2 that
prevented vm universe jobs from working when using the Xen or KVM
hypervisor.
\Ticket{3781}

\item Fixed a bug that could cause the \Condor{collector} to 
become unresponsive if the remote HTCondorView server, 
specified with configuration variable \Macro{CONDOR\_VIEW\_HOST},
becomes unavailable.
\Ticket{3758}

\item Code to publish Linux distribution attributes in the machine ClassAd
is now more robust in the event that the \File{/etc/issue} file was edited.
\Ticket{3854}

\item Fixed a bug that could cause jobs to be incorrectly placed on hold upon
	completion with a hold reason claiming an out-of-memory event.
\Ticket{3824}

\item Fixed a bug that prevented work fetch scripts from running
on systems where cgroup based tracking and management was enabled.
\Ticket{3819}

\item Fixed a bug that could cause the \Condor{negotiator} to give out the same
slot twice, and result in a scary entry in the \File{NegotiatorLog} file 
with the wording:
\begin{verbatim}
  INSANE: bestCached != bestSoFar
\end{verbatim}
\Ticket{2245}

\item Fixed a bug introduced in HTCondor version 7.9.3,
in which concurrency limits were not respected across negotiation cycles when
\Macro{NEGOTIATOR\_CONSIDER\_PREEMPTION} was \Expr{False}.
\Ticket{3815}

\item Fixed a bug from HTCondor version 7.9.6. 
The bug exhibited itself when using CCB to connect to the \Condor{startd};
the \Condor{negotiator} and \Condor{schedd} would sometimes crash and then be restarted
with the following error message in the log:

\begin{verbatim}
ERROR "Selector::add_fd(): fd -1 outside valid range 0-1023"
\end{verbatim}

A workaround for the problem is relevant to HTCondor versions 7.9.6 through
8.0.1. Configure
\begin{verbatim}
  SERVICE_COMMAND_SOCKET_MAX_SOCKET_INDEX = -1
\end{verbatim}
\Ticket{3801}

\item Fixed a bug in the \Condor{qsub} script that caused it to exit
with a syntax error when a job with a memory requirement was
submitted.
\Ticket{3808}

\item Fix a bug causing security groups for EC2 jobs to be ignored.  
Also, the code respects the use of commas, as documented, 
to separate the items in the list of security groups specified by
the submit description file command \SubmitCmd{ec2\_security\_groups}. 
\Ticket{3787}

\item When invoking \Prog{glexec}, environment variable
\Env{GLEXEC\_TARGET\_PROXY} is now set to \File{/dev/null}.  
In some situations, it was previously set
to a nonexistent path, which caused errors in some configurations.
\Ticket{3794}

\item HTCondor daemons are now less vulnerable to long connection delays
when attempting to connect to hosts that are off-line.  A specific case
where this helps is when \Condor{schedd} is using a high availability
configuration, and the primary machine running the \Condor{collector} 
is off-line.
\Ticket{3828}

\item Fixed a bug that could cause \Condor{dagman} to hang 
due to a rarely seen event ordering.
This bug could have been triggered when using the
configuration variable \Macro{DAGMAN\_MAX\_JOBS\_IDLE},
or its equivalent command line option \Opt{-maxidle}.
\Ticket{3834}

\item Fixed a bug that caused job submission from Windows platforms
using \Condor{submit} with the \Opt{-spool} option to always fail.
\Ticket{3791}

\end{itemize}

\noindent Known Bugs:

\begin{itemize}

\item DAGMan recovery mode does not work for Pegasus-generated SUBDAGs.
For SUBDAGs, doing \Condor{hold} or \Condor{release} on
the \Condor{dagman} job, or stopping and re-starting the 
\Condor{schedd} with the DAGMan
job in the queue will result in failure of the DAG.  This can be
avoided by doing a \Condor{rm} of the DAGMan job, which produces a Rescue
DAG, and re-submitting the DAG; the Rescue DAG is automatically run.
This bug was introduced in HTCondor version 8.0.1.
\Ticket{3882}

\end{itemize}

\noindent Additions and Changes to the Manual:

\begin{itemize}

\item None.

\end{itemize}


%%%%%%%%%%%%%%%%%%%%%%%%%%%%%%%%%%%%%%%%%%%%%%%%%%%%%%%%%%%%%%%%%%%%%%
\subsection*{\label{sec:New-8-0-1}Version 8.0.1}
%%%%%%%%%%%%%%%%%%%%%%%%%%%%%%%%%%%%%%%%%%%%%%%%%%%%%%%%%%%%%%%%%%%%%%

\noindent Release Notes:

\begin{itemize}

\item HTCondor version 8.0.1 released on July 17, 2013.

\end{itemize}


\noindent New Features:

\begin{itemize}

\item HTCondor now provides the Debian Linux 7.0 (wheezy) platform,
including support for the standard universe.
\Ticket{3665}

\end{itemize}

\noindent Configuration Variable and ClassAd Attribute Additions and Changes:

\begin{itemize}

\item None.

\end{itemize}

\noindent Bugs Fixed:

\begin{itemize}

\item Fixed a bug that prevented per-slot settings of the 
\MacroNI{STARTD\_ATTRS} configuration variable from being set
correctly for partitionable slots named with a \Expr{SLOTX\_} prefix.
\Ticket{3726}

\item Fixed a bug that caused \Condor{status} \Opt{-submitters} to report twice
as many jobs running as were actually running. 
This bug appeared in HTCondor versions 7.9.6 and 8.0.0.
\Ticket{3713}

\item Fixed a bug with hierarchical group quotas in the \Condor{negotiator}
in which group hierarchies with parent groups that set 
configuration variable \Macro{GROUP\_ACCEPT\_SURPLUS} to
\Expr{False} would be assigned allocations above their quota.
\Ticket{3695}

\item Fixed a bug in which scheduler universe jobs that
have an \SubmitCmd{on\_exit\_hold}
expression that evaluates to \Expr{True} could have duplicate hold messages
in the user log.
\Ticket{3651}

\item Fixed a bug in which \Condor{dagman} would submit multiple copies of the
same job, fail, write a Rescue DAG, and leave the jobs in the queue. 
This was due to a warning from \Condor{submit} that the submit description file
was not using lines containing the string \Expr{"cluster"}. 
As a fix, \Condor{dagman} will search for the
string \Expr{" submitted to cluster "}.
This will generate fewer false alarms. 
If the submission succeeds and \Condor{dagman} gets confused, 
the jobs will be removed when \Condor{dagman} writes a Rescue DAG.
\Ticket{3658}

\item Added \Expr{libdate-manip-perl} as a dependency to the Debian packages.
It is required in order to run the \Condor{gather\_info} script.
\Ticket{3692}

\item Configuration variable \Macro{CCB\_ADDRESS} did not correctly 
support a list of CCB servers.  Only the first one in the list was used.
\Ticket{3699}

\item Fixed a bug that caused some communication layer log messages 
to end with binary characters.
\Ticket{3706}

\item Fixed a bug that can cause the \Condor{procd} to erroneously exit
on Mac OS X when many processes are created in a short period of time.
\Ticket{3725}

\item Removed a bug that caused \Condor{dagman} to have problems restarting
after an upgrade from HTCondor version 7.8.
\Ticket{3707}

\item Fixed a bug that caused the command 
\begin{verbatim}
  condor_q -dag -run
\end{verbatim}
to print garbage.
\Ticket{3578}

\item Fixed a bug that prevented jobs with an invalid \SubmitCmd{ec2\_keypair}
from being removed.
\Ticket{3485}

\item Fixed a memory leak and potential crash in the \Condor{gridmanager}
when requests to an EC2 service fail.
\Ticket{3701}

\item Fixed a bug in the \Condor{gridmanager} that can cause EC2 jobs to be
submitted a second time during recovery.
\Ticket{3705}

\item Fixed a memory leak in the \Condor{gridmanager} that was triggered when
submitting EC2 grid universe jobs.
\Ticket{3720}

\end{itemize}

\noindent Known Bugs:

\begin{itemize}

\item None.

\end{itemize}

\noindent Additions and Changes to the Manual:

\begin{itemize}

\item None.

\end{itemize}


%%%%%%%%%%%%%%%%%%%%%%%%%%%%%%%%%%%%%%%%%%%%%%%%%%%%%%%%%%%%%%%%%%%%%%
\subsection*{\label{sec:New-8-0-0}Version 8.0.0}
%%%%%%%%%%%%%%%%%%%%%%%%%%%%%%%%%%%%%%%%%%%%%%%%%%%%%%%%%%%%%%%%%%%%%%

\noindent Release Notes:

\begin{itemize}

%\item HTCondor version 8.0.0 not yet released.
\item HTCondor version 8.0.0 released on June 6, 2013.

\end{itemize}


\noindent New Features:

\begin{itemize}

% would have been in 7.9.7, but there was no 7.9.7 release
\item The \Condor{chirp} \Opt{write} command now accepts an 
optional \Arg{numbytes} parameter following the local file name.
This allows the write to be limited to the specified number of bytes.
\Ticket{3548}

% would have been in 7.9.7, but there was no 7.9.7 release
\item The HTCondor Python bindings now build on Mac OS X.
\Ticket{3584}

\item Updated the sample \File{condor.plist} file to work better with
current versions of Mac OS X.
\Ticket{3624}

\end{itemize}

\noindent Configuration Variable and ClassAd Attribute Additions and Changes:

\begin{itemize}

\item The new configuration variable
\Macro{DEDICATED\_SCHEDULER\_WAIT\_FOR\_SPOOLER}
permits the specification of a very strict execution order for 
parallel universe jobs handed to a remote scheduler.
\Ticket{2946}

\end{itemize}

\noindent Bugs Fixed:

\begin{itemize}

\item Fixed a bug that happened with partitionable slots, jobs that
requested more than one cpu, and a negotiator with
\Macro{NEGOTIATOR\_CONSIDER\_PREEMPTION} was false.  This would
cause the negotiator to incorrectly assume that each slot had
a slot weight of one.
\Ticket{3737}

\item The redundant configuration variable \Macro{CheckpointPlatform} has
been removed and the configuration variable \Macro{CHECKPOINT\_PLATFORM}
documented.
\Ticket{3544}

\item A standard universe job will no longer crash, and it will no longer 
cause the \Condor{shadow} to crash
if the job calls \Procedure{mmap} with an unsupported combination of flags.
\Ticket{3565}

\item Support for \Prog{VMware Workstation} and \Prog{VMware Player} 
under the \SubmitCmd{vm} universe now works properly on Windows platforms.
\Ticket{3627}

% would have been in 7.9.7, but there was no 7.9.7 release
\item For grid universe jobs intended for an EC2 grid resource,
errors which have no response body now report the HTTP code.
\Ticket{3541}

% would have been in 7.9.7, but there was no 7.9.7 release
\item \Condor{chirp} \Opt{put} would experience an assertion failure when
used on an empty file.  This bug has been fixed, and \Opt{put} can now be
used on an empty file.
\Ticket{3542}

% would have been in 7.9.7, but there was no 7.9.7 release
\item The 32-bit \Condor{starter} could fail to execute jobs when the initial
working directory of the job was on a subsystem containing 64-bit metadata,
such as inode numbers.
\Ticket{3605}

% would have been in 7.9.7, but there was no 7.9.7 release
\item \Condor{dagman} failed to react correctly if a nested DAG file
did not exist. It now reacts correctly and prints a more
helpful error message.
\Ticket{3623}

% would have been in 7.8.9, but there was no 7.8.9 release
\item Fixed a bug that caused the \Condor{master} daemon on Windows platforms
to think there were new binaries
when changing to and from daylight savings time.
The \Condor{master} daemon would then kill and restart itself,
as well as all of the daemons,
if configuration variable \Macro{MASTER\_NEW\_BINARY\_RESTART} was set
to its default value of \Expr{GRACEFUL}.
\Ticket{3572}

\item Fixed a bug that caused redundant lines to show up in the user log
at the end of the partitionable resource usage summary.
\Ticket{3621}

\item Fixed several bugs that can cause the \Condor{procd} to fail on Mac OS X
and not be restartable.
\Ticket{3617}
\Ticket{3618}
\Ticket{3620}

\item The \Condor{procd} now ignores process id 0 on Mac OS X.
\Ticket{3516}

\item Fixed memory leaks in the \Condor{shadow} and the \Condor{startd};
fixed a file descriptor leak in the standard universe \Condor{starter}.
\Ticket{3590}

\item Fixed a bug in which \Condor{dagman} would miscount the number
of held jobs when
multiple copies of hold events were written to the user log.
\Ticket{3650}

\end{itemize}

\noindent Known Bugs:

\begin{itemize}

\item The following obsolete binaries have not yet been removed from
the HTCondor tarballs:  
  \begin{itemize}
  \item \emph{classad\_functional\_tester}
  \item \emph{classad\_version}
  \item \Condor{test\_match}
  \item \Condor{userlog\_job\_counter}
  \end{itemize}
\Ticket{3670}

\item \Condor{status} \Opt{-submitters} reports twice
as many jobs running as were actually running.
\Ticket{3713}

\end{itemize}

\noindent Additions and Changes to the Manual:

\begin{itemize}

\item Fixed the \Condor{configure} man page and added a corresponding
\Condor{install} man page.
\Ticket{3619}

\item Added stub man pages for the Bosco commands.
\Ticket{3634}

\end{itemize}



% as of the 8.2.0 release, the 7-9 and 7-8 version histories no longer included.
%%%%      PLEASE RUN A SPELL CHECKER BEFORE COMMITTING YOUR CHANGES!
%%%      PLEASE RUN A SPELL CHECKER BEFORE COMMITTING YOUR CHANGES!
%%%      PLEASE RUN A SPELL CHECKER BEFORE COMMITTING YOUR CHANGES!
%%%      PLEASE RUN A SPELL CHECKER BEFORE COMMITTING YOUR CHANGES!
%%%      PLEASE RUN A SPELL CHECKER BEFORE COMMITTING YOUR CHANGES!

%%%%%%%%%%%%%%%%%%%%%%%%%%%%%%%%%%%%%%%%%%%%%%%%%%%%%%%%%%%%%%%%%%%%%%
\section{\label{sec:History-7-9}Development Release Series 7.9}
%%%%%%%%%%%%%%%%%%%%%%%%%%%%%%%%%%%%%%%%%%%%%%%%%%%%%%%%%%%%%%%%%%%%%%

This is the development release series of HTCondor.
The details of each version are described below.

%% NO 7.9.7 RELEASE IS PLANNED.  PLEASE PUT YOUR ITEMS INTO THE 
%%  8.0.0 VERSION HISTORY   !!!

%%%%%%%%%%%%%%%%%%%%%%%%%%%%%%%%%%%%%%%%%%%%%%%%%%%%%%%%%%%%%%%%%%%%%%
\subsection*{\label{sec:New-7-9-6}Version 7.9.6}
%%%%%%%%%%%%%%%%%%%%%%%%%%%%%%%%%%%%%%%%%%%%%%%%%%%%%%%%%%%%%%%%%%%%%%

\noindent Release Notes:

\begin{itemize}

\item HTCondor version 7.9.6 released on May 8, 2013.

\end{itemize}


\noindent New Features:

\begin{itemize}

\item The new \Condor{ping} command line tool attempts one or more
targeted security negotiations to see if it succeeds or fails,
potentially helping to debug security configuration.
\Ticket{3371}

\item The \Condor{schedd} will now also advertise demand by jobs for slots,
weighted by the count of requested CPUs, if the configuration variable
\Macro{NEGOTIATOR\_USE\_WEIGHTED\_DEMAND} is set to \Expr{True}.
The default value is \Expr{False}.
\Ticket{3574}

\item Negotiation under groups now prefers the specification of
groups with the new submit commands \SubmitCmd{accounting\_group} and
\SubmitCmd{accounting\_group\_user}.
See section~\ref{sec:group-accounting} for details.
\Ticket{2728}

\item The \SubmitCmd{vm} universe now supports \Prog{VMware Workstation}
and \Prog{VMware Player}.
\Ticket{740}

\item On Linux platforms where cgroups are supported and enabled, the 
\Condor{starter} will now detect and trap if a vanilla universe job 
would otherwise be killed by the system Out Of Memory (OOM) killer.  
This situation is especially likely when a job sets \SubmitCmd{RequestMemory}
lower than needed.  The job will now be put on hold.
\Ticket{2992}

\item The new \Opt{-force-graceful} command-line option to \Condor{off}
allows administrators to issue a graceful shutdown command, even after
issuing a \Opt{-peaceful} command. Previously, a peaceful \Condor{off}
command would preclude a \Opt{-graceful} off command.
\Ticket{2949}

\item The \Condor{gather\_info} tool now includes the output of the Unix
\Prog{uptime} and \Prog{free} programs, 
as well as logs of the \Condor{master}, \Condor{startd}, and \Condor{starter}
of the machine where the job most recently ran, 
if \Condor{gather\_info} has the necessary permissions to fetch those logs.
\Ticket{3246}

\item The rotation of a daemon log file can now be specified 
in terms of time (seconds) or in terms of maximum size (bytes).
Only size was allowed previously.
\Ticket{3560}

\item The new \Condor{qsub} command line tool emulates submission to PBS, SGE, 
and Torque-like systems. It handles both scripts and command line options.
\Ticket{2699}

\item When submitting a grid universe job with a grid type of batch,
the value of \SubmitCmd{request\_memory} is now propagated to the batch system
submission request.
\Ticket{3398}

\item The set of Python bindings introduced in HTCondor version 7.9.4 is now
distributed as part of HTCondor, not as a contrib module.
\Ticket{3586}

\item Several improvements have been made to the
\Condor{gather\_info} tool.  It now prints the name of the
tarball it emits, and now also checks the history file for the
job in question, and if found, uses and displays the information 
there.
\Ticket{3239}
\Ticket{3240}

\end{itemize}

\noindent Configuration Variable and ClassAd Attribute Additions and Changes:

\begin{itemize}

\item The new configuration variable 
\Macro{GSI\_DELEGATION\_CLOCK\_SKEW\_ALLOWABLE}, expressed in seconds,
allows HTCondor to adjust the amount of
allowable clock skew between two parties.
This is relevant when delegating X.509 proxies.
\Ticket{3557}

\item The new configuration variable \Macro{CheckpointPlatform}
is a string that may be set by
an administrator to override the auto-detected
platform used to determine if a standard universe job that produced 
a checkpoint on one machine can be started on another.
\Ticket{3544}

%% Yes, SSSE 3 has three Ss.
\item The new Machine ClassAd attributes \AdAttr{Has\_sse4\_1},
\AdAttr{Has\_sse4\_2}, and \AdAttr{Has\_ssse3} are set to \Expr{True}
if the corresponding instruction set additions exist on that machine.
These attributes will be undefined otherwise.
\Ticket{3544}

\item The name of the configuration variable \MacroNI{MEMORY\_LIMIT}
introduced in HTCondor version 7.9.2 has changed.
This variable is now called \Macro{CGROUP\_MEMORY\_LIMIT\_POLICY}.
\Ticket{3564}

\item The new configuration variable \Macro{EXPIRE\_INVALIDATED\_ADS},
when set to \Expr{True}, causes invalidated ClassAds that would have
been removed from the \Condor{collector} right away to instead
be treated as expired ClassAds, such that they may become absent ClassAds.
See section~\ref{sec:Absent-Ads} for details on absent ClassAds.
\Ticket{3085}

\item The new configuration variable
\Macro{GLEXEC\_HOLD\_ON\_INITIAL\_FAILURE} controls whether jobs are put
on hold when a failure is encountered in the glexec setup phase of
managing the job.  The default value is \Expr{True}, 
which implements the previous behavior of putting a job on hold when
there is a failure.
\Ticket{3569}

\item The new configuration variable
\Macro{NEGOTIATOR\_CONSIDER\_EARLY\_PREEMPTION} controls whether jobs
can be matched to slots that still have retirement time remaining
before the existing job can be evicted.  The default is \Expr{False}.
The old behavior can be enabled by setting it to \Expr{True}.  The
new default behavior is intended to improve scheduling behavior
when \Expr{MaxJobRetirementTime} is used.
\Ticket{3539}

\item The new configuration variable \Macro{SCHEDD\_AUDIT\_LOG}
defines a file name, such that the
\Condor{schedd} can now write an audit log that records all
commands issued by users that modify the job queue.
\Ticket{3493}

\item The per-user file transfer I/O statistics now have a prefix of
\AdAttr{Owner\_<username>\_}.  In HTCondor version 7.9.5, 
they had a prefix of
\AdAttr{<username>\_}.  This can be configured via
\Macro{TRANSFER\_QUEUE\_USER\_EXPR}.
\Ticket{3496}

\item The new configuration variable
\Macro{BATCH\_GAHP\_CHECK\_STATUS\_ATTEMPTS} controls how often the
\Condor{gridmanager} should retry a failed job status check when using
the \Prog{batch\_gahp}. The default is 5.
\Ticket{3533}

\end{itemize}

\noindent Bugs Fixed:

\begin{itemize}

\item Fixed a bug in the setting of \Macro{ASSIGN\_CPU\_AFFINITY},
when running on systems with partitionable slots.
\Ticket{3597}

\item Fixed a bug in the \Condor{starter} in which file systems mounted by
the automounter would not be seen by the job.
\Ticket{3601}

\item Fixed the \Condor{updates\_stats} tool such that it works with
modern Perl interpreters.
\Ticket{3579}

\item Fixed a bug that caused the \Condor{schedd} to crash when 
\Condor{rm} was used with the \Opt{-f} option on parallel universe jobs.
\Ticket{3561}

\item Fixed a bug that could cause HTCondor-C jobs to fail when they are
part of a DAG. The jobs would be held with a hold reason of the form
\footnotesize
\begin{verbatim}
Failed to initialize user log to /dev/null or <DAG log path>.
\end{verbatim}
\normalsize
\Ticket{3474}

\item Fixed a severe leak of file descriptors in the \Condor{ft-gahp}.
\Ticket{3559}

\item Fixed a bug that occurred when using privilege separation;
the bug made it impossible for the \Condor{startd} to clean
up execute directories after a \Condor{starter} was prematurely killed.
\Ticket{3573}

\item Fixed a bug that sometimes caused fetching user priorities 
from the \Condor{negotiator} daemon to take a long time,
as the fetch potentially had to wait until the end of the negotiation cycle. 
The fetch no longer needs to wait.
\Ticket{3535}

\item Fixed a bug that would cause parallel universe jobs to fail when 
\Expr{USE\_NFS=True}. This might have caused a potential issue with 
doing a look up using \Condor{chirp}, 
although testing seems to show that this is not an issue.
\Ticket{3390}

\item Fixed a bug introduced in HTCondor version 7.9.4 that sometimes caused
attributes
\AdAttr{ExitCode} and \AdAttr{ExitBySignal} not to be set in the job
ClassAd when the job terminated.  These attributes were correctly
reported in the user log, but they were not propagated to the job ClassAd and
were therefore not available for querying with \Condor{history}.
\Ticket{3577}

\item Fixed a bug that caused the \Condor{starter} to leak file descriptors 
to file transfer plug-ins. 
\Ticket{3570}

\item Fixed a bug that caused the \Prog{batch\_gahp} to leak file descriptors 
when reading a job's X.509 proxy.
\Ticket{3558}

\item Fixed a bug with the dedicated scheduler reconnect mode.
If the \Condor{schedd} crashed with a parallel universe job running after the 
job's first job lease duration interval, parallel universe jobs would 
not successfully reconnect, and they
would restart from the beginning.
\Ticket{2291}

\end{itemize}

\noindent Known Bugs:

\begin{itemize}

\item Using \Condor{userprio} with the \Opt{-grouprollup} option
will fail to produce any output
if the \Condor{negotiator} it queries is of a version older 
than HTCondor version 7.9.6 and the \Condor{userprio} executable is 
HTCondor version 7.9.6.
\Ticket{3600}

\end{itemize}

\noindent Additions and Changes to the Manual:

\begin{itemize}

\item None.

\end{itemize}


%%%%%%%%%%%%%%%%%%%%%%%%%%%%%%%%%%%%%%%%%%%%%%%%%%%%%%%%%%%%%%%%%%%%%%
\subsection*{\label{sec:New-7-9-5}Version 7.9.5}
%%%%%%%%%%%%%%%%%%%%%%%%%%%%%%%%%%%%%%%%%%%%%%%%%%%%%%%%%%%%%%%%%%%%%%

\noindent Release Notes:

\begin{itemize}

\item HTCondor version 7.9.5 released on April 17, 2013.

\end{itemize}


\noindent New Features:

\begin{itemize}

\item The new command line tool \Condor{tail}
displays files that are in the sandbox of a running job.
See details in the manual page at
section~\ref{man-condor-tail}.
\Ticket{3522}

\item When there are multiple users
waiting to transfer files within the limits set by
configuration variables
\Macro{MAX\_CONCURRENT\_UPLOADS} and/or
\Macro{MAX\_CONCURRENT\_DOWNLOADS}, the scheduling algorithm
now gives the users
an equal share of the transfer slots.  How shares are counted can be
configured with \Macro{TRANSFER\_QUEUE\_USER\_EXPR}.
\Ticket{3487}

\item When using the \Opt{-remote} or \Opt{-spool} options to
\Condor{submit}, the job owner will now be set based
upon how the job submitter was authenticated. 
This will make it easier to submit jobs
to a remote \Condor{schedd} where the credentials may map to a different
account name.
\Ticket{3370} 

\item New functions are available in the Python Bindings contrib module.
ClassAds now more closely mimic Python dictionaries and provide
support for lists and values that are ClassAds. 
\Ticket{3494}

\item If a job is submitted specifying \SubmitCmd{keep\_claim\_idle},
the claim is kept not only when the job exits,
but also when the job is removed.
\Ticket{3491}

\item \Condor{dagman} now publishes in its own job ClassAd,
attributes with the DAG status,
such as total number of nodes, nodes queued, and nodes finished.
See section~\ref{sec:DAGStatusClassad} for more information.
\Ticket{1782}

\end{itemize}

\noindent Configuration Variable and ClassAd Attribute Additions and Changes:

\begin{itemize}

\item The new configuration variable \Macro{GSI\_DELEGATION\_KEYBITS}
allows the number of bits in a delegated proxy to be specified
by the receiving side.
\Ticket{3503}

\item When using file transfer concurrency limits, additional I/O
usage statistics are now published as attributes in the ClassAd of the
\Condor{schedd}.  This includes the sum and rate of bytes
transferred as well as time spent reading and writing to files and
to the network.  These statistics are reported for the sum of all
users and, when increased verbosity is configured, individually for
recently active users.  
These ClassAd attributes 

\begin{description}
  \item \AdAttr{FileTransferUploadBytes}
  \item \AdAttr{FileTransferUploadBytesPerSecond\_<timespan>}
  \item \AdAttr{FileTransferDownloadBytes}
  \item \AdAttr{FileTransferDownloadBytesPerSecond\_<timespan>}
  \item \AdAttr{FileTransferFileReadSeconds}
  \item \AdAttr{FileTransferFileReadLoad\_<timespan>} 
  \item \AdAttr{FileTransferFileWriteSeconds}
  \item \AdAttr{FileTransferFileWriteLoad\_<timespan>}
  \item \AdAttr{FileTransferNetReadSeconds}
  \item \AdAttr{FileTransferNetReadLoad\_<timespan>} 
  \item \AdAttr{FileTransferNetWriteSeconds}
  \item \AdAttr{FileTransferNetWriteLoad\_<timespan>} 
\end{description}
are fully described 
within the section on scheduler attributes at 
section~\pageref{sec:FT-Scheduler-ClassAd-Attributes}.
\Ticket{3496}

\item \Macro{NOT\_RESPONDING\_TIMEOUT} now internally adds some random skew
to avoid synchronization of heartbeat messages, which can lead to UDP
buffer overflow and incorrect determination that daemons are hung.
\Ticket{3510}

\end{itemize}

\noindent Bugs Fixed:

\begin{itemize}

\item The EC2 GAHP now treats OpenStack's \Code{stopped} state as if it
were \Code{shutoff}, terminating instances which enter this state and
preventing the instances from remaining in the queue forever.
\Ticket{3507}

\item Two EC2 GAHP bugs are fixed.
It now correctly parses XML namespaces as returned for
some installations of Eucalyptus.
The second bug caused HTCondor to put the job on hold, 
as it incorrectly believed that the cloud
service had purged it.
\Ticket{3492}

\item The EC2 GAHP now reports the
bidding status, defined at

\URL{http://docs.aws.amazon.com/AWSEC2/latest/UserGuide/using-spot-instances-bid-status.html},
for spot instances. 
\Ticket{3388}

\item The \Condor{negotiator} now checks to see if 
the time set by configuration variable 
\Macro{NEGOTIATOR\_MAX\_TIME\_PER\_SUBMITTER}
has been exceeded while negotiating with a single \Condor{schedd} daemon.
This configuration variable was previously only effective 
if a submitter used multiple \Condor{schedd} daemons.
\Ticket{3504}

\item \Condor{dagman} will now recover correctly in a DAG where a node has been 
skipped because of a \Macro{PRE\_SKIP} has triggered.
\Ticket{2966}

\item Fixed a bug in the \Condor{gridmanager} and \Condor{ft-gahp} that
could cause a crash when transferring files for grid universe jobs of 
grid type batch going to a remote cluster.
\Ticket{3529}

\item Fixed a bug in which \Condor{status} queries would not work,
with output of \AdStr{Access denied}, unless the \Condor{collector}
and the machine doing the query had synchronized clocks.
\Ticket{3360}

\item Fixed a Linux platform bug in which mount points were leaked
to the greater namespace when the configuration set
\MacroNI{MOUNT\_UNDER\_SCRATCH} for file systems
that have been mounted with shared propagation enabled.
\Ticket{3505}

\item Fixed a bug in the logging code that was causing grid universe batch jobs
to abort and drop a dprintf error file during file transfer once the log had
grown large enough to rotate.
\Ticket{3528}

\item On Windows platforms, running the \Condor{kbdd} no
longer creates a visible console window.
\Ticket{2805}

\end{itemize}

\noindent Known Bugs:

\begin{itemize}

\item Running \Condor{rm} with the \Opt{-f} option on a parallel universe
job can cause the \Condor{schedd} to crash.
\Ticket{3561}

\item Using privilege separation may cause execute directories to be leaked,
if the \Condor{starter} is shut down prematurely; 
for example, shut down may occur by a hard kill signal or power interruption.
\Ticket{3573}

\item  If a job has any output files and uses the file transfer mechanism,
the job ClassAd attribute \Attr{ExitCode} may be lost,
causing its value to be reported as 0.
\Ticket{3577}

\end{itemize}

\noindent Additions and Changes to the Manual:

\begin{itemize}

\item None.

\end{itemize}


%%%%%%%%%%%%%%%%%%%%%%%%%%%%%%%%%%%%%%%%%%%%%%%%%%%%%%%%%%%%%%%%%%%%%%
\subsection*{\label{sec:New-7-9-4}Version 7.9.4}
%%%%%%%%%%%%%%%%%%%%%%%%%%%%%%%%%%%%%%%%%%%%%%%%%%%%%%%%%%%%%%%%%%%%%%

\noindent Release Notes:

\begin{itemize}

\item HTCondor version 7.9.4 released on February 20, 2013.

\end{itemize}


\noindent New Features:

\begin{itemize}

\item Per job PID namespaces are available for Linux RHEL 6 platforms.
See section~\ref{sec:PIDNamespaces} for details.
\Ticket{1959}

\item The EC2 GAHP now batches requests for status updates, significantly
reducing its resource requirements.
\Ticket{3436}

\item The maximum total size of file transfers for a job may now be
specified using the new configuration variables
\Macro{MAX\_TRANSFER\_INPUT\_MB} and \Macro{MAX\_TRANSFER\_OUTPUT\_MB} 
and/or the new submit commands 
\SubmitCmd{max\_transfer\_input\_mb} and
\SubmitCmd{max\_transfer\_output\_mb}.
\Ticket{3333}

\item The \Prog{batch\_gahp} no longer relies on programs
\Prog{grid-proxy-info} and \Prog{grid-proxy-init} from the Globus
Toolkit to handle the X.509 proxies of jobs.
\Ticket{3431}

\item When the job's executable is transferred, always set the execute
bits on the copy.
\Ticket{3028}

\item By default, \Condor{dagman} now issues a fatal error
if any log file, which is either
the default log file or the log file specified for a node job,
is in \File{/tmp}, because this can cause DAGMan to fail.
This error can be downgraded to a warning by setting the
configuration variable
\MacroNI{DAGMAN\_USE\_STRICT} value to 0.
\Ticket{1419}

\item The \Condor{collector} will accept and display collector ClassAds for
multiple collectors from the same machine. For this to work, the
collectors must configured with different values for configuration
variable \MacroNI{COLLECTOR\_NAME}.
\Ticket{3467}

% Depending on who wins the argument between Igor and TJ, this might go into
% 7.8 series.  I think this is probably the right place for it, in 7.9. But
% Kent, TJ, or Igor may well have other ideas
\item \Condor{dagman} now will successfully set attributes for submitted jobs
using the \Condor{submit} syntax of placing a \Expr{+} sign just to the
left of the attribute name. 
See section~\ref{dagman:VARS} for more details.
\Ticket{3469}

\item The HTCondor contrib now includes a set of Python bindings in
two modules.
The \Code{htcondor} module interacts with the \Condor{schedd} and 
\Condor{collector} daemons. 
The \Code{classad} module provides an interface to work with ClassAds.
\Ticket{3407}

\item When using \Condor{compile}, 
Pthreads are not normally permitted to be used by standard universe jobs.
However, \Condor{compile} will now tell a user that they
should be linking to the GNU Pth library,
which is  built with the \verb@--enable-pthread@ flag.
This will permit jobs that use Pthreads to be built with \Condor{compile}.
\Ticket{3319}


\end{itemize}

\noindent Configuration Variable and ClassAd Attribute Additions and Changes:

\begin{itemize}

\item The new configuration variable \Macro{USE\_PID\_NAMESPACES}
enables per job PID namespaces for Linux RHEL 6 platforms when \Expr{True}.
\Ticket{1959}

\item The new configuration variable \Macro{FLOCK\_INCREMENT} allows
administrators to more aggressively flock to remote \Condor{collector} daemons,
as more pools will be considered.
\Ticket{3375}

\item The new configuration variable \Macro{HOST\_ALIAS} specifies the
  fully qualified host name that clients authenticating this daemon with 
  GSI should
  expect the daemon's certificate to match.  The alias is advertised
  to the \Condor{collector} as part of the address of the daemon.
  When this is not set, clients validate the daemon's certificate
  host name by matching it against DNS A records for the host they
  are connected to.  See \Macro{GSI\_SKIP\_HOST\_CHECK} for ways
  to disable this validation step.
\Ticket{1605}

\item The configuration variable \MacroNI{DAGMAN\_USE\_STRICT} now
defaults to a value of 1, rather than 0.
See the definition at section~\ref{param:DAGManUseStrict}.
\Ticket{3418}

\item The new configuration variable \Macro{GRACEFULLY\_REMOVE\_JOBS}
is a boolean value that controls whether jobs to be removed are 
gracefully removed.
The default is to do graceful removal.
\Ticket{3470}
\end{itemize}

\noindent Bugs Fixed:

\begin{itemize}

\item When HTCondor creates a key pair at an EC2 job's request, it no
longer fails to remove the private key from disk when the job leaves
the queue.
\Ticket{3477}

\item The EC2 GAHP now recognizes the OpenStack \Code{shutoff} state and
terminates instances which enter this state, 
preventing the instances from remaining in the queue forever.
\Ticket{3367}

\item \Condor{dagman} no longer does unnecessary sleeps for log file
consistency when a single default/workflow log file is used.
\Ticket{3456}

\item Fixed a bug introduced in HTCondor version 7.9.0 that caused 
the following configuration variables to not sort ClassAds properly 
when they evaluated to \Expr{True} or \Expr{False}:
\Macro{NEGOTIATOR\_PRE\_JOB\_RANK},
\Macro{NEGOTIATOR\_POST\_JOB\_RANK}, \Macro{PREEMPTION\_RANK}, and
\Macro{SCHEDD\_PREEMPTION\_RANK}.
\Ticket{3468}

\item Fixed a bug that can cause grid universe jobs of type \Expr{batch}
to fail when submitted to an HTCondor cluster with a large history file.
\Ticket{3429}

\item Corrected the submission of interactive jobs for cases 
in which the submit description file specified \SubmitCmd{Arguments}.
\Ticket{3455}

\item The semantics of signals sent to jobs were changed.
They have been changed back to the semantics defined in version 7.6.   
\Ticket{3470}

\end{itemize}

\noindent Known Bugs:

\begin{itemize}

\item None.

\end{itemize}

\noindent Additions and Changes to the Manual:

\begin{itemize}

\item None.

\end{itemize}

%%%%%%%%%%%%%%%%%%%%%%%%%%%%%%%%%%%%%%%%%%%%%%%%%%%%%%%%%%%%%%%%%%%%%%
\subsection*{\label{sec:New-7-9-3}Version 7.9.3}
%%%%%%%%%%%%%%%%%%%%%%%%%%%%%%%%%%%%%%%%%%%%%%%%%%%%%%%%%%%%%%%%%%%%%%

\noindent Release Notes:

\begin{itemize}

\item HTCondor version 7.9.3 released on January 16, 2013.

\end{itemize}


\noindent New Features:

\begin{itemize}

\item When the new configuration variable \Macro{ASSIGN\_CPU\_AFFINITY}
is set to \Expr{True}, 
the \Condor{startd} will automatically set the CPU affinity
mask jobs run with, so that a multi-threaded job will not use
more cores than the number it requests.
\Ticket{3348}

\item When configuration variable \Macro{NEGOTIATOR\_CONSIDER\_PREEMPTION}
is \Expr{False}, the \Condor{negotiator}
now fetches machine ClassAds more quickly from the \Condor{collector}
 by skipping most attributes of the busy machines.  
This can make negotiation much faster in
a very large pool of mostly claimed machines.
\Ticket{3366}

\item Round-robin scheduling is now used when there are multiple users
waiting to transfer files in the limits set by
\Macro{MAX\_CONCURRENT\_UPLOADS} and/or
\Macro{MAX\_CONCURRENT\_DOWNLOADS}.  Previously, the file transfer
queue was scheduled in first-in-first-out order, so one user with
many files to transfer could delay other users for as long as it took
to transfer those files.  Now, when choosing a new job to allow to
transfer, the first job belonging to the user who has least
recently been given an opportunity to transfer will be selected.
The old behavior, or variations on the new behavior, can be achieved
by configuring \Macro{TRANSFER\_QUEUE\_USER\_EXPR}.
\Ticket{3333}

\item \Condor{dagman} will now try twice to write a POST script terminate
event, rather than trying once and exiting. 
If it is unable to write the event, \Condor{dagman} exits, 
writing a Rescue DAG. 
\Ticket{965}

\item The \Condor{gridmanager} now cleans up temporary files and directories
that are sometimes left by the \Prog{batch\_gahp} when executing a grid
universe job of grid type \SubmitCmd{batch}.
\Ticket{3276}

\item Added counts of nodes in various states to the \Condor{dagman}
node status file.  Refer to section~\ref{sec:DAG-node-status} for
more information.
\Ticket{2075}

\item When submitting jobs to a remote batch system (for example, Bosco),
file transfer no longer requires a network connection from the remote machine
back to the local one.
\Ticket{3293}

\end{itemize}

\noindent Configuration Variable and ClassAd Attribute Additions and Changes:

\begin{itemize}

\item The new expert-only configuration variable 
\Macro{STATISTICS\_WINDOW\_QUANTUM}
allows administrators to set the time interval, 
known as a quantum, that divides a window over which statistics are
kept into smaller pieces.  The window advances one quantum at a time. 
\Ticket{3288}


\end{itemize}

\noindent Bugs Fixed:

\begin{itemize}

\item Jobs of the EC2 grid type which make invalid requests of the
service no longer go on hold when removed.
An example of this is when a job specifies a nonexistent AMI. 
\Ticket{3287}

\item Jobs of the EC2 grid type which cannot authenticate with the
service no longer go on hold when removed.
\Ticket{3387}

\item Fixed a problem with \Prog{glexec} that caused jobs not to start 
due to permission errors on the execute directory.
\Ticket{3369}

\item A change was made to more accurately implement the
minimum time defined by the configuration variable
\Macro{NEGOTIATOR\_CYCLE\_DELAY}. 
\Ticket{3332}

\item The \Prog{batch\_gahp} is no longer dependent on the Perl module 
\Code{XML::Simple} when submitting jobs to SGE.
\Ticket{3350}

\item The \Prog{batch\_gahp} now properly handles job X.509 proxies that 
are not in the old proxy format.
\Ticket{3362}

\item On 32-bit platforms,
setting configuration variable \Macro{STARTER\_RLIMIT\_AS} to a value 
larger than 4096 could cause jobs to abort on start up.
Since values larger than 2047 have no real meaning on 32-bit platforms,
the fix treats values larger than 2047 as no limit on 32-bit platforms.
\Ticket{3309}

\item Fixed a bug that can cause proxy refresh to fail for pbs, lsf,
and sge grid jobs.
\Ticket{3383}

\item When doing remote pbs, lsf, or sge grid job submissions, the
\Condor{gridmanager} now ensures that no unusual characters are used in
the name of the job sandbox directory it creates.
\Ticket{3294}

\item When a GAHP server fails to start, the \Condor{gridmanager} now
puts the affected jobs on hold.
\Ticket{3301}

\item Environment variable \Env{GLOBUS\_LOCATION} is now set for
\Prog{batch\_gahp},
allowing it to find proxy management that it needs for jobs that have an
X.509 proxy.
\Ticket{3015}

\item The installation RPM now requires Security Enhanced Linux (SELinux) 
scripts at post install time,
so that the scripts can set the appropriate security contexts.
\Ticket{3313}

\end{itemize}

\noindent Known Bugs:

\begin{itemize}

\item None.

\end{itemize}

\noindent Additions and Changes to the Manual:

\begin{itemize}

\item Initial documentation for EC2 spot instances can be found
in section~\ref{sec:spot-instances}.
\Ticket{3209}

\end{itemize}


%%%%%%%%%%%%%%%%%%%%%%%%%%%%%%%%%%%%%%%%%%%%%%%%%%%%%%%%%%%%%%%%%%%%%%
\subsection*{\label{sec:New-7-9-2}Version 7.9.2}
%%%%%%%%%%%%%%%%%%%%%%%%%%%%%%%%%%%%%%%%%%%%%%%%%%%%%%%%%%%%%%%%%%%%%%

\noindent Release Notes:

\begin{itemize}

%\item HTCondor version 7.9.2 not yet released.
\item HTCondor version 7.9.2 released on December 11, 2012.
This release contains all of the bug fixes in the version 7.8.6 
stable release,
and most of the bug fixes in the
soon to be released version 7.8.7 stable release.

\end{itemize}


\noindent New Features:

\begin{itemize}

\item The permissions for the temporary execute directory of a job
have been tightened for vanilla universe jobs, 
such that only the owner of the job is allowed to see or
modify the contents.
\Ticket{3315}

\item Added experimental support for EC2 spot instances.
\Ticket{3209}

\item (This feature was added in version 7.9.1.)  
There are two new protocols for the submission of grid type EC2 jobs,
\Expr{euca3://} and \Expr{euca3s://}.
These protocols exist to work correctly when the resources do not support 
the \Param{InstanceInitiatedShutdownBehavior} parameter.
\Ticket{2974}

\item (This feature was added in version 7.9.1.)  
Added both a \Opt{-suppress\_notification},
a \Opt{-dont\_suppress\_notification} command line option,
and corresponding
\Macro{DAGMAN\_SUPPRESS\_NOTIFICATION} configuration variable
to \Condor{dagman} and \Condor{submit\_dag}.
This enables a user of DAGMan to stop email notification of job
events for jobs submitted by \Condor{dagman}. The value of
\MacroNI{DAGMAN\_SUPPRESS\_NOTIFICATION} defaults to \Expr{True},
so that jobs submitted
by \Condor{dagman} will not send email notification. 
\Ticket{3352}

\item The default for job notification email has changed
from \Expr{Complete} to \Expr{Never}. 
There is also a new configuration variable, \Macro{JOB\_DEFAULT\_NOTIFICATION},
which permits administrators to change the default for all jobs.
\Ticket{2155}

\item For platforms supporting cgroups,
resource limits can now be applied per job,
where a job may consist of multiple processes.
See section~\ref{sec:Resource-Limits-Cgroup} for details.
\Ticket{2734}

\end{itemize}

\noindent Configuration Variable and ClassAd Attribute Additions and Changes:

\begin{itemize}

\item The new configuration variable \Macro{MEMORY\_LIMIT}
supports implementing memory resource limits on a per-job basis under cgroups.
\Ticket{2734}

\end{itemize}

\noindent Bugs Fixed:

\begin{itemize}

\item \Condor{schedd} and \Condor{shadow} were not respecting the
\Macro{DAGManNodesMask} attribute. This caused extra events to be written to
the DAGMan node log.
\Ticket{3311}

\item Removed a spurious newline from the output of \Condor{submit}.
\Ticket{3316}

\item Fixed a bug that caused the \Condor{shadow} to set job attribute
\Attr{X509UserProxySubject} to the wrong value when the job's X.509
proxy file was updated. It incorrectly set the value to be 
the proxy's subject name, rather than to the correct value, which is
its identity.
\Ticket{3265}

\item The \Prog{batch\_gahp} no longer modifies the environment variable
\Env{LD\_LIBRARY\_PATH}.
In some instances, modifying \Env{LD\_LIBRARY\_PATH} caused the
batch system's command line tools to fail when run by the \Prog{batch\_gahp}.
\Ticket{3317}

\item Grid-type \SubmitCmd{batch} jobs now work properly on machines
where the gLite software has been installed.
\Ticket{3269}

\item The \Condor{shadow} would never print the allocated amount of
partitionable resources in the job log.
\Ticket{3318}

\item \Condor{who} would sometimes incorrectly display blank or partial
values in the PROGRAM column.
\Ticket{3314}

\end{itemize}

\noindent Known Bugs:

\begin{itemize}

\item None.

\end{itemize}

\noindent Additions and Changes to the Manual:

\begin{itemize}

\item None.

\end{itemize}


%%%%%%%%%%%%%%%%%%%%%%%%%%%%%%%%%%%%%%%%%%%%%%%%%%%%%%%%%%%%%%%%%%%%%%
\subsection*{\label{sec:New-7-9-1}Version 7.9.1}
%%%%%%%%%%%%%%%%%%%%%%%%%%%%%%%%%%%%%%%%%%%%%%%%%%%%%%%%%%%%%%%%%%%%%%

\noindent Release Notes:

\begin{itemize}

\item Condor version 7.9.1 released on October 22, 2012.

\item Condor no longer looks for its main configuration file in the
location \File{\MacroUNI{GLOBUS\_LOCATION}/etc/condor\_config}.
\Ticket{2830}

\item \Security This version contains an important security bug fix.  See below
for details of this and other bugs fixed.

\end{itemize}


\noindent New Features:

\begin{itemize}

\item There are two new protocols for the submission of grid type EC2 jobs,
\Expr{euca3://} and \Expr{euca3s://}.
These protocols exist to work correctly when the resources do not support 
the \Param{InstanceInitiatedShutdownBehavior} parameter.
\Ticket{2974}

\item \Condor{job\_router} can now submit the routed copy of jobs to a
different \Condor{schedd} than the one that serves as the source of
jobs to be routed.  The spool directories of the two
\Condor{schedds} must still be directly accessible to
\Condor{job\_router}.  This feature is enabled by using the new
optional configuration settings:

\begin{itemize}
\item \Macro{JOB\_ROUTER\_SCHEDD1\_SPOOL}
See definition at section~\ref{param:JobRouterSchedd1Spool}.
\item \Macro{JOB\_ROUTER\_SCHEDD2\_SPOOL}
See definition at section~\ref{param:JobRouterSchedd2Spool}.
\item \Macro{JOB\_ROUTER\_SCHEDD1\_NAME}
See definition at section~\ref{param:JobRouterSchedd1Name}.
\item \Macro{JOB\_ROUTER\_SCHEDD2\_NAME}
See definition at section~\ref{param:JobRouterSchedd2Name}.
\item \Macro{JOB\_ROUTER\_SCHEDD1\_POOL}
See definition at section~\ref{param:JobRouterSchedd1Pool}.
\item \Macro{JOB\_ROUTER\_SCHEDD2\_POOL}
See definition at section~\ref{param:JobRouterSchedd2Pool}.
\end{itemize}
\Ticket{3030}

\item The \Condor{job\_router} can now optionally transform jobs in place,
rather than creating a second transformed version (copy) of the job.
\Ticket{3185}

\item The \Condor{defrag} daemon now has a policy option implemented
by configuration to cancel the draining
of a machine that is in the Draining mode.  This can be used to effect
partial draining of machines.
\Ticket{2993}

\item Communication between the \Condor{c-gahp} and the \Condor{schedd} has
been improved. A large number of Condor-C jobs should no longer cause
other clients of the remote \Condor{schedd} to time out trying to get the
\Condor{schedd} daemon's attention.
\Ticket{2575}

\item \Condor{history} and \Condor{q} can now be told to read job records
from a user log, instead of parsing the history file or querying the
\Condor{schedd}.  This can be used to monitor the status of jobs with
reduced load on the \Condor{schedd}.
\Ticket{3188}

\item Eucalyptus 3.x support has been added to the EC2 GAHP.
\Ticket{2974}

\item File transfer remaps now support remapping directories.
\Ticket{3039}

\item The \Condor{schedd} can now dynamically spawn a local \Condor{startd}
to manage local universe jobs.
\Ticket{3129}

\item \Condor{q} \Opt{-jobads} will now respect the \Opt{-constraint} option.
\Ticket{3191}

\item Added Bosco, a set of tools that makes it easy to use a Personal
Condor to run jobs on remote batch systems without administrator
assistance or manual installation of software on the remote systems.
See \URL{https://twiki.grid.iu.edu/bin/view/CampusGrids/BoSCO} for more
information about Bosco.
\Ticket{2421}

\end{itemize}


\noindent Configuration Variable and ClassAd Attribute Additions and Changes:

\begin{itemize}

\item Dynamic slots now fill the values for attributes of with names
that begin with
\Attr{TotalSlot}, 
for configured local resources in a way consistent with standard resources
such as \Attr{TotalSlotCpus}.
Previously those values were all given the value zero on dynamic slots.
\Ticket{3229}

\item The \Condor{schedd} now advertises the value of configuration variable
\MacroNI{COLLECTOR\_HOST} as attribute \Attr{CollectorHost} in 
its daemon ClassAd.  This allows one to determine if a given
\Condor{schedd} reporting to a \Condor{collector} is flocking to that 
\Condor{collector} or not.
\Ticket{3202}

\item Added the attribute \Attr{DAGManNodesMask} to control the verboseness of
the log referred to by \Attr{DAGManNodesLog}.
\Ticket{3351}

\item The new configuration variable
\Macro{QUEUE\_SUPER\_USER\_MAY\_IMPERSONATE} specifies a regular
expression that matches the user names that
the queue super user may impersonate when managing jobs.  When not
set, the default behavior is to allow impersonation of any user who
has had a job in the queue during the life of the \Condor{schedd}.  For
proper functioning of the \Condor{shadow}, the \Condor{gridmanager}, and
the \Condor{job\_router}, this expression, if set, must match the owner
names of all jobs that these daemons will manage.
\Ticket{3030}

\item The new configuration variable \Macro{DEFRAG\_CANCEL\_REQUIREMENTS}
is an expression that specifies which draining machines should have 
draining be canceled.  
This defaults to \MacroUNI{DEFRAG\_WHOLE\_MACHINE\_EXPR}.  
This could be used to drain partial rather than whole machines.
\Ticket{2993}

\item The new submit command \SubmitCmd{use\_x509userproxy} can be set
to \Expr{True} to indicate that an X.509 user proxy is required for the job. 
If \SubmitCmd{x509userproxy} is not set, 
then the proxy file will be looked for in the standard locations.
\Ticket{3025}

\item If \Condor{submit} is used to submit an interactive job,
and the job is interrupted before the interactive job starts,
an attempt is made to immediately remove the interactive job from the queue.
Similarly, \Condor{ssh\_to\_job} has a new option \Opt{-remove-on-interrupt}.
\Ticket{3242}

\item Changes to were made to the ClassAd machine attributes 
\Attr{OpSys}, \Attr{OpSysVer}, \Attr{Distro}, as well as others,
in order to do a better job of identifying the operating system.
\Ticket{2366}

\item \Macro{GRIDMANAGER\_MAX\_SUBMITTED\_JOBS\_PER\_RESOURCE} can now be a
list, specifying different values for different hosts.
\Ticket{3220}

\item The new configuration parameter \Macro{GRIDMANAGER\_JOB\_PROBE\_RATE}
limits the number of job status requests sent to each remote resource.
\Ticket{3023}

\item The default value of \Macro{GRIDMANAGER\_JOB\_PROBE\_INTERVAL} has
changed from 300 to 60.
\Ticket{3023}.

\item The configuration parameters \Macro{CONDOR\_JOB\_POLL\_INTERVAL} and
\Macro{INFN\_JOB\_POLL\_INTERVAL} should no longer be used. Use
\Macro{GRIDMANAGER\_JOB\_PROBE\_INTERVAL\_CONDOR} and
\Macro{GRIDMANAGER\_JOB\_PROBE\_INTERVAL\_BATCH} instead.
\Ticket{3023}

\end{itemize}

\noindent Bugs Fixed:

\begin{itemize}

\item \Security Fixed a bug which allowed jobs submitted to the standard
universe to escalate privilege on the submit machine and execute code as 
\Login{root}.
(CVE-2012-5390)
\Ticket{3268}

\item A fix only invokes Globus callouts when actually needed, 
thereby avoiding a program segfault if
the call out mechanism is misconfigured or broken.
\Ticket{2104}

\item Fixed a bug in all daemons wherein the \Attr{DaemonStartTime} attribute 
in the ClassAd for all daemons would be reset to the current time when they are
reconfigured.
\Ticket{3235}

\item Fixed a bug wherein the \Arg{-dont\_use\_default\_node\_log} command line flag
to \Condor{submit\_dag} had no effect.
\Ticket{3352}

\item \Security Although not user-visible, 
there were multiple updates that removed places
in the code where potential buffer overruns could occur, 
thus preventing potential attacks.  
None of these overruns were known to be exploitable.

\item \Security Although not user-visible, 
there were updates to the code to improve
the error checking of system calls,
thereby removing some potential security threats.  
None were known to be exploitable.

\item \Security Although not user-visible, 
removed some code that was no longer used.
The presence of this code could have led to a Denial-of-Service attack,
which would allow an attacker to stop another user's jobs from running.

\item \Security Filesystem (FS) authentication was improved to check 
the UNIX permissions of the directory used for authentication.  
Without this, an attacker may have
been able to impersonate another submitter on the same submit machine.

\item The \Condor{negotiator} now checks the accountant log file for sanity
once only on start up,  
thereby increasing efficiency of iteration through 
the accountant ClassAd log structure.
\Ticket{3011}

\item The ClassAd functions \Procedure{splitUserName} and 
\Procedure{splitSlotName}
no longer leak a small amount of memory each time they are evaluated.  
This bug was introduced when these functions were added in Condor version 7.7.6.
\Ticket{3082}

\item There are several bug fixes for grid-type batch jobs:
  \begin{itemize}
  \item Monitoring the status of jobs submitted to PBS and SGE has been
    improved. \Ticket{3067} \Ticket{3157} \Ticket{3181}
  \item Job command-line arguments containing 
    left parenthesis, \verb@(@, right parenthesis, \verb@)@, 
    and ampersand, \verb@&@, characters are now handled properly. 
    \Ticket{3057}
  \item Removing PBS jobs that have just completed no longer causes the jobs
    to become held. \Ticket{3016}
  \item Added a work-around for a bug when submitting jobs to
    a Condor pool running Condor versions 7.7.6 through 7.8.2.
    A bug in \Condor{history} \Opt{-f} caused an error in determining
    a job's status.
    \Ticket{3133}
  \item Improved the handling of job files when the batch system has a shared
    file system. \Ticket{3195}
  \end{itemize}

\item Changes introduced in Condor version 7.9.0 caused jobs submitted by
\Condor{dagman} in the local universe to not write to the default node log file,
when \Macro{DAGMAN\_ALWAYS\_USE\_NODE\_LOG} was \Expr{True} (the default),
and a user log was also defined. This is fixed. 
\Ticket{3111}

\item Fixed a bug introduced in Condor version 7.9.0 that caused grid type
cream jobs to be held with a hold reason of 
\footnotesize
\begin{verbatim}
  CREAM_Delegate Error: Cannot set credentials in the gsoap-plugin context.
\end{verbatim}
\normalsize
\Ticket{3234}

\item Fixed a problem that could have caused the \Condor{collector} to crash
when receiving an invalid packet.
\Ticket{3161}

\end{itemize}

\noindent Known Bugs:

\begin{itemize}

\item None.

\end{itemize}

\noindent Additions and Changes to the Manual:

\begin{itemize}

\item None.

\end{itemize}


%%%%%%%%%%%%%%%%%%%%%%%%%%%%%%%%%%%%%%%%%%%%%%%%%%%%%%%%%%%%%%%%%%%%%%
\subsection*{\label{sec:New-7-9-0}Version 7.9.0}
%%%%%%%%%%%%%%%%%%%%%%%%%%%%%%%%%%%%%%%%%%%%%%%%%%%%%%%%%%%%%%%%%%%%%%

\noindent Release Notes:

\begin{itemize}

\item Condor version 7.9.0 released on August 16, 2012.

\end{itemize}


\noindent New Features:

\begin{itemize}

\item Machine slots can now be configured to identify and
divide customized local resources.
Jobs may then request these resources.
See section~\ref{sec:Configuring-SMP} for details.
\Ticket{2905}

\item Condor now supports and implements the caching of ClassAds 
to reduce memory footprints. 
This feature is experimental and is currently disabled by default.
It can be enabled by setting
the new configuration variable \Macro{ENABLE\_CLASSAD\_CACHING}
to \Expr{True}.
\Ticket{2541}
\Ticket{3127}

\item \Condor{status} now returns the \Condor{schedd} ClassAd directly 
from the \Condor{schedd} daemon,
if both options \Opt{-direct} and \Opt{-schedd} are given on the command line.
\Ticket{2492}

\item The new \Opt{-status} and \Opt{-echo} command line options to 
\Condor{wait} command cause it to show job start and terminate information,
and to print events to \Code{stdout}.
\Ticket{2926}

\item Added a \Expr{DEBUG} logging level output flag \Dflag{CATEGORY},
which causes Condor to include the logging level
flags in effect for each line of logged output.
\Ticket{2712}

\item \Condor{status} and \Condor{q} each have a new \Opt{-autoformat} option
to make some output format specifications easier than the existing
\Opt{-format} option.
See the \Condor{status} manual page located on page~\pageref{man-condor-status}
and the \Condor{q} manual page located on page~\pageref{man-condor-q} 
for details.
\Ticket{2941}

\item Enhanced the ClassAd log system to report the log line number 
on parse failures, 
and improved the ability to detect parse failures closer to 
the point of corruption.
\Ticket{2934}

\item Added an \Opt{-evaluate} option to \Condor{config\_val}, which causes the configured value queried from
a given daemon to be evaluated with respect to that daemon's ClassAd.
\Ticket{856}

\item Added code to \Condor{dagman},
such that a \Expr{VARS} assignment in a top-level DAG is applied to splices.
\Ticket{1780}

\item Condor now uses libraries from Globus 5.2.1.
\Ticket{2838}

\item When authenticating Condor daemons with GSI and
configuration variable \MacroNI{GSI\_DAEMON\_NAME} is undefined, 
Condor checks that the server name in the certificate matches the 
host name that the client is connecting to. 
When \MacroNI{GSI\_DAEMON\_NAME} is defined,
the old behavior is preserved: only certificates matching
\MacroNI{GSI\_DAEMON\_NAME} pass the authentication step, 
and no host name check is performed.  
The behavior of the host name check
may be further controlled with the new configuration variables
\MacroNI{GSI\_SKIP\_HOST\_CHECK} and
\MacroNI{GSI\_SKIP\_HOST\_CHECK\_CERT\_REGEX}.
\Ticket{1605}

\item Added new capability to \Condor{submit} to allow recursive macros in
submit description files. 
This facility allows one to update variables recursively. 
Before this new capability was added,
recursive definition would send \Condor{submit} into an
infinite loop of expanding the macro,
such that the expansion would fill up memory.
See section~\ref{macro-in-submit-description-file} for details.
\Ticket{406}

\item A DAGMan limitation and restriction has been removed.  
It is now permitted to define a \SubmitCmd{log} command using a macro,
within a node job's submit description file.
\Ticket{2428}

\item To enforce the dependencies of a DAG,
DAGMan now uses and watches only the default node
user log of the \Condor{dagman} job for events.  
DAGMan requests the \Condor{schedd} and \Condor{shadow} daemons to write each
event to this default log, 
in addition to writing to a log specified by the node job.
\Condor{dagman} writes POST script terminate events only to its default log;
these terminate events are not written to the user log.
This behavior can be turned off by setting the configuration variable
\Macro{DAGMAN\_ALWAYS\_USE\_NODE\_LOG} to \Expr{False}.
For correct behavior,
\MacroNI{DAGMAN\_ALWAYS\_USE\_NODE\_LOG} should be set to \Expr{False}
if \Condor{dagman} version 7.9.0 or later is submitting jobs 
to an older version of
a \Condor{schedd} daemon or of a \Condor{submit} executable.
\Ticket{2807}

\item \Condor{submit} has a new \Opt{-interactive} option for
platforms other than Windows,
which schedules and runs a job that provides a shell prompt
on the execute machine.
\Ticket{3088}

\end{itemize}

\noindent Configuration Variable and ClassAd Attribute Additions and Changes:

\begin{itemize}

\item The new configuration variables \Macro{MACHINE\_RESOURCE\_NAMES}
(see section~\ref{param:MachineResourceNames})
and \Macro{MACHINE\_RESOURCE\_<name>}
(see section~\ref{param:MachineResourceResourcename})
identify and specify the use of customized local machine resources.
\Ticket{2905}

\item The new configuration variable \MacroNI{ENABLE\_CLASSAD\_CACHING}
controls whether the new caching feature of ClassAds is used.
The default value is \Expr{False}.
\Ticket{3127}

\item The new configuration variable \Macro{CLASSAD\_LOG\_STRICT\_PARSING}
controls whether ClassAd log files such as the job queue
log are read with strict parse checking on ClassAd expressions.
\Ticket{3069}

\item The default value for configuration variable \Macro{USE\_PROCD}
is now \Expr{True} for the \Condor{master} daemon.  
This means that by
default the \Condor{master} will start a \Condor{procd} daemon to be used 
by it and all other daemons on that machine.
\Ticket{2911}

\item There is a new configuration variable used by the \Condor{starter}.
If \Macro{STARTER\_RLIMIT\_AS} is set to an integer value, 
the \Condor{starter}
will use the \Procedure{setrlimit} system call with the 
\Code{RLIMIT\_AS} parameter to
limit the virtual memory size of each process in the user job.  
The value of this configuration variable is a ClassAd expression, 
evaluated in the context of both the machine and job ClassAds, 
where the machine ClassAd is the \Expr{my} ClassAd, 
and the job ClassAd is the \Expr{target} ClassAd.
\Ticket{1663}

\item New configuration variables were added to to the \Condor{schedd} to
define statistics that count subsets of jobs. 
These variables have the form \Macro{SCHEDD\_COLLECT\_STATS\_BY\_<Name>},
and should be defined by a ClassAd expression that evaluates to a string.
See section~\ref{param:ScheddCollectStatsByName}
for the complete definition.
The optional configuration variable of the form
\Macro{SCHEDD\_EXPIRE\_STATS\_BY\_<Name>} can be used to set an expiration time,
in seconds, for each set of statistics.
\Ticket{2862}

\item The new \SubmitCmd{batch\_queue} submit description file command
and new job ClassAd attribute \Attr{BatchQueue} specify which job
queue to use for grid universe jobs of type
\SubmitCmd{pbs}, \SubmitCmd{lsf}, and \SubmitCmd{sge}.
\Ticket{2996}

\item The new configuration variable \Macro{GSI\_SKIP\_HOST\_CHECK} is
a boolean that controls whether a check is performed during
GSI authentication of a Condor daemon.  
When the default value \Expr{False},
the check is not skipped, so the daemon host name must match the
host name in the daemon's certificate, unless otherwise exempted
by values of \MacroNI{GSI\_DAEMON\_NAME} or
\MacroNI{GSI\_SKIP\_HOST\_CHECK\_CERT\_REGEX}.
When \Expr{True}, this check is skipped, and hosts will not be rejected
due to a mismatch of certificate and host name.
\Ticket{1605}

\item The new configuration variable
\MacroNI{GSI\_SKIP\_HOST\_CHECK\_CERT\_REGEX} may be set to a
regular expression.  GSI certificates of Condor daemons with a
subject name that are matched in full by this regular expression
are not required to have a matching daemon host name and certificate
host name.  The default is an empty regular expression, which will
not match any certificates, even if they have an empty subject name.
\Ticket{1605}

\end{itemize}

\noindent Bugs Fixed:

\begin{itemize}

\item Fixed a bug in which usage of cgroups incorrectly included the page cache 
in the maximum memory usage.
This bug fix is also included in Condor version 7.8.2.
\Ticket{3003}

\item The EC2 GAHP will now respect the value of the environment variable
\Env{X509\_CERT\_DIR} and the configuration variable
\Macro{GSI\_DAEMON\_TRUSTED\_CA\_DIR} for \emph{all} secure connections.
\Ticket{2823}

\item Condor will avoid selecting down (disabled) network interfaces.  Previously Condor could select a down interface over an up (active) interface.
\Ticket{2893}

\item Made logic in the \Condor{negotiator} that computes submitter limits 
properly aware of the configuration variable
\Macro{NEGOTIATOR\_CONSIDER\_PREEMPTION}.
\Ticket{2952}


\item Condor no longer back-dates file modification times by 3 minutes
when transferring job input files into the job spool directory or the job
execute directory.
\Ticket{2423}

\item Fixed a bug in which the use of a pipe in the configuration file 
on Windows platforms would cause a visible console window 
to show up whenever the configuration was read.
\Ticket{1534}

\end{itemize}

\noindent Known Bugs:

\begin{itemize}

\item None.

\end{itemize}

\noindent Additions and Changes to the Manual:

\begin{itemize}

\item Machine ClassAd attribute string values relating to \Attr{OpSys} have
been documented for Scientific Linux platforms.
\Ticket{2366}

\end{itemize}


%%%%      PLEASE RUN A SPELL CHECKER BEFORE COMMITTING YOUR CHANGES!
%%%      PLEASE RUN A SPELL CHECKER BEFORE COMMITTING YOUR CHANGES!
%%%      PLEASE RUN A SPELL CHECKER BEFORE COMMITTING YOUR CHANGES!
%%%      PLEASE RUN A SPELL CHECKER BEFORE COMMITTING YOUR CHANGES!
%%%      PLEASE RUN A SPELL CHECKER BEFORE COMMITTING YOUR CHANGES!

%%%%%%%%%%%%%%%%%%%%%%%%%%%%%%%%%%%%%%%%%%%%%%%%%%%%%%%%%%%%%%%%%%%%%%
\section{\label{sec:History-7-8}Stable Release Series 7.8}
%%%%%%%%%%%%%%%%%%%%%%%%%%%%%%%%%%%%%%%%%%%%%%%%%%%%%%%%%%%%%%%%%%%%%%

This is a stable release series of HTCondor.
As usual, only bug fixes (and potentially, ports to new platforms)
will be provided in future 7.8.x releases.
New features will be added in the 7.9.x development series.

The details of each version are described below.

%%%%%%%%%%%%%%%%%%%%%%%%%%%%%%%%%%%%%%%%%%%%%%%%%%%%%%%%%%%%%%%%%%%%%%
\subsection*{\label{sec:New-7-8-8}Version 7.8.8}
%%%%%%%%%%%%%%%%%%%%%%%%%%%%%%%%%%%%%%%%%%%%%%%%%%%%%%%%%%%%%%%%%%%%%%

\noindent Release Notes:

\begin{itemize}

\item HTCondor version 7.8.8 released on March 28, 2013.

\end{itemize}


\noindent New Features:

\begin{itemize}

\item When using \Prog{glexec}, HTCondor now automatically retries
each \Prog{glexec} operation if \Prog{glexec} exits with an error
code that is likely to be caused by a transient error, such as a
communication error with the mapping service.  Previously, any
\Prog{glexec} error would cause the job to be put on hold.  Now, the
job will only go on hold if the maximum number of \Prog{glexec}
retries is exceeded.
\Ticket{2415}

\end{itemize}

\noindent Configuration Variable and ClassAd Attribute Additions and Changes:

\begin{itemize}

\item The new configuration variable \Macro{GLEXEC\_RETRIES} is an
integer value that specifies the maximum number of times to retry a
call to \Prog{glexec} when \Prog{glexec} exits with status 202 or
203, error codes that indicate a possible transient error condition.
The default number of retries is 3.
\Ticket{2415}

\item The new configuration variable \Macro{GLEXEC\_RETRY\_DELAY} is
an integer value that specifies the minimum number of seconds to
wait between retries of a failed call to \Prog{glexec}.
The default is 5 seconds.
The actual delay to be used is determined by a random exponential
backoff algorithm that chooses a delay with a minimum of
the value of \MacroNI{GLEXEC\_RETRY\_DELAY} 
and a maximum of 100 times that value.
\Ticket{2415}

\end{itemize}

\noindent Bugs Fixed:

\begin{itemize}

\item Fixed a bug that caused the \Condor{gridmanager} to frequently
delegate the X.509 proxy for jobs of the condor grid type to the remote
\Condor{schedd} when the delegated proxy's lifetime is not limited.
This occurred when configuration variable
\Macro{DELEGATE\_JOB\_GSI\_CREDENTIALS\_LIFETIME}
or job ClassAd attribute \Attr{DelegateJobGSICredentialsLifetime} was set to 0.
\Ticket{3395}

\item Fixed a bug in \Condor{advertise} that could cause failure to
publish ClassAds to \Condor{collector} daemons other than the first 
one in the list of \Condor{collector} daemons.
\Ticket{3404}

\item (This bug was fixed in HTCondor version 7.8.4.)
\Condor{userlog} no longer ignores \Expr{ULOG\_JOB\_RECONNECT\_FAILED} events.
\Ticket{3215}

\item Fixed a bug that caused cream grid jobs to become held with a
\Attr{HoldReason} attribute stating
\footnotesize
\begin{verbatim}
"CREAM error: Transfer failed: globus_ftp_client: 
  an invalid value for url was used"
\end{verbatim}
\normalsize
when the jobs did not have any input or output files to transfer.
\Ticket{3415}

\item Fixed a bug that could cause HTCondor daemons to abort on 
\Condor{reconfig} when the value of configuration variable 
\Macro{STATISTICS\_WINDOW\_SECONDS} was reduced.
\Ticket{3443}

\item Fixed a bug that could cause daemons using CCB to fail to reconnect to
the CCB server after becoming disconnected.  This condition would cause the
daemon to become unreachable via CCB until the daemon was restarted.
\Ticket{3476}

\item If \Condor{shared\_port} was using a dynamic port and the
\Condor{master} was using the shared port, 
then if \Condor{shared\_port} died, all
subsequent attempts to restart it on a different port failed.
\Ticket{3478}

\item The \Prog{nordugrid\_gahp} no longer exits if the job ClassAd contains
a bad \Attr{NordugridRSL} attribute value.
\Ticket{3495}

\item Fixed a bug that caused the command-line arguments of grid universe
jobs of grid type cream 
to be ignored if specified using the new quoting syntax in
the submit description file.
\Ticket{3473}

\item Reduced the likelihood of a problem that caused the
\Condor{master} to restart some of its children after a recent
reconfiguration, because the \Condor{master} incorrectly concluded that the
children were hung.
\Ticket{3510}

\end{itemize}

\noindent Known Bugs:

\begin{itemize}

\item None.

\end{itemize}

\noindent Additions and Changes to the Manual:

\begin{itemize}

\item None.

\end{itemize}


%%%%%%%%%%%%%%%%%%%%%%%%%%%%%%%%%%%%%%%%%%%%%%%%%%%%%%%%%%%%%%%%%%%%%%
\subsection*{\label{sec:New-7-8-7}Version 7.8.7}
%%%%%%%%%%%%%%%%%%%%%%%%%%%%%%%%%%%%%%%%%%%%%%%%%%%%%%%%%%%%%%%%%%%%%%

\noindent Release Notes:

\begin{itemize}

\item HTCondor version 7.8.7 released on December 18, 2012.

\end{itemize}

\noindent New Features:

\begin{itemize}

\item None.

\end{itemize}

\noindent Configuration Variable and ClassAd Attribute Additions and Changes:

\begin{itemize}

\item None.

\end{itemize}

\noindent Bugs Fixed:

\begin{itemize}

\item Fixed a bug wherein running the \Condor{suspend} command on a scheduler
universe job would cause the \Condor{schedd} to crash.
\Ticket{3259}

\item HTCondor now places EC2 jobs on hold when they fail to authenticate,
rather than leaving them and other jobs with the same authentication tokens
idle indefinitely.
\Ticket{2274}

\item For EC2 resource jobs within the grid universe,
HTCondor now only destroys the user's half of the SSH keypair
when sure that the instance has terminated.  This prevents transient
problems from rendering an instance permanently inaccessible.
\Ticket{3289}

\item For EC2 resource jobs within the grid universe,
HTCondor no longer generates an SSH keypair if the user did not
request one.
\Ticket{3061}

\item HTCondor no longer generates SSH keypair names that are incompatible
with OpenStack v4 (Essex).
\Ticket{3060}

\item Jobs that were submitted with \Condor{submit} \Opt{-spool}
and failed during submission were 
left indefinitely in the queue in the Hold state.  
This has been fixed, such that these jobs are removed from the queue.
In addition, the \Condor{schedd} daemon will periodically 
check for jobs that have
been in Hold state due to failed file transfer for at least twelve hours;
these jobs will be removed from the queue.
\Ticket{3200}

\item Fixed a problem where an \Prog{ssh\_to\_job} or an interactive job 
session would be terminated prematurely if the execute machine 
was configured to track process trees via a dedicated login.
That is, when configuration variable 
\Macro{DEDICATED\_EXECUTE\_ACCOUNT\_REGEXP} is being used.  
\Ticket{3232}

\item ClassAd functions \Procedure{int} and \Procedure{real} now ignore 
trailing characters within
a string argument that contains a valid number.
\Ticket{3102}

\item Contrary to the intended behavior, jobs run via \Prog{glexec} did not
get put on hold shortly before their proxy expired.
\Ticket{3283}

\item Starting in version 7.8.0, when using \Prog{glexec}, 
the job was put on hold shortly after the user's proxy was refreshed.
The incorrect, but stated hold reason was, \Expr{"Proxy about to expire."}
\Ticket{3280}

\item The configuration variable
\Macro{DELEGATE\_JOB\_GSI\_CREDENTIALS\_REFRESH} had no effect,
as it was implemented using the incorrect variable name of
\MacroNI{DELEGATE\_JOB\_GSI\_CREDENTIALS\_RENEWAL}.
The implementation now uses the correct name.
\Ticket{3282}

\item When using privilege separation, jobs would be put on hold after
they finished running if the working directory contained links to
files that were not globally readable.
\Ticket{2904}

\item The configuration variable \MacroNI{ENABLE\_ADDRESS\_REWRITING}
was not correctly applied to the \Condor{schedd} address when
claiming a slot.  
The incorrect behavior observed was always as though
\MacroNI{ENABLE\_ADDRESS\_REWRITING} was \Expr{False}.
This could result
in communication errors for jobs running from multi-homed submit machines.
\Ticket{3330}

\item When using \Condor{defrag} or \Condor{drain}, a rare sequence of
events could result in the \Condor{startd} exiting with the
following error message:

\begin{verbatim}
ERROR "match_info() called with unexpected state (Drained)"
\end{verbatim}
\Ticket{3331}

\item The \Condor{gridmanager} no longer crashes when a CREAM grid job
is submitted with an X.509 proxy that does not have VOMS attributes.
\Ticket{3356}

\item \Condor{chirp} fetch could only transfer text files on Windows. It would
truncate or corrupt binary files.
\Ticket{3355}

\item Fixed a bug in the \Condor{gridmanager} that prevented the use
of grid-type \SubmitCmd{batch} to submit jobs into an HTCondor pool.
The \Condor{gridmanager} would attempt to use the wrong GAHP server.
\Ticket{3364}

\item The default for the undocumented configuration variable
\Macro{X\_RUNS\_HERE} was inverted from \Expr{True} to \Expr{False}
starting with the release of version 7.7.3. 
Its default has been reset to \Expr{True}. 
When \Expr{False}, the \Condor{master} will not start the \Condor{kbdd}.
\Ticket{3343}

\item Fixed a bug in the init script for Red Hat derived Linux systems that
prevented the Condor service from being stopped during system shutdown.
\Ticket{3368}

\item The \Condor{master} would sometimes crash on reconfiguration when the
High Availability configuration had changed. It no longer crashes.
\Ticket{3292}

\item Fixed a bug that caused the \Condor{starter} to crash 
when \Condor{chirp} is used and there is a configuration variable
setting of \Expr{USE\_NFS = True} or \Expr{USE\_AFS = True}.
This will happen with parallel universe jobs,
because the MPI scripts invoke \Condor{chirp}.
\Ticket{3361}

\end{itemize}

\noindent Known Bugs:

\begin{itemize}

\item None.

\end{itemize}

\noindent Additions and Changes to the Manual:

\begin{itemize}

\item The manual incorrectly identified configuration variable
\Macro{COLLECTOR\_PERSISTENT\_AD\_LOG} as \MacroNI{PERSISTENT\_AD\_LOG}.
This has now been corrected throughout the manual.
\Ticket{3205}

\end{itemize}


%%%%%%%%%%%%%%%%%%%%%%%%%%%%%%%%%%%%%%%%%%%%%%%%%%%%%%%%%%%%%%%%%%%%%%
\subsection*{\label{sec:New-7-8-6}Version 7.8.6}
%%%%%%%%%%%%%%%%%%%%%%%%%%%%%%%%%%%%%%%%%%%%%%%%%%%%%%%%%%%%%%%%%%%%%%

\noindent Release Notes:

\begin{itemize}

\item Condor version 7.8.6 released on October 25, 2012.

\item \Security This version contains an important security bug fix.  See below
for details of this and other bugs fixed.

\end{itemize}

\noindent Bugs Fixed:

\begin{itemize}

\item \Security Fixed a bug which allowed jobs submitted to the standard
universe to escalate privilege on the submit machine and execute code as root.
(CVE-2012-5390)

\end{itemize}


%%%%%%%%%%%%%%%%%%%%%%%%%%%%%%%%%%%%%%%%%%%%%%%%%%%%%%%%%%%%%%%%%%%%%%
\subsection*{\label{sec:New-7-8-5}Version 7.8.5}
%%%%%%%%%%%%%%%%%%%%%%%%%%%%%%%%%%%%%%%%%%%%%%%%%%%%%%%%%%%%%%%%%%%%%%

\noindent Release Notes:

\begin{itemize}

\item Condor version 7.8.5 released on October 22, 2012.

\end{itemize}


\noindent New Features:

\begin{itemize}

\item Condor now contains a tool called \Prog{accountant\_log\_fixer}, 
that can fix the damage to the file \File{Accountantnew.log} 
caused by a bug in the Condor version 7.8.4 \Condor{negotiator}.
\Ticket{3221}

\end{itemize}

\noindent Configuration Variable and ClassAd Attribute Additions and Changes:

\begin{itemize}

\item None.

\end{itemize}

\noindent Bugs Fixed:

\begin{itemize}

\item Fixed a problem with jobs that are submitted with
\begin{verbatim}
  noop_job = true
\end{verbatim}
These jobs would, in rare cases, cause the \Condor{schedd} daemon to crash.  
\Ticket{3156}

\item The \Condor{startd} daemon crashed if a job failed to match 
a partitionable slot after the application 
of configuration variables of the \MacroNI{MODIFY\_REQUEST\_EXPR\_} category.
\Ticket{3260}

\item The \Condor{schedd} daemon would mark scheduler universe jobs
as completed and remove them from the job queue,
even when they should have been requeued, according to policy.
This caused \Condor{dagman} jobs to fail to restart. This bug exists
in all Condor versions 7.8.0 through 7.8.4.
Upon upgrading from these Condor versions, users will need to intervene
in order to restart their \Condor{dagman} jobs. \Condor{dagman} should not
need such intervention when upgrading from Condor version 7.8.5.  
To restart a \Condor{dagman} job, 
the simplest solution is to issue the command
\begin{verbatim}
  condor_submit <DAGFile>.condor.sub
\end{verbatim}
where the original DAG was submitted with 
\begin{verbatim}
  condor_submit <DAGFile>
\end{verbatim}
\Ticket{3207}

\item The Condor version 7.8.4 \Condor{negotiator} daemon wrote 
corrupt resource entries to the file \File{Accountantnew.log},
which it would then not be able to read.
The Condor version 7.9.0 \Condor{negotiator} daemon will abort 
when trying to read these corrupted resource entries. 
The \Condor{negotiator} will now correct these corrupt
resource entries over time.
\Ticket{3221}

\item Fixed a bug in which the \Condor{schedd} statistics ClassAd attributes
\Attr{JobsAccumExecuteTime}
and \Attr{JobsAccumPostExecuteTime} were sometimes much too large for jobs
that had been vacated and then restarted.
Note that these currently undocumented attributes would only appear
in the ClassAd if the verbosity level for \Condor{schedd} statistics
was set at the high value of 2 by the configuration variable
\MacroNI{STATISTICS\_TO\_PUBLISH}.
\Ticket{3227}

\end{itemize}

\noindent Known Bugs:

\begin{itemize}

\item None.

\end{itemize}

\noindent Additions and Changes to the Manual:

\begin{itemize}

\item None.

\end{itemize}


%%%%%%%%%%%%%%%%%%%%%%%%%%%%%%%%%%%%%%%%%%%%%%%%%%%%%%%%%%%%%%%%%%%%%%
\subsection*{\label{sec:New-7-8-4}Version 7.8.4}
%%%%%%%%%%%%%%%%%%%%%%%%%%%%%%%%%%%%%%%%%%%%%%%%%%%%%%%%%%%%%%%%%%%%%%

\noindent Release Notes:

\begin{itemize}

\item Condor version 7.8.4 released on September 19, 2012.

\item This release contains several important security fixes and all users should upgrade as soon as possible.

\end{itemize}


\noindent New Features:

\begin{itemize}

\item None.

\end{itemize}

\noindent Configuration Variable and ClassAd Attribute Additions and Changes:

\begin{itemize}

\item The new configuration variable \Macro{GSI\_AUTHZ\_CONF}
fixes a bug in which an instance of Condor may utilize the  
wrong Globus mapping.
The configuration variable defines a path and file name 
to the file that contains the Globus mapping library. 
See the complete definition at
~\ref{param:GSIAuthzConf}.
\Ticket{2103}


\end{itemize}

\noindent Bugs Fixed:

\begin{itemize}

\item \Security Some code that was no longer used was removed.  The presence
of this code could expose information which would allow an attacker to control
another user's job.  (CVE-2012-3493)

\item \Security Some code that was no longer used was removed.  The presence
of this code could have lead to a Denial-of-Service attack which would allow
an attacker to remove another user's idle job.  (CVE-2012-3491)

\item \Security Filesystem (FS) authentication was improved to check the UNIX
permissions of the directory used for authentication.  Without this, an
attacker may have been able to impersonate another submitter on the same submit
machine.  (CVE-2012-3492)

\item \Security Although not user-visible, there were multiple updates to
remove places in the code where potential buffer overruns could occur, thus
removing potential attacks.  None were known to be exploitable.

\item \Security Although not user-visible, there were updates to the code to
improve error checking of system calls, removing some potential security
threats.  None were known to be exploitable.


% https://access.redhat.com/security/cve/CVE-2012-349X


\item Fixed the \Condor{schedd} daemon; 
it would crash when a submit description file
contained a malformed \verb@$$()@ expansion macro that contained
a period.
\Ticket{3216}

\item Fixed a case in which a daemon could crash and leave behind a log
file owned by \Login{root}.  This \Login{root}-owned file would then cause
subsequent attempts to restart the daemon to fail.
\Ticket{2894}

\item Fixed a special case bug in which configuration variables
defined utilizing initial substrings of \Expr{\$(DOLLAR)},
for example  \Expr{\$(D)} and  \Expr{\$(DO)},  
were not expanded properly.
\Ticket{3217}

\item The command \Condor{q} \Opt{-run} now displays correct HOST field 
information for local universe jobs.
\Ticket{3150}

\end{itemize}

\noindent Known Bugs:

\begin{itemize}

\item None.

\end{itemize}

\noindent Additions and Changes to the Manual:

\begin{itemize}

\item None.

\end{itemize}


%%%%%%%%%%%%%%%%%%%%%%%%%%%%%%%%%%%%%%%%%%%%%%%%%%%%%%%%%%%%%%%%%%%%%%
\subsection*{\label{sec:New-7-8-3}Version 7.8.3}
%%%%%%%%%%%%%%%%%%%%%%%%%%%%%%%%%%%%%%%%%%%%%%%%%%%%%%%%%%%%%%%%%%%%%%

\noindent Release Notes:

\begin{itemize}

\item Condor version 7.8.3 released on September 6, 2012.

\end{itemize}


\noindent New Features:

\begin{itemize}

\item The \File{libcondorapi} library for reading and writing job event
logs is again available as a shared library on Linux and Mac OS platforms.
Since Condor 7.5.x, it had only been available as a static library.
\Ticket{3047}

\end{itemize}

\noindent Configuration Variable and ClassAd Attribute Additions and Changes:

\begin{itemize}

\item To avoid the output of an unnecessary DAGMan error message,
the value of \Macro{DAGMAN\_LOG\_ON\_NFS\_IS\_ERROR}
is ignored when both \MacroNI{CREATE\_LOCKS\_ON\_LOCAL\_DISK}
and \MacroNI{ENABLE\_USERLOG\_LOCKING} are \Expr{True}.
\Ticket{3087}

\end{itemize}

\noindent Bugs Fixed:

\begin{itemize}

\item Fixed a bug in which usage of cgroups incorrectly included the
page cache in the maximum memory usage.
This bug fix is also included in Condor version 7.9.0.
\Ticket{3003}

\item Jobs from a hook to fetch work, 
where the hook is defined by configuration variable 
\MacroNI{<Keyword>\_HOOK\_FETCH\_WORK},
now correctly receive dynamic slots from a partitionable slot 
instead of claiming the entire partitionable slot.
\Ticket{2819}

\item Fixed a bug in which a slot might become stuck in the Preempting state
when a \Condor{startd} is configured with a hook to fetch work,
as defined by \Macro{<Keyword>\_HOOK\_FETCH\_WORK}.
\Ticket{3076}

\item Fixed a bug that caused Condor to transfer a job's input files from
the execute machine back to the submit machine as if they were output files.
This would happen if the
job's input files were stored in Condor's spool directory;
occurred if the job was submitted via Condor-C or via 
\Condor{submit} with the \Opt{-spool} or \Opt{-remote} options.
\Ticket{2406}

\item Fixed a bug that could cause the first grid-type cream jobs destined 
for a particular CREAM server to never be submitted to that server.
This bug was probably introduced in Condor version 7.6.5.
\Ticket{3054}

\item Fixed several problems with the XML parsing class
\Code{ClassAdXMLParser} in the ClassAds library:
  \begin{itemize}
  \item Several methods named \Procedure{ParseClassAd} were declared, 
  but never implemented. 
\Ticket{3049}
  \item The parser silently dropped leading white space in string values.
\Ticket{3042}
  \item The parser could go into an infinite loop or leak memory when
    reading a malformed ClassAd XML document. 
\Ticket{3045}
  \end{itemize}

\item Fixed a bug that prevented the \Opt{-f} command line option to
\Condor{history} from being recognized.
The \Opt{-f} option was being interpreted as \Opt{-forward}. 
At least four letters are now required for the \Opt{-forward} option
(\Opt{-forw}) to prevent ambiguity.
\Ticket{3044}

\item The implementation of the \Condor{history} \Opt{-backwards} option, 
which is the default ordering for reading the history file,
in the 7.7 series did not work on Windows platforms.
This has been fixed.
\Ticket{3055}

\item Fixed a bug that caused an invalid proxy to be delegated when
refreshing the job's X.509 proxy when configuration variable
\Macro{DELEGATE\_JOB\_GSI\_CREDENTIALS\_LIFETIME} was set to 0.
\Ticket{3059}

\item Fixed a bug in which DAGMan did not account properly for jobs being
suspended and then unsuspended.
\Ticket{3108}

\item \Condor{dagman} now takes note of job reconnect failed 
events (event code 24) in the user log, for counting idle jobs.
\Ticket{3189}

\item Job IDs generated by NorduGrid ARC 12.05 and above are now
properly recognized.
\Ticket{3062}

\item Fixed a bug in which Condor would not mark grid-type nordugrid jobs
as Running due to variation in the format of the job status value.
NorduGrid ARC job statuses of the form \Expr{INLRMS: ?} are now
properly recognized both with and without the space after the colon.
\Ticket{3118}

\item The \Condor{gridmanager} now properly handles X.509 proxy files
that are specified in the job ClassAd with a relative path name.
\Ticket{3027}

\item Fixed a bug that caused daemon names,
as set in configuration variables such as \MacroNI{STARTD\_NAME},
containing a period character to be ignored.
\Ticket{3172}

\item Fixed a bug that prevented the \Condor{schedd} from removing old
execute directories for local universe jobs on start up.
\Ticket{3176}

\item The \Condor{defrag} daemon sometimes scheduled fewer draining attempts 
than specified.
\Ticket{3199}

\item Fixed a bug that could cause the \Condor{gridmanager} to crash if a
grid universe job's X.509 user certificate did not contain an e-mail
address.
\Ticket{3203}

\item Fixed a bug introduced in Condor version 7.7.5 that caused multiple
\Condor{schedd} daemons running on the same machine to share the job queue
with each other due to way in which the default value of configuration
variable \MacroNI{JOB\_QUEUE\_LOG} was set.
\Ticket{3196}

\item Fixed a bug that could cause \Condor{q} to not print all jobs when
it thought it was querying an old \Condor{schedd} daemon.
\Ticket{3206}

\item Fixed a bug that could cause a job's standard output and standard
error files to be written in the job's initial working directory,
despite the submit description file's specification to write them 
to a different directory.
This would happen when the file transfer mechanism was used, 
the execution machine was running Condor version 7.7.1 or earlier, 
and either Condor's security negotiation
was disabled or the configuration variable
\MacroNI{SEC\_ENABLE\_MATCH\_PASSWORD\_AUTHENTICATION} was set to \Expr{True}.
\Ticket{3208}

\item The log message generated when the \MacroNI{EXECUTE} directory
is missing is now more helpful.
\Ticket{3194}

\item The load average was incorrect for non-English versions on 
Windows platforms.
This has been fixed for Windows Vista and more recent versions.
\Ticket{3182}

\end{itemize}

\noindent Known Bugs:

\begin{itemize}

\item None.

\end{itemize}

\noindent Additions and Changes to the Manual:

\begin{itemize}

\item There is now documentation for the submit description file commands
\SubmitCmd{encrypt\_input\_files},
\SubmitCmd{encrypt\_output\_files},
\SubmitCmd{dont\_encrypt\_input\_files}, and
\SubmitCmd{dont\_encrypt\_output\_files} in the \Condor{submit}
manual page.
These commands have been available since Condor version 6.7.2,
but were never documented.
See descriptions starting at
~\ref{man-condor-submit-dont-encrypt-input-files}.
\Ticket{3174}


\end{itemize}


%%%%%%%%%%%%%%%%%%%%%%%%%%%%%%%%%%%%%%%%%%%%%%%%%%%%%%%%%%%%%%%%%%%%%%
\subsection*{\label{sec:New-7-8-2}Version 7.8.2}
%%%%%%%%%%%%%%%%%%%%%%%%%%%%%%%%%%%%%%%%%%%%%%%%%%%%%%%%%%%%%%%%%%%%%%

\noindent Release Notes:

\begin{itemize}

\item Condor version 7.8.2 released on August 14, 2012.

\item \Security Fixed a critical problem with DNS handling.

\end{itemize}

\noindent New Features:

\begin{itemize}

\item None.

\end{itemize}

\noindent Configuration Variable and ClassAd Attribute Additions and Changes:

\begin{itemize}

\item None.

\end{itemize}

\noindent Bugs Fixed:

\begin{itemize}

\item \Security Fixed a critical problem with DNS handling.

\end{itemize}

\noindent Known Bugs:

\begin{itemize}

\item None.

\end{itemize}

\noindent Additions and Changes to the Manual:

\begin{itemize}

\item None.

\end{itemize}

%%%%%%%%%%%%%%%%%%%%%%%%%%%%%%%%%%%%%%%%%%%%%%%%%%%%%%%%%%%%%%%%%%%%%%
\subsection*{\label{sec:New-7-8-1}Version 7.8.1}
%%%%%%%%%%%%%%%%%%%%%%%%%%%%%%%%%%%%%%%%%%%%%%%%%%%%%%%%%%%%%%%%%%%%%%

\noindent Release Notes:

\begin{itemize}

\item Condor version 7.8.1 released on June 15, 2012.

\end{itemize}


\noindent New Features:

\begin{itemize}

\item None.

\end{itemize}

\noindent Configuration Variable and ClassAd Attribute Additions and Changes:

\begin{itemize}

\item (Added in 7.8.0.) The new configuration variable
\Macro{ENABLE\_DEPRECATION\_WARNINGS} causes \Condor{submit} to issue
warnings when a job requests features that are no longer supported.
\Ticket{2968}

\item (Added in 7.7.6) The new configuration variable
\Macro{BATCH\_GAHP} should be used instead of \Macro{PBS\_GAHP},
\Macro{LSF\_GAHP} and \Macro{SGE\_GAHP}. These older configuration
variables are still recognized, but their use is now discouraged.
\Ticket{2670}

\item The default value for \Macro{GROUP\_SORT\_EXPR} was changed 
so that the \Expr{<none>} group would always negotiate last 
when using hierarchical group quotas.
Associated with that, 
the default value for \Macro{NEGOTIATOR\_ALLOW\_QUOTA\_OVERSUBSCRIPTION} 
was changed to \Expr{True}.  
These changes were made to make negotiation behave more like it did 
in the stable 7.4 series of Condor,
before hierarchical group quotas were added.
\Ticket{3040}

\end{itemize}

\noindent Bugs Fixed:

\begin{itemize}

\item Fixed a bug that caused events to not be written to the job event
log when the log is written in XML and a job policy expression triggering
the event contains any double quote marks.
\Ticket{3048}

\item Fixed a bug in the Condor init script that would cause
the init script to hang if Condor was not running.
\Ticket{2872}

\item Fixed a bug that caused parallel universe jobs using
Parallel Scheduling Groups 
(see section ~\ref{sec:Configure-Dedicated-Groups})
to occasionally stay idle even when
there were available machines to run them.
\Ticket{3017}

\item Fixed a bug that caused the \Condor{gridmanager} to crash when
attempting to submit jobs to a local PBS, LSF, or SGI cluster.
\Ticket{3014}

\item Fixed a bug in the handling of local universe jobs which caused
the \Condor{schedd} to log a spurious \Expr{ERROR} message
every time a local universe job exited, 
and then further caused the statistics for local universe jobs to be 
incorrectly computed.
\Ticket{3008}

\item Changed the internally used \Condor{ckpt\_probe} executable
to link statically, which should make the
checkpoint signature more resistant to non-significant changes in the system
configuration.
\Ticket{2901}

\item Restored Globus and VOMS support for the Mac OS X platform.
\Ticket{2991}

\item Fixed a bug when Condor runs under the PrivSep model,
in which if a job created a hard link from one file to another,
Condor was unable to transfer the files back to the submit side,
and the job was put on hold.
\Ticket{2987}

\item When configuration variables \MacroNI{MaxJobRetirementTime} or
\MacroNI{MachineMaxVacateTime} were very large, estimates of machine
draining badput and completion time were sometimes nonsensical
because of integer overflow.
\Ticket{3001}

\item Fixed a bug where per-job subdirectories and their contents
in \MacroUNI{SPOOL} would not be removed when the associated job
left the queue.
\Ticket{2942}

\item Fixed a bug that could cause the \Condor{schedd} to 
occasionally crash due to a race condition when running local universe jobs.
Associated with the bug would be the error message
\footnotesize
\begin{verbatim}
No local universe jobs were expected to be running, but one just exited!
\end{verbatim}
\normalsize
\Ticket{3009}

\end{itemize}

\noindent Known Bugs:

\begin{itemize}

\item None.

\end{itemize}

\noindent Additions and Changes to the Manual:

\begin{itemize}

\item Submit description file commands introduced in Condor version 7.7.1
have now been documented.
See the \Condor{submit} manual page at ~\ref{man-condor-submit} for
the newly added definitions of
\begin{description}
  \item[\SubmitCmd{ec2\_availability\_zone}]
  \item[\SubmitCmd{ec2\_ebs\_volumes}]
  \item[\SubmitCmd{ec2\_elastic\_ip}]
  \item[\SubmitCmd{ec2\_keypair\_file}]
  \item[\SubmitCmd{ec2\_vpc\_ip}]
  \item[\SubmitCmd{ec2\_vpc\_subnet}]
\end{description}

\item There is now a manual page for \Condor{router\_rm}, 
a script that provides additional features convenient for removing
jobs managed by the Condor Job Router.

\item Documentation not completed for the 7.7.6 release is now available.
The use of configuration variable \MacroNI{BATCH\_GAHP},
as well as the use of the new \SubmitCmd{grid\_resource} of
type \Expr{batch} for local submission of PBS, LSF, and SGE
jobs is documented.
See section ~\ref{sec:batch} for details.
\Ticket{2670}

\end{itemize}


%%%%%%%%%%%%%%%%%%%%%%%%%%%%%%%%%%%%%%%%%%%%%%%%%%%%%%%%%%%%%%%%%%%%%%
\subsection*{\label{sec:New-7-8-0}Version 7.8.0}
%%%%%%%%%%%%%%%%%%%%%%%%%%%%%%%%%%%%%%%%%%%%%%%%%%%%%%%%%%%%%%%%%%%%%%

\noindent Release Notes:

\begin{itemize}

\item Condor version 7.8.0 released on May 10, 2012.

\end{itemize}


\noindent New Features:

\begin{itemize}

\item (Added in 7.7.6.)  The new \Arg{-\_condor\_relocatable} argument
may be given as part of the invocation of a program that uses
standalone checkpointing.  This allows checkpointed programs to restart
without attempting to change to their original directory.
\Ticket{2877}

\item (Added in 7.7.5.) Added the \Arg{-absent} flag to \Condor{status},
which displays absent ClassAds.
\Ticket{2690}

\item (Added in 7.7.5.) Implement absent ads, which help track pool membership
in a persistent way.
\Ticket{2608} 

\end{itemize}

\noindent Configuration Variable and ClassAd Attribute Additions and Changes:

\begin{itemize}

\item The job ClassAd attribute \Attr{RemotePool} is now saved in
  \Attr{LastRemotePool} when the job finishes running.

\end{itemize}

\noindent Bugs Fixed:

\begin{itemize}

\item (Fixed in 7.7.6.) Fix \Arg{-absent}, \Arg{-vm}, and \Arg{-java}
flags to \Condor{status} so that they work with the \Arg{-long} option.
\Ticket{2943}

\item Support glob() on Scientific Linux 6 and others using the new
Linux system call fstatat(), but only when not using remote system calls.
\Ticket{2945}

\item Fixed potential startd crash introduced in v7.7.5 when claiming 
a partitionable slot that was in the Owner state. 
\Ticket{2936}

\item When ClassAd function stringListMember() is called with an empty
string as the second argument, it now evaluates to \Expr{False}.
Previously, it incorrectly evaluated to \Expr{Undefined}.
\Ticket{2953}

\item Format tags \%v and \%V for the \Opt{-format} option now properly
print all ClassAd value types. Previously, \Expr{True} and \Expr{False}
were printed as integers, and new ClassAd types like lists and nested
ClassAds could not be printed.
\Ticket{2960}

\item Restored RCS keyword strings CondorVersion and CondorPlatform to
the Condor binaries. These strings are found and printed by the 
\Opt{ident} program on Unix. They were missing in Condor versions 7.7.3
and later.
\Ticket{2932}

\item \Condor{job\_router} failed to route spooled source jobs.
\Ticket{2955}

\item Fixed a bug on Debian 6 and RHEL 6 that could cause standard
universe jobs to never checkpoint. This would happen if the job
triggered a call to NSCD (Name Service Caching Daemon) but NSCD 
wasn't running. 
Calls to NSCD can be triggered by a look up of a user account or
resolving a machine hostname to an IP address.
Now, NSCD is never consulted by a standard universe
job (this was already the behavior on other platforms).
\Ticket{2973}

\end{itemize}

\noindent Known Bugs:

\begin{itemize}

\item None.

\end{itemize}

\noindent Additions and Changes to the Manual:

\begin{itemize}

\item None.

\end{itemize}



% as of the 8.0.0 release, the 7-7 and 7-6 version histories no longer included.
%\input{version-history/gotchas.tex}
%\input{version-history/7-7.history.tex}
%\input{version-history/7-6.history.tex}
% as of April 2012, Karen no longer wants to include these older
% version histories with the 7.4 and 7.5 manuals.
%\input{version-history/7-5.history.tex}
%%%%      PLEASE RUN A SPELL CHECKER BEFORE COMMITTING YOUR CHANGES!
%%%      PLEASE RUN A SPELL CHECKER BEFORE COMMITTING YOUR CHANGES!
%%%      PLEASE RUN A SPELL CHECKER BEFORE COMMITTING YOUR CHANGES!
%%%      PLEASE RUN A SPELL CHECKER BEFORE COMMITTING YOUR CHANGES!
%%%      PLEASE RUN A SPELL CHECKER BEFORE COMMITTING YOUR CHANGES!

%%%%%%%%%%%%%%%%%%%%%%%%%%%%%%%%%%%%%%%%%%%%%%%%%%%%%%%%%%%%%%%%%%%%%%
\section{\label{sec:History-7-4}Stable Release Series 7.4}
%%%%%%%%%%%%%%%%%%%%%%%%%%%%%%%%%%%%%%%%%%%%%%%%%%%%%%%%%%%%%%%%%%%%%%

This is a stable release series of Condor.
As usual, only bug fixes (and potentially, ports to new platforms)
will be provided in future 7.4.x releases.
New features will be added in the 7.5.x development series.

The details of each version are described below.

%%%%%%%%%%%%%%%%%%%%%%%%%%%%%%%%%%%%%%%%%%%%%%%%%%%%%%%%%%%%%%%%%%%%%%
\subsection*{\label{sec:New-7-4-5}Version 7.4.5}
%%%%%%%%%%%%%%%%%%%%%%%%%%%%%%%%%%%%%%%%%%%%%%%%%%%%%%%%%%%%%%%%%%%%%%

\noindent Release Notes:

\begin{itemize}

\item Condor version 7.4.5 not yet released.
%\item Condor version 7.4.5 released on Month Date, 2010.

\end{itemize}


\noindent New Features:

\begin{itemize}

% gittrac #1713
\item \Condor{dagman} now prints a message in the \File{dagman.out} file
whenever it truncates a node job user log file.

% gittrac #1715
\item \Condor{dagman} now prints additional diagnostic information in the
case of certain log file errors.

\end{itemize}

\noindent Configuration Variable and ClassAd Attribute Additions and Changes:

\begin{itemize}

\item None.

\end{itemize}

\noindent Bugs Fixed:

\begin{itemize}

% gittrac #1750
\item A network disconnect between the submit machine and execute
  machine during the transfer of output files caused the
  \Condor{starter} daemon to immediately give up, rather than waiting
  for the \Condor{shadow} to reconnect.  This problem was introduced
  in Condor version 7.4.4.

% gittrac #1743
\item If \Condor{ssh\_to\_job} attempted to connect to a job while the
  job's input files were being transferred, this caused the file
  transfer to fail, which resulted in the job returning to the idle
  state in the queue.

% gittrac #1785
\item In privsep mode, the transfer of output failed if a job's execute
  directory contained symbolic links to non-existent paths.

\end{itemize}

\noindent Known Bugs:

\begin{itemize}

\item None.

\end{itemize}

\noindent Additions and Changes to the Manual:

\begin{itemize}

\item None.

\end{itemize}


%%%%%%%%%%%%%%%%%%%%%%%%%%%%%%%%%%%%%%%%%%%%%%%%%%%%%%%%%%%%%%%%%%%%%%
\subsection*{\label{sec:New-7-4-4}Version 7.4.4}
%%%%%%%%%%%%%%%%%%%%%%%%%%%%%%%%%%%%%%%%%%%%%%%%%%%%%%%%%%%%%%%%%%%%%%

\noindent Release Notes:

\begin{itemize}

\item Condor version 7.4.4 released on October 18, 2010.

% gittrac #1508
\item \Security 
This release fixes a bug in which Amazon EC2 jobs
(jobs with \SubmitCmd{universe = grid} and \SubmitCmd{grid\_resource = amazon})
that use the \SubmitCmd{amazon\_keypair\_file}
command may expose the private SSH key to other users.
The created file had insecure permissions,
allowing other users on the submit host to read the file.
Any other user who could see the file could learn about these EC2 jobs
using \Condor{q}, 
and the other user could then connect to them with the private SSH key.

To work around the bug without installing this release,
do one or both of the following:
\begin{itemize}
\item Do not use the submit description file command
\SubmitCmd{amazon\_keypair\_file}.
\item Ensure that the directory holding the private SSH key 
has suitably restrictive permissions,
such that other users cannot read files inside the directory.
\end{itemize}


% gittrac #1524
\item Condor can now be built on Mac OS X 10.6.

% gittrac #1696
\item The \Condor{master} shutdown program, which is configured via 
  the \Macro{MASTER\_SHUTDOWN\_$<$Name$>$} configuration variable,
  is now run with root (Unix) or administrator (Windows) privileges.
  The adminstrator must ensure
  that this cannot be used in such a way as to violate system integrity.

\end{itemize}


\noindent New Features:

\begin{itemize}

\item \SubmitCmd{load\_profile} is now supported by the Unix version of
\Condor{submit} when submitting jobs to Windows.  Previously, this command
was only supported by the Windows version of \Condor{submit}.

\item Added an example Mac OS X launchd configuration file for starting Condor.

\end{itemize}

\noindent Configuration Variable and ClassAd Attribute Additions and Changes:

\begin{itemize}

\item None.

\end{itemize}

\noindent Bugs Fixed:

\begin{itemize}

% gittrac 1434
\item Fixed bad behavior in \Condor{quill} where, under certain error
conditions, many copies of the \File{schedd\_sql.log} file would be
inserted into the database, filling up the disk volume used by the
database. As a consequence of this bug fix, the \verb@LogBody@ column
for each row in the \verb@Error_SqlLogs@ table will be empty. Please
consult the \Condor{quill} daemon log file for the error instead.

% gittrac 1654
\item Fixed a bug with how the \SubmitCmd{standard} universe 
remote system call \Syscall{getrlimit} functioned.
Under certain conditions with
32-bit and 64-bit \SubmitCmd{standard} universe binaries,
\Syscall{getrlimit} would erroneously report 2147483647 bytes as a limit,
when \Expr{RLIM\_INFINITY} should have been the correct response.

% gittrac 1631
\item Fixed a misleading error message issued by \Condor{run},
which stated
\begin{verbatim}
The DAGMan job was aborted by the user.
\end{verbatim}
when the job submitted by \Condor{run} was aborted by the user.
It now states 
\begin{verbatim}
The job was aborted by the user.
\end{verbatim}

% gittrac 1543
\item When the \Condor{startd} daemon is running with an execute directory on
a very large file system, with more than 32 bits worth of free blocks
on a 32-bit system, it would incorrectly report 0 free bytes.  This
has been fixed.

\item For spooled jobs, input files were sometimes transferred twice from
the submit machine to the execute machine.  This happened if the input files
were specified without any path information,
as with a file name with no directory specified.
This problem has existed since before Condor version 7.0.0.

% gittrac 457
\item The configuration variable \MacroNI{NETWORK\_INTERFACE} did not
work in some situations, because of Condor's attempts to
automatically rewrite published addresses to match the IP address of
the network interface used to make the publication.

% gittrac 961
\item Fixed a bug in which the default unit of configuration variable
\MacroNI{STARTD\_CRON\_TEST\_PERIOD}
should have been seconds, but instead was \Expr{Undefined}.

% gittrac 1485
\item Fixed a bug in which \Condor{submit} checked for bad \Condor{schedd} cron 
 arguments incorrectly within a submit description file.
 Now \Condor{submit} will detect the problem and print out an error message.

% gittrac 1565
\item With some versions of \Prog{ssh}, \Condor{ssh\_to\_job} failed if
the \Env{SHELL} environment variable was set to \Prog{/bin/csh}.

% gittrac 1567
\item Submission of \SubmitCmd{vm} universe jobs via Globus was not possible,
because the Globus Condor jobmanager explicitly set the input, output,
and error files to \File{/dev/null},
and \Condor{submit} refused any setting of these files for
\SubmitCmd{vm} universe jobs.  
Now, \File{/dev/null} is an allowed setting for the input, output,
and error files for \SubmitCmd{vm} universe jobs.

% gittrac #1564
\item Fixed a bug that caused a \SubmitCmd{vm} universe job's output files
to be incorrectly transferred back to the submit machine, 
when the submit description file command \SubmitCmd{vm\_no\_output\_vm}
was set to \Expr{false},
indicating that no files should be transferred.

% gittrac #416
\item String literals within \verb@$$([])@ expressions within a submit
description file failed to be evaluated and resulted in the job going on hold.
This problem has existed since before Condor 7.0.0.

% gittrac #106
\item \Condor{preen} was not able to clean up files in the \MacroNI{EXECUTE}
directory when in privsep mode.

% gittrac #1589
\item A problem was fixed that could cause a Condor daemon that
  connects to other daemons via CCB to permanently run out of space
  for more registered sockets until restarted.  This problem appeared
  in the logs as the following message:

\begin{verbatim}
file descriptor safety level exceeded
\end{verbatim}

% gittrac #1596
\item Fixed a problem that could cause the \Condor{collector} to crash
when receiving updated matchmaking information for offline ClassAds that do
not exist.

% gittrac #1518
\item \Condor{ssh\_to\_job} did not work when
\MacroNI{SEC\_DEFAULT\_NEGOTIATION} was set to \MacroNI{OPTIONAL}.

% gittrac #1611 #1612
\item The \SubmitCmd{vm} universe now works properly on machines that 
have Condor's Privilege Separation mechanism enabled.

% gittrac #1624
\item \Condor{submit} no longer pads a \SubmitCmd{vm} universe job's disk usage
estimation by 100MB.

% gittrac #1553
\item Fixed a bug with the \Macro{vm\_cdrom\_files} submit file
command, that caused VMware \SubmitCmd{vm} universe jobs to fail if the virtual
machine already had a CD-ROM image associated with it.

% gittrac #1465
\item Configuration variables \Macro{SOAP\_SSL\_CA\_DIR} and
\Macro{SOAP\_SSL\_CA\_FILE} are now properly used when authenticating
with Amazon EC2 servers.

% gittrac #1484
\item Fix a bug with the \Macro{<subsys>\_LOCK} configuration variable.
It could let daemons writing to the same daemon log overwrite each other's
entries and cause daemons to exit when the log is rotated.

% gittrac #1557
\item Fixed a bug that caused nordugrid jobs to fail if the
\SubmitCmd{grid\_resource} attribute included a port as part of the server
host name.

% gittrac #1672
\item Fixed a confusing error message mentioning
  \verb@LocalUserLog::logStarterError()@ when the \Condor{starter} fails to
  communicate with the \Condor{shadow} before the job has started.

% gittrac #1602
\item Fixed the event log and shadow log for standard universe jobs to 
identify the checkpoint server on which a job might have failed to store 
its checkpoint or from which it might have failed to restore its checkpoint.

\item Fixed a bug in the \Condor{gridmanager} that could cause it to crash
while handling grid-type cream jobs.

% gittrac #1699
% gittrac #1700
\item Improved the \Condor{gridmanager}'s handling of grid-type cream jobs
that are held or removed by the user. Canceling the cream job is much less
likely to fail and jobs can no longer get stuck in the cream state of
CANCELED.

% gittrac #1701
\item Fixed the web server feature controlled by \Macro{ENABLE\_WEB\_SERVER}.
Previously, all HTTP GET requests would fail on non-linux Unix machines.

\end{itemize}

\noindent Known Bugs:

\begin{itemize}

\item None.

\end{itemize}

\noindent Additions and Changes to the Manual:

\begin{itemize}

\item The Windows platform installation instructions have been updated.

\item Section~\ref{sec:file-transfer} on Condor's File Transfer Mechanism
has been revised and updated.

\item Section~\ref{classad-query-examples}, providing examples of utilizing
ClassAd expressions within the \Opt{-constraint} option of \Condor{q}
or \Condor{status} commands has been expanded to clarify both
Unix and Windows platform specifics.

\end{itemize}


%%%%%%%%%%%%%%%%%%%%%%%%%%%%%%%%%%%%%%%%%%%%%%%%%%%%%%%%%%%%%%%%%%%%%%
\subsection*{\label{sec:New-7-4-3}Version 7.4.3}
%%%%%%%%%%%%%%%%%%%%%%%%%%%%%%%%%%%%%%%%%%%%%%%%%%%%%%%%%%%%%%%%%%%%%%

\noindent Release Notes:

\begin{itemize}

\item Condor version 7.4.3 released on August 16, 2010.

\end{itemize}


\noindent New Features:

\begin{itemize}

\item None.

\end{itemize}

\noindent Configuration Variable and ClassAd Attribute Additions and Changes:

\begin{itemize}

\item The new configuration variable \Macro{ENABLE\_CHIRP} 
defaults to \Expr{True}. 
An administrator may set it to \Expr{False}, which 
disables Chirp remote file access from execute machines.

\item The new configuration variable
  \Macro{ENABLE\_ADDRESS\_REWRITING} defaults to \Expr{True}.  It may
  be set to \Expr{False} to disable Condor's dynamic algorithm for choosing
  which IP address to publish in multi-homed environments.  The dynamic
  algorithm chooses the IP address associated with the network interface
  used to make the publication, for example, the network interface used 
  to communicate with the \Condor{collector}.

% gittrac 1407
\item Configuration variable \Macro{VM\_BRIDGE\_SCRIPT} has been removed
  and is no longer valid.

% gittrac 1402 and 1407
\item The new configuration variable
  \Macro{VM\_NETWORKING\_BRIDGE\_INTERFACE} specifies the networking interface
  that Xen or KVM \SubmitCmd{vm} universe jobs will use.
  See section~\ref{param:VMNetworkingBridgeInterface} for documentation.

% gittrac #1333
\item
Allowed the configuration file entries \MacroNI{GSI\_DAEMON\_TRUSTED\_CA\_DIR}
and \MacroNI{GSI\_DAEMON\_DIRECTORY} to be set with environment variables,
as the rest of Condor configuration variables can be.

\end{itemize}

\noindent Bugs Fixed:

\begin{itemize}

\item
When using file transfer semantics,
if the job exited in such a manner so as to not produce all
output files specified in \SubmitCmd{transfer\_output\_files},
then which files were transferred was potentially incorrect.
Now, all existing files are transferred back,
and the files which are not able to be transferred back due to non-existence
appear as zero length files.
An example of when this occurred would be the job dumping core
and then being placed on hold.

% gittrac 1185
\item
Fetch work hooks to prepare are now invoked as the \Login{condor} user,
instead of as the job user.

\item
Improved the file extension detection on Windows platforms.

\item
\Condor{wait} could occasionally get stuck in an infinite loop,
if it missed the execution event of the job it is waiting for.
This is now fixed.

% gittrac 1413
\item
Fixed a bug within the \Condor{startd} cron capabilities,
that only occurred on Windows platforms.
\Condor{startd} cron scripts failed to run if an initial directory was left
unspecified.

% gittrac 1012
\item
Fixed a bug in which a job would be incorrectly placed on hold, with 
a confusing error message that appeared similar to
\footnotesize
\begin{verbatim}
Condor failed to start your job 9090.-1 because job attribute Args contains $$(VirtualMachineID).
\end{verbatim}
\normalsize
This occurred if the submit command \SubmitCmd{copy\_to\_spool} 
was \Expr{true},
the submit description file for the job contained \$\$ macros,
and \Condor{preen} ran after the job was submitted and before it started.

% gittrac 1427
\item
Added the jobs\_vertical\_history table to the list of tables that
\Condor{quill} periodically re-indexes.

\item
Fixed bug in \Condor{preen} in which it would delete \Condor{startd} daemon
history files.

% gittrac 487
\item
  Fixed a bug where if a user job using file transfer with
  \SubmitCmd{transfer\_output\_files} and \SubmitCmd{when\_to\_transfer\_output}
  is set to \Expr{ON\_EXIT\_OR\_EVICT} fails
  to produce all of the specified files and exit, as when core
  dumping due to a fault, then the stdout, stderr, and core file of the
  job were not transferred back to the submitting machine.

\item
  Fixed numerous, small, rare memory leaks.

\item 
  Fixed a bug in which processor affinity settings were ignored if
  privilege separation was enabled.

% gittrac 1329
\item Network connections for Condor file transfers were ignoring
  private network settings.  The connection from the execute node to
  the submit node always attempted to use the public network address
  of the submit machine.

% gittrac 1405
\item The configuration variable \MacroNI{TCP\_FORWARDING\_HOST} did not work
in some situations
because of Condor's attempts to automatically rewrite published addresses to
match the IP address of the network interface used to make the publication.

% gittrac 1346
\item A single job could match multiple offline slots in a single
negotiation cycle.  This problem could cause \Condor{rooster} to
wake up too many offline machines for the number of jobs available
to run on them.  The fix for this problem requires updating both
the \Condor{negotiator} and the \Condor{schedd}.

% gittrac 1349
\item Fixed a problem that caused the \Condor{startd} daemon to
crash in some cases when \MacroNI{STARTD\_SENDS\_ALIVES} was \Expr{True}.
This setting is \Expr{False} by default.

% gittrac #1337
\item Fixed a problem where the \Condor{kbdd} has a chance of
entering an infinite loop on platforms that use X-Windows.
Microsoft Windows and Mac OS X platforms were not affected.  This bug is
present in all earlier 7.4.x Condor releases.

\item The default path to \Prog{sftp-server} has been improved,
 so that \Condor{ssh\_to\_job} can use \Prog{sftp} out-of-the-box on 
RedHat Enterprise Linux 5 platforms.

% gittrac #1383
\item A \Prog{nordugrid\_gahp} binary built on RedHat Enterprise Linux 3
no longer crashes
when run on a RedHat Enterprise Linux 4 or Scientific Linux 4 machine.

% gittrac #1418
\item Fixed a bug in \Condor{rm} that caused it to misinterpret user names
that begin with a digit, such as \Expr{4abc}.
It incorrectly used them as cluster numbers. 

% gittrac #1423
\item Fixed a bug that caused the \Condor{startd} to invoke the
  ``power\_state'' plug-in as the condor user.  This caused
  hibernation to fail because power\_state requires root privileges
  to function properly.

% gittrac #1330
\item Fixed a bug that could cause the \Condor{schedd} to crash if there
were any idle scheduler universe jobs when files were staged into the
\Condor{schedd} for a new job.

% gittrac #1404
\item Fixed a bug in the \Prog{nordugrid\_gahp} that could cause it to exit
when connecting to a misconfigured LDAP server.

% gittrac #1352
\item Fixed a bug that prevented the log file defined with the configuration
variable \Macro{NEGOTIATOR\_MATCH\_LOG} from rotating.
See section~\ref{param:SubsysLevelLog} for the definition of this variable. 

% gittrac #1413
\item Fixed a bug that caused \Prog{startd\_cron} jobs to fail on Windows. 
This bug is present in all earlier 7.4.x Condor releases.

% gittrac #1551
\item The submit description file command \SubmitCmd{vm\_cdrom\_files}
now works properly with Windows execute machines. 
Previously, creation of the ISO file would fail, 
causing job execution to be aborted.

% gittrac #1423
\item Fixed a bug that caused the \Condor{startd} to invoke the
  \Prog{power\_state} plug-in as the condor user.
  This caused hibernation to fail, 
  because \Prog{power\_state} requires root privileges to function properly.

\end{itemize}

\noindent Known Bugs:

\begin{itemize}

\item None.

\end{itemize}

\noindent Additions and Changes to the Manual:

\begin{itemize}

\item Searching the PDF version of the manual for items containing 
underscore characters, such as many configuration variable names,
now works correctly.

% gittrac #1340
\item The new subsection~\ref{ClassAd:examples} provides examples of
evaluation results when using the operators \Expr{==}, \Expr{=?=},
\Expr{!=}, and \Expr{=!=}.

\item Section~\ref{sec:vmuniverse} with specifics on \SubmitCmd{vm}
universe jobs has been updated to contain more details about
both checkpoints and \SubmitCmd{vm} universe jobs in general.

\end{itemize}


%%%%%%%%%%%%%%%%%%%%%%%%%%%%%%%%%%%%%%%%%%%%%%%%%%%%%%%%%%%%%%%%%%%%%%
\subsection*{\label{sec:New-7-4-2}Version 7.4.2}
%%%%%%%%%%%%%%%%%%%%%%%%%%%%%%%%%%%%%%%%%%%%%%%%%%%%%%%%%%%%%%%%%%%%%%

\noindent Release Notes:

\begin{itemize}

\item Condor version 7.4.2 released on April 6, 2010.

\end{itemize}


\noindent New Features:

\begin{itemize}

\item None.

\end{itemize}

\noindent Configuration Variable and ClassAd Attribute Additions and Changes:

\begin{itemize}

% gittrac #1001
\item When \MacroNI{WANT\_SUSPEND} is defined and evaluates to anything
other than the value \Expr{True},
it is now utilized as if it were \Expr{False}.
If \MacroNI{WANT\_SUSPEND} is not explicitly defined,
the \Condor{startd} exits with an error message.
Previously, if \Expr{Undefined}, it was treated as an error,
which caused the \Condor{startd} to exit with an error message.

\end{itemize}

\noindent Bugs Fixed:

\begin{itemize}

% gittrac 1217
\item Fixed a bug in which the \Condor{schedd} would sometimes negotiate
  for and try to run
  more jobs than specified by \MacroNI{MAX\_RUNNING\_JOBS}.  Once the
  jobs started running, it would then kill them off to get back below
  the limit.  This was more likely to happen with slow preemption
  caused by \MacroNI{MaxJobRetirementTime} or by a large timeout
  imposed by \MacroNI{KILL}.  This problem has existed since before
  Condor 6.5.  When this problem happened, the following message
  appeared in the \Condor{schedd} log:

\begin{verbatim}
Preempting X jobs due to MAX_JOBS_RUNNING change
\end{verbatim}

% gittrac 1250
\item Fixed a problem that caused \Condor{ssh\_to\_job} to fail to connect
to a job running on a slot with multiple '@' signs in its name.  This bug
has existed since the introduction of \Condor{ssh\_to\_job} in 7.3.2.

% gittrac 116
\item In all previous versions of Condor, \Condor{status} refused to
  accept \Opt{-long}, \Opt{-xml}, and \Opt{-format} when followed by
  an argument such as \Opt{-master} that specified which type of
  daemon to look at.  The order of the arguments had to be reversed or
  it would produce a message such as the following:

\begin{verbatim}
Error:  arg 4 (-master) contradicts arg 1 (-format)
\end{verbatim}

% gittrac #1201
\item Fixed a bug which caused the \Condor{master} to crash if
\MacroNI{VIEW\_SERVER} was included in \MacroNI{DAEMON\_LIST} and
\MacroNI{CONDOR\_VIEW\_HOST} was unset.

% gittrac #1196
\item Fixed a bug that caused configuration parameter
\MacroNI{LOCAL\_CONFIG\_DIR} to be ignored if it was set in a local
configuration file, as opposed to the top-level configuration file.

% gittrac #1202
\item Fixed a bug that could cause the \Condor{schedd} to behave
incorrectly when reading an invalid job queue log on startup.

% gittrac #1215
\item Fixed a bug that could corrupt the job queue log
if the \Condor{schedd} daemon's attempt to compact it fails.

% gittrac #1256
\item Fixed a problem that in rare cases caused the \Condor{schedd} to
crash shortly after the \Condor{gridmanager} exited.
This bug has existed since before Condor version 6.8.

% gittrac #1270
\item Fixed a problem that was resulting in messages such as the following:

\footnotesize
\begin{verbatim}
ERROR: receiving new UDP message but found a long message still waiting
to be closed (consumed=0). Closing it now.
\end{verbatim}
\normalsize

\item The file extension specified to \Condor{fetch\_log} can no longer
contain a path delimiter.

% gittrac 1299
\item When in graceful shutdown mode, the \Condor{schedd} was
  sometimes starting idle scheduler universe jobs.  With a large
  enough number of scheduler universe jobs, this could lead to a cycle
  of stopping and restarting jobs until the graceful shutdown time
  expired.

% gittrac #1259
\item Fixed multiple bugs that prevented Condor from building on or
  running correctly on OpenSolaris X86/64 version 2009.06.

% gittrac #1238
\item Fixed a bug which caused the \Condor{startd} to incorrectly
  count the number of processors on some machines with
  Hyper-threading enabled.  This bug was introduced in
  Condor version 7.3.2, and exists in 7.4.0 and 7.4.1.

% gittrac #1167
\item Fixed a problem with GSI authentication in Condor that would cause
daemons to consume more and more memory over time.  The biggest source
of trouble was introduced in Condor version 7.3.2.
However, a smaller memory leak that
existed in all previous versions of Condor has also been fixed.

% gitrack #553
\item Fixed a bug where if \Condor{compile} is invoked in a manner such as:
\begin{verbatim}
  condor_compile gcc -print-prog-name=ld 
\end{verbatim}
an error would be emitted,
and \Condor{compile} would exit with a bad exit code.

% gittrac #1093
\item The sort based on \Condor{status} output accidentally changed in 
Condor version 7.3,
so that the output was based on the slot name first, then machine name.
The behavior is now restored to the original sorting: first on machine name,
then slot name.

% gittrac #728
\item If one machine running a parallel job crashed,
and job leases are enabled (which they are by default),
the job would not exit until the job lease duration expired.
As the \Condor{starter} will not get respawned,
there is no need to wait.
Many sites set long job lease durations,
to prevent jobs from being killed when the machine running
the \Condor{schedd} daemon reboots.
Now, if one node goes away, the whole computation is shut down immediately.

\item Fixed the verbosity level of some \Condor{dagman} messages written to
the \File{dagman.out} file.

% gittrac #1137
\item Fixed a bug introduced in Condor version 7.3.2 that resulted in
  messages such as the following even in cases where no problem in
  communicating with the \Condor{collector} had been encountered:

\begin{verbatim}
Collector <X> is still being avoided if an alternative succeeds.
\end{verbatim}

This problem was believed to be fixed in Condor 7.4.1, but some cases
of the problem remained in that version.

% gittrac 1160
\item Fixed a bug from Condor version 6.1.14,
that resulted in the \Condor{schedd} performing
the operation scheduled via \MacroNI{WALL\_CLOCK\_CKPT\_INTERVAL} at the
specified frequency (default time of 1 hour),
multiplied by the number of times the
\Condor{schedd} daemon had been reconfigured during its lifetime.
This could lead to degraded performance,
especially prior to Condor version 7.4.1,
when this operation was more disk-intensive.

% gittrac 1184
% gittrac 1181
\item 32-bit Linux versions of Condor running in a 64-bit environment would
sometimes not detect the existence of some processes and sometimes
wrongly detect that a tracked process belonged to root when it
actually belonged to some other user.  This could lead to failure to run
jobs or failure to properly monitor and clean up after them.  When the wrong
process ownership problem happened,
the following message appeared in the \Condor{master} and/or \Condor{procd}
logs:

\begin{verbatim}
ProcAPI: fstat failed in /proc! (errno=75)
\end{verbatim}

If \Condor{procd} failed to detect the existence of its own parent process,
it would exit with the following message in its log:

\begin{verbatim}
ERROR: master has exited
\end{verbatim}

% gittrac 1186
\item Fixed a problem in the \Condor{job\_router} daemon,
  introduced in Condor version 7.2.2,
  that could cause the daemon to crash when failing to carry out the change
  of state dictated by a job's periodic policy expressions,
  for example, the failure to put a job on hold when \AdAttr{periodic\_hold}
  becomes \Expr{True}.

% gittrac #1209
\item Fixed a bug introduced in Condor 7.3.2 that caused Grid Monitor
jobs to receive a full X.509 proxy. Now, it always receives a limited
proxy, which was the previous behavior.

% gittrac #1070
\item Fixed a bug that could cause the nordugrid\_gahp to crash.

% gittrac #742
\item Fixed a problem introduced in 7.4.0 that could cause two 
  \Condor{schedd} daemons
  with a match to the same slot to both fail to claim it, rather than
  letting the first one to claim it succeed.  This sort of situation
  can happen when the \Condor{negotiator} has a stale view of the pool,
  either because the gap between negotiation cycles is configured to
  be shorter than usual, or because updates from the \Condor{startd}
  to the \Condor{collector}
  are not reliably delivered and processed.

% gittrac #1251
\item The \Condor{kbdd} is no longer ignored by the \Condor{startd}
when the configuration variable \Macro{CONSOLE\_DEVICES} is defined.

% gittrac #92
\item When using the \Opt{-d} command line argument with a daemon,
the values of \MacroNI{LOG}, \MacroNI{SPOOL}, and \MacroNI{EXECUTE}
no longer change every time a \Condor{reconfig} command is received.

\end{itemize}

\noindent Known Bugs:

\begin{itemize}

% gittrac #1337
\item The \Condor{kbdd} has a chance of entering an infinite loop
on platforms that use X-Windows.  Microsoft Windows and Mac OS X
are not affected.  Removing KBDD from \MacroNI{DAEMON\_LIST} is a
workaround, although this impairs Condor's ability to detect
console usage.  This bug is fixed in Condor version 7.4.3.

\end{itemize}

\noindent Additions and Changes to the Manual:

\begin{itemize}

\item Descriptions of all the commands that may be placed into a
submit description file are now located within the \Condor{submit}
manual page, instead of within Chapter 2, the Users' Manual.

\item An initial, but not yet complete set of configuration variables
that require a restart when changed,
is listed in section~\ref{sec:Macros-Requiring-Restart}.
Using \Condor{reconfig} to change these variables' values is not sufficient.

\end{itemize}


%%%%%%%%%%%%%%%%%%%%%%%%%%%%%%%%%%%%%%%%%%%%%%%%%%%%%%%%%%%%%%%%%%%%%%
\subsection*{\label{sec:New-7-4-1}Version 7.4.1}
%%%%%%%%%%%%%%%%%%%%%%%%%%%%%%%%%%%%%%%%%%%%%%%%%%%%%%%%%%%%%%%%%%%%%%

\noindent Release Notes:

\begin{itemize}

% gittrac #1018
\item \Security A flaw was found that could allow a user who already is authorized to
submit jobs into Condor, to queue a job under the guise of  a different
user.  In this way, someone who has access to a Condor submission
service and is allowed to submit jobs into Condor could gain access to
another non-root or non-administrator account on the system.
This flaw was discovered during the development process; no incidents
have been reported.  Details of the problem will be made available on Feb 1st,
2010.

% gittrac #918
\item The default value of \MacroNI{JOB\_ROUTER\_NAME} has changed
  from an empty string to \verb|jobrouter| in order to address
  problems caused by the previous default.  Without special handling,
  this means that jobs being managed by \Condor{job\_router} before
  upgrading will not be adopted by the new version of
  \Condor{job\_router} if the default \MacroNI{JOB\_ROUTER\_NAME} was
  being used.  To correct this, follow the instructions given in the
  description of \MacroNI{JOB\_ROUTER\_NAME} on
  page~\pageref{JobRouterName}.

\end{itemize}


\noindent New Features:

\begin{itemize}

% gittrac #921, #999
\item Condor allows submit files to specify an \SubmitCmd{IwdFlushNFSCache}
expression,
to control whether or not Condor tries to flush the NFS cache for 
a job's initial working directory on job completion.

% gittrac #929, #943
\item The new \Opt{-attributes} option to \Condor{status}
  explicitly specifies the attributes to be listed when using the
  \Opt{-xml} or \Opt{-long} options.

\end{itemize}

\noindent Configuration Variable and ClassAd Attribute Additions and Changes:

\begin{itemize}

% gittrac #161, #935, #936
\item New VOMS attributes have been introduced into the job ad to keep them
separate from the X509UserProxySubjectName.

\item The default for \MacroNI{JOB\_ROUTER\_NAME} has changed from an
  empty string to \verb|jobrouter|.  See the release notes for more
  information about upgrading from an old version.

\item The configuration variable \Macro{TCP\_FORWARDING\_HOST}
  has existed in Condor since version 7.0.0, but was not documented.
  See section~\ref{param:TcpForwardingHost} for documentation.

% gittrac #933
\item The new configuration variable \MacroNI{STARTD\_PER\_JOB\_HISTORY\_DIR}
allows ClassAds of completed jobs to be stored in a directory separate 
from the existing one specified with \MacroNI{PER\_JOB\_HISTORY\_DIR}.

\end{itemize}

\noindent Bugs Fixed:

\begin{itemize}

% gittrac #749
\item  Condor no longer creates the job sandbox in its \MacroNI{SPOOL}
directory if it is not needed.

% gittrac #1019
\item Fixed a problem introduced in Condor version 7.4.0 that caused GSI
authentication between Condor processes to fail with using a
non-legacy format X.509 proxy.

% gittrac #1028
\item Fixed a problem with CCB under Windows platforms that has existed since
Condor version 7.3.0.  
This problem caused CCB-enabled daemons to become unresponsive
after the exit of a child process.

% gittrac #931 -- Fixed minor spelling errors, not worthy of listing.

% gittrac #923
\item Improved the handling of previously-submitted gt2 grid jobs upon
release from hold, when there is no Globus job manager for the job running
on the remote resource.

% gittrac #453
\item Fixed a problem with job leases for jobs that use a \Condor{shadow}.
Previously, while these jobs were running, lease renewals from the 
submitter would not be
noticed, and the job would be aborted when the original lease expired.

% gittrac #870
\item Fixed a bug that only allowed approximately 50 splices to be included into
a DAG input file. There is now no limit to the number of splices
one may include into a DAG input file except, of course, for the
implicit memory allocation limit of the \Condor{dagman} process.

% gittrac #909
\item Removed attempted limiting of swap space via \Prog{ulimit -v} using the
\Attr{VirtualMemory} machine ClassAd attribute in the script
\File{condor\_limits\_wrapper.sh}.

% gittrac #899
\item Fixed a bug that caused \MacroNI{ALLOW\_CONFIG} and
  \MacroNI{HOSTALLOW\_CONFIG}, as well as the corresponding
  \MacroNI{DENY} configuration variables to incorrectly handle a
  setting consisting of a single \Expr{*} or the equivalent \Expr{*/*}.  This
  also fixes a bug that caused incorrect merging of \MacroNI{ALLOW}
  and \MacroNI{HOSTALLOW} settings when one, but not both, consisted of
  a single \Expr{*} or the equivalent \Expr{*/*}.
  These bugs have existed since before Condor version 6.8.

% gittrac #905
\item Fixed a bug introduced in Condor version 7.3.0 that could cause 
Condor daemons to crash when reading malformed network addresses.

% gittrac #883
\item Removed a check for root ownership of a script specified by
the configuration variable \MacroNI{VM\_SCRIPT}.

% gittrac #884
\item Fixed a bug in writing the header of the file identified by
the configuration variable \MacroNI{EVENT\_LOG}.

% gittrac #891
\item Fixed a bug that could cause the \Condor{startd} to segfault on shutdown
when using dynamic slots.

% gittrac #871
\item Fixed a problem introduced in Condor version 7.3.2 that changed 
  the behavior of
  an undocumented method for selecting attributes to be displayed in
  \Condor{q} \Opt{-xml}.  Prior to this bug, the following command
  would produce XML output with the attributes \Attr{A} and \Attr{B},
  plus a few other attributes that were always shown.

\begin{verbatim}
condor_q -xml -format "%s" A+B
\end{verbatim}

In Condor versions 7.3.2 and 7.4.0,
this same command produced an empty XML ClassAd.
The workaround was to use multiple \Opt{-format} options, each listing
just one desired attribute, rather than a single one with an
expression of all desired attributes.  Although this is now fixed, the
more straightforward way to select attributes since Condor version 7.3.2
is to use the \Opt{-attributes} option.

% gittrac #907
\item Fixed a bug introduced in Condor version 7.3.2 that resulted in 
  messages such
  as the following even in cases where no problem in communicating
  with the \Condor{collector} had been encountered:

\begin{verbatim}
Collector <X> is still being avoided if an alternative succeeds.
\end{verbatim}

% gittrac #859
\item Fixed a bug that has been in the \Condor{startd} since before
  Condor version 6.8.  If the \Condor{startd} ever failed to send signals to the
  \Condor{starter} process, it could fail to properly clean up the
  machine ClassAd, leaving attributes from
  \MacroNI{STARTD\_JOB\_EXPRS} in the ClassAd but not making them visible
  in \Condor{status} queries.  One possible problem resulting from
  this could be matches being made by the \Condor{negotiator} that are then
  rejected by the \Condor{startd}.  Repeated messages such as the following
  would then result in the \Condor{startd} log:

\begin{verbatim}
slot1: Request to claim resource refused.
\end{verbatim}

%gittrac #908
\item Fixed a problem that resulted in the following message in the
  \Condor{startd} log:

\begin{verbatim}
Timer -1 not found
\end{verbatim}

%gittrac #937
\item Fixed a problem in which security sessions were not cached
  correctly when using CCB.  This resulted in re-authentication in
  some cases where a cached security session could have been used.

% gittrac #161, #935, #936, #1020
\item Fixed multiple problems with the handling of VOMS attributes in GSI
proxies.

% gittrac #934
\item Fixed a bug that caused \Condor{dagman} to hang when running a
DAG with POST scripts, if the global event log is turned on.

% gittrac #973
\item Improved how the private network address is published when using
  the configuration variables \MacroNI{PRIVATE\_NETWORK\_NAME} and
  \MacroNI{PRIVATE\_NETWORK\_INTERFACE}.  In some cases, this
  information was not being used, and therefore connections were made
  to the public address when they could have been made to the private
  address.

% gittrac #801
\item Fixed a bug exhibited under Windows XP,
where using \MacroNI{USE\_VISIBLE\_DESKTOP}
would cause strange behavior after a job completed.

% gittrac #713
\item CCB now works with \MacroNI{TCP\_FORWARDING\_HOST}.  Previously,
  the reverse connection was made to the private address rather than
  to the host defined by \MacroNI{TCP\_FORWARDING\_HOST}.

% gittrac #852
\item Removed a bad optimization that caused some information about job
execution to be lost during job completion or removal,
if a history file was not configured.

% gittrac #893
\item Condor now checks whether the configuration variable
\MacroNI{GRIDFTP\_URL\_BASE} is set before
submitting cream grid jobs, as that variable is required for cream jobs
to function properly. If the variable is not set, cream jobs are put on
hold with an appropriate message.

% gittrac #920
\item Fixed a bug that allowed running virtual machines to be leaked
if the \Condor{startd} crashed.

% gittrac #912
\item Fixed a bug in \Prog{cream\_gahp} which could cause crashes when
there were more than 500 cream jobs queued.

% gittrac #972
\item Improved recovery when Condor crashes during the submission of a cream
grid job. Before, affected jobs would remain in \Expr{REGISTERED} state
on the cream server, but never run.

% gittrac #954
\item Improved the \Attr{HoldReason} message when cream grid jobs are
held by the \Condor{gridmanager}.

% gittrac #895
\item When naming a resource for a cream grid job, Condor now properly
recognizes the format used by the standard cream client UI:
\File{https://foo.edu:8443/cream-pbs-cream\_queue}.

% gittrac #795
% The memory leak is not worth documenting.
\item The configuration variable \MacroNI{SOAP\_SSL\_CA\_FILE} is now 
consulted in addition to
\MacroNI{SOAP\_SSL\_CA\_DIR} when authenticating
an https proxy for Amazon EC2, when \MacroNI{AMAZON\_HTTP\_PROXY} is defined.

% gittrac #485
\item Previously, if \Condor{rm} and friends were given both a constraint
and a user name or cluster id, they would act on all jobs matching the
constraint and all jobs associated with the user or cluster. Now, this
combination of arguments results in an error.

% gittrac #1062
\item Failure to purge a cream grid universe job from the remote server
because it was previously purged no longer results in the job being held.

% gittrac #1044
\item The \Condor{gridmanager} now recognizes VOMS attributes in X.509
proxies and will handle them appropriately. For example, it recognizes
that two proxies with the same identity but different VOMS attributes may
be mapped to different accounts on a remote machine.

% gittrac #947
% Not documenting, as the parameter being removed was added for a specifc
% customer and never documented.

% gittrac #932
% Not documenting, as the bug hasn't caused any problems.

% gittrac #1043
% Not documenting, as the problem never made it into a release.

% gittrac #979
\item Fixed a bug in \Condor{dagman}, introduced in 7.3.2, that will
cause \Condor{dagman} running on Windows to hang on any DAG using
more than one log file for the node jobs.

% gittrac #967
\item Fixed a bug in \Condor{dagman}, introduced in 7.3.2, that could
cause \Condor{dagman} to fail on a DAG using node job log files on
multiple devices, if log files on different devices happened to have
the same inode number.

% gittrac #981
\item Fixed a bug that caused the \Condor{schedd} daemon to segfault when
spooling more than 9 files.

% gittrac #1011
\item Fixed a bug that caused the \Condor{startd} daemon to crash on
Debian Stable.

% gittrac #1033
\item Fixed keyboard activity detection on the Windows XP platform.

% gittrac #1068
\item Fixed a bug in the \Condor{had} daemon that caused it to not start
the controlled daemon if CCB was enabled.

\end{itemize}

\noindent Known Bugs:

\begin{itemize}

% gittrac #1337
\item The \Condor{kbdd} has a chance of entering an infinite loop
on platforms that use X-Windows.  Microsoft Windows and Mac OS X
are not affected.  Removing KBDD from \MacroNI{DAEMON\_LIST} is a
workaround, although this impairs Condor's ability to detect
console usage.  This bug is fixed in Condor version 7.4.3.

% gittrac #983
\item \Condor{dagman} may fail on Windows if the set of node job log
file names includes multiple paths that are hard links (not symbolic links)
to the same file.

% gittrac #1081
\item \Condor{dagman} PRE and POST script arguments (and the names of
the scripts themselves) cannot contain spaces.

% gittrac #1082
\item \Condor{dagman} VARS values cannot contain single quotes.

\end{itemize}

\noindent Additions and Changes to the Manual:

\begin{itemize}

% gittrac #725
\item Added documentation about how to include spaces (and other
special characters) in \Condor{dagman} VARS values.

\end{itemize}


%%%%%%%%%%%%%%%%%%%%%%%%%%%%%%%%%%%%%%%%%%%%%%%%%%%%%%%%%%%%%%%%%%%%%%
\subsection*{\label{sec:New-7-4-0}Version 7.4.0}
%%%%%%%%%%%%%%%%%%%%%%%%%%%%%%%%%%%%%%%%%%%%%%%%%%%%%%%%%%%%%%%%%%%%%%

\noindent Release Notes:

\begin{itemize}

\item The default configuration file within the release now uses
  \MacroNI{ALLOW}/\MacroNI{DENY} in place of
  \MacroNI{HOSTALLOW}/\MacroNI{HOSTDENY} for security related settings.
  We recommend making this
  same change throughout all configuration files.  That way,
  a policy that depends on the default policy should continue to
  work as it did before.  The behavior of these configuration variables
  remains unchanged.  The \MacroNI{ALLOW}/\MacroNI{DENY} lists are
  added to the \MacroNI{HOSTALLOW}/\MacroNI{HOSTDENY} lists to form the
  authorization policy.  Both styles support the same syntax.  
  This change permits an anticipated
  phasing out of the \MacroNI{HOSTALLOW}/\MacroNI{HOSTDENY}  configuration
  variables, in order to simplify configuration.

\item As of Condor version 7.3.2, \Condor{q} \Opt{-xml} output no longer 
  begins with the non-XML consisting of two blank lines followed by a line
  of the following form:

\begin{verbatim}
-- Submitter: schedd-name : <IP> : hostname
\end{verbatim}

\item All \Prog{Stork} data placement is now supported by the Stork
project at the 
LSU Center for Computation and Technology
(\URL{http://www.cct.lsu.edu/www.cct.lsu.edu}).
Please see the home page of the Stork project at
\URL{http://www.cct.lsu.edu/~kosar/stork/index.php} for details and
software.

\end{itemize}


\noindent New Features:

\begin{itemize}

\item Condor is now integrated with the Hadoop Distributed File System (HDFS). 
See documentation in section~\ref{sec:Condor-HDFS} and 
section~\ref{sec:HDFS-Config-File-Entries}.

% commit af65de7ccc1a281c2b05b8f68ac70bcfa56b2dd1
\item \Condor{q} using the options \Opt{-analyze} and \Opt{-better-analyze}
  now provide analysis for scheduler and local universe jobs.
  Specifically, the \MacroNI{START\_SCHEDULER\_UNIVERSE} and
  \MacroNI{START\_LOCAL\_UNIVERSE} expressions are checked.

% #824
\item Added the ClassAd attributes
\Attr{TotalLocalJobsRunning}, \Attr{TotalLocalJobsIdle},
\Attr{TotalSchedulerJobsRunning}, and \Attr{TotalSchedulerJobsIdle}
to the published ClassAd for the \Condor{schedd}.  This means that
\Condor{q} \Opt{-analyze} can still give helpful information about
why local or scheduler universe jobs are idle when
the configuration variables \MacroNI{START\_LOCAL\_UNIVERSE} or
\MacroNI{START\_SCHEDULER\_UNIVERSE} refer to these attributes.
These attributes were already present internally within 
the \Condor{schedd} daemon, 
just not published.

% #688
\item The \Condor{vm-gahp} now supports KVM and links with libvirt, rather 
than calling virsh command-line tools.

% #760 #771 #769 #772 #773 #775
\item Greatly improved the \Condor{gridmanager}'s scalability when handling
many grid type gt2 grid universe jobs.  Improvements include more quickly
processing updated X.509 certificates, not checking jobs for status updates if 
they have not been submitted to the remote site, and eliminating unnecessary 
updates to the \Condor{schedd} daemon.

% commit 75f6b2fe920b88717712a0d41765d16665ad7fe6
\item Latency in the submission and cleaning up of Condor-C jobs
has been improved by changing the default value of
\Macro{C\_GAHP\_CONTACT\_SCHEDD\_DELAY} from 20 to 5.

% commit 8c2d88c695d6981be3bdab7e10c5d92e9f6bb55b
\item The \Expr{eval()} ClassAd function added in Condor version 7.3.2
is now also understood by the \Condor{job\_router} and
\Condor{q} using the \Opt{-better-analyze} option.

\item The submit command \SubmitCmd{run\_as\_owner} is now supported
for Unix platforms. Previously, it was only supported on Windows platforms.

% #795
\item When setting \MacroNI{AMAZON\_HTTP\_PROXY}, a username and password
can now be given as part of the proxy URL.
The value of \MacroNI{SOAP\_SSL\_CA\_DIR} is now consulted when authenticating
an https proxy for Amazon EC2, when \MacroNI{AMAZON\_HTTP\_PROXY} is defined.

% #694
\item The \Condor{collector} daemon now advertises to itself, and will appear
in the output of \Condor{status} \Opt{-collector}.

% #775, cf02764d9d0fdd2b36ef3629f862f856ee41a717, and more
\item Optimizations in core Condor systems should provide minor speed 
improvements.

% 823
\item Updated the maximum log size to the maximum operating system's file size.

\end{itemize}

\noindent Configuration Variable and ClassAd Attribute Additions and Changes:

\begin{itemize}

% commit 0e8800c201f81eac54cba925b3d7f6d81a83aeca
\item The undocumented configuration variable 
  \Macro{TOOLS\_PROVIDE\_OLD\_MESSAGES} is no longer recognized by Condor.

% #768
\item The new configuration variable 
  \Macro{SCHEDD\_JOB\_QUEUE\_LOG\_FLUSH\_DELAY} sets an
  upper bound in seconds on how long it takes for changes to the job
  ClassAd to be visible to the Condor Job Router and to Quill.
  The default value is 5 seconds.
  Previously, there was no upper limit.  Typically, other activity in
  the job queue, such as jobs being submitted or completed would cause
  buffered data to be flushed to disk, such that the effective upper bound was
  a function of how busy the job queue was.

% commit 55525e0a338be8b2ba2d9173ce832e43d05413c3
\item The default configuration file now uses
  \MacroNI{ALLOW}/\MacroNI{DENY} in place of
  \MacroNI{HOSTALLOW}/\MacroNI{HOSTDENY}.  See the release notes above
  for more information.

% commit 7199e217f9228082a8465b85aaee18c2ebb19787
\item The default value for \Macro{MAX\_JOBS\_RUNNING} has changed.
  Previously, it was 200.  Now it is defined by an expression that depends 
  on the total amount of memory and the operating system.  The default
  expression requires 1MByte of RAM per running job, on the submit machine.
  In some environments and configurations, this is overly
  generous and can be cut by as much as 50\%.  Under Windows, the
  number of running jobs is still capped at 200.
  A 64-bit version of Windows  is recommended in order to raise the value
  above the default.
  Under Unix, the maximum default is now 10,000.  To scale higher, we
  recommend that the system ephemeral port range is extended
  such that there are at least 2.1 ports per running job.

% #767 commit 18296bfdfa92f16684a73d8d57a54d231b48dc33
\item The default value of \MacroNI{RESERVED\_SWAP} has changed to
  the value 0, which
  disables the \Condor{schedd} daemon's check for sufficient swap space
  before starting more jobs.  The new expression defined with 
  \MacroNI{MAX\_JOBS\_RUNNING} has a more appropriate memory check, so
  the configuration variable \MacroNI{RESERVED\_SWAP} will no longer
  be used in the near future.
  For cases where 
  \MacroNI{RESERVED\_SWAP} is not set to 0, the default value
  of \MacroNI{SHADOW\_SIZE\_ESTIMATE} has changed to 800 Kbytes.
  Previously, it was 200 if not set,
  but it was set to 1800 in the example configuration file.

% #767 commit c80e8a40e67ef4faa4e2b32b3671877eae1e1a19
\item The default values of \Macro{START\_LOCAL\_UNIVERSE} and
  \Macro{START\_SCHEDULER\_UNIVERSE} have changed.  Previously,
  these were set to \Expr{True}.  Now, they are set using an expression
  that allows
  up to 200 local universe and 200 scheduler universe jobs to run.

% #767 commit c4f4d911a808e1bdb18552e1cdeb0407b6344969
\item The default value of
  \Macro{GRIDMANAGER\_MAX\_SUBMITTED\_JOBS\_PER\_RESOURCE} has
  changed from 100 to 1000.

% #767 commit 9e6dfa463c71c28c8dc2c0c0c215b51d6762e811
% commit b4fd08ad1a8c69da24c371565796ef73522a61fc
\item The default value of \Macro{NEGOTIATOR\_INTERVAL}
   has changed from 300 to 60.

% #767 commit 8b91877ec8186810887402e1dd1df07b6341ade7
% Probably at least one other commit
\item The default value of \Macro{ENABLE\_GRID\_MONITOR} has been
  changed from \Expr{False} to \Expr{True}.  This variable
  was assigned to \Expr{True} in the example configuration file, so
  the change in default value now matches the value set in the example
  configuration.

% #631
\item The configuration variable \MacroNI{VM\_VERSION} has been removed,
as has the machine ClassAd attribute of the same name.
When the virtual machine version information is needed in the machine ClassAd,
the configuration variable \MacroNI{STARTD\_ATTRS} can be used to
add it.
 
% #861
\item The default configuration now uses
  \MacroNI{VM\_BRIDGE\_SCRIPT} and \MacroNI{VM\_SCRIPT} in place of
  \MacroNI{XEN\_BRIDGE\_SCRIPT} and \MacroNI{XEN\_SCRIPT} due to the
  support of KVM. 
  Submit description file commands have also been added, and they include:
  \SubmitCmd{kvm\_disk}, \SubmitCmd{kvm\_transfer\_files},
   and \SubmitCmd{kvm\_cd\_rom\_device}.

% #872
\item The configuration variables \MacroNI{XEN\_DEFAULT\_KERNEL}
  and \MacroNI{XEN\_DEFAULT\_INITRD} have been removed.
  Corresponding to this, the submit description file command
  \Expr{xen\_kernel = any} is no longer valid.

\end{itemize}

\noindent Bugs Fixed:

\begin{itemize}

\item Fixed a bug that prevented parallel universe jobs from running 
  on \Condor{startd} daemons with dynamic slots enabled.

% #706
\item Fixed a race condition bug in the \Condor{startd} which allowed
it to send Unix signals, intended for \Condor{starter} processes, as
root to non-Condor related processes.

% 735
\item A Windows platform bug has been fixed.
The bug caused a 20-second interval in which
the \Condor{shadow}, \Condor{startd}, and \Condor{starter} daemons
appeared as deadlocked. 
The bug was visible if a job ClassAd update from the \Condor{starter} caused
the job's periodic hold or remove policy to become \Expr{True}.

%gittrac #622
\item Fixed a bug that could cause \Condor{dagman} to generate an
illegal rescue DAG, if it read events incorrectly in recovery mode.
\Condor{dagman} now checks for events that violate DAG semantics
when reading events in recovery mode, and it exits without creating a
rescue DAG if it reads such an event.

% gittrac #744
\item Fixed a bug that could cause \Condor{dagman} to abort if it saw
the combination of a terminated event and an aborted event on a node with
retries.

% commit 5039a08cf00b0d0fafcc3fd8241518d1854ec3a1
\item Changed some logged warnings in \Condor{dagman} to not be
printed at the default verbosity setting.

% gittrac #825
\item The version compatibility checking between a \File{.condor.sub}
file and the \Condor{dagman} binary which is done at DAG startup
is now much more permissive.
Currently, \File{.condor.sub} files with
Condor versions of 7.1.2 and later accepted by \Condor{dagman}.

% gittrac #851
\item Fixed a bug introduced with the new \Condor{dagman} lazy log file
evaluation code in Condor version 7.3.2.
The bug sometimes caused failure when running rescue DAGs.

% #211 commit d6c0144739000523e94205a192be3cf9afe9ca9f
\item Fixed a bug originating in Condor version 7.1.4.
When a user submitted a job
with an executable that did not have execute permission enabled,
Condor was running as root, and file transfer was specified in the job,
Condor failed to automatically turn on execute permission after
transferring the file.

% commit 3bb847691bfda4f26d2f570bed1a412fb3afb439
\item Fixed a bug that appeared in Condor version 7.3.2.
The configuration variable
\MacroNI{COUNT\_HYPERTHREAD\_CPUS} was ignored and was effectively
treated as \Expr{False} in all cases.

% #761
\item Fixed a bug in which the Condor Job Router was not able
to see matchmaking diagnostic attributes such as \Attr{LastRejMatchTime}.
Therefore, when evaluating policy
expressions that referred to these attributes, they were effectively
treated as though \Expr{Undefined}.
Quill was also not able to see these attributes.

% #822
\item Fixed a bug introduced in Condor version 7.3.2 that could cause the
\Condor{gridmanager} to crash repeatedly on startup,
if the job queue
contained grid type gt2 jobs that had been previously submitted.

% #724, #774, #786
\item Fixed two bugs introduced in Condor version 7.3.2,
and related to VOMS. 
The first bug
prevented jobs with X.509 proxies from being submitted on platforms
on which Condor does not support VOMS.
The second bug prevented submission
of jobs with VOMS proxies, if the authenticity of the VOMS extensions
could not be verified.
At the same time, improved memory usage when VOMS extensions are not used.

\item Fixed a bad default in the file \File{batch\_gahp.config},
that prevented
Condor from observing job state changes for grid universe jobs
with a grid type of pbs or lsf.

% #748
\item Fixed a bug that caused Condor-C jobs to fail if
the submit description file command \SubmitCmd{transfer\_executable}
was set to \Expr{False}.

% #784
\item Fixed a bug that caused Condor-C jobs to fail if the executable
or one of the \File{stdin}, \File{stdout}, or \File{stderr} file names
contained a comma.

% #460
\item File transfer for grid type gt4 jobs requires an empty directory
within \File{/tmp}, which the \Condor{gridmanager} creates. 
If this directory is deleted, the \Condor{gridmanager} will now recreate it.

%gittrac #790
\item Fixed a bug that could cause the user job log to become
  corrupted on Windows platforms.  This bug would manifest itself only if the
  same log file was specified with different paths.  For example, the
  following submit file could have triggered this bug:
\begin{verbatim}
...
initialdir = /data/job1
log = ../JobLog
queue

initialdir = /data/job2
log = ../JobLog
queue
\end{verbatim}


% commit a26fcd9fe54cd3920fe777d5d8e0b2ffefbc3b1f
\item Fixed a memory leak introduced into Condor version 7.3.2.
The leak was in the \Condor{collector} daemon.

% commit 1663b7e183e6bf1df8152af98d9387412c2ae146
\item Fixed a bug introduced in Condor version 7.3.2
that resulted in the \Condor{negotiator} daemon
refusing to run, if the configuration variable \MacroNI{GROUP\_QUOTA}
for any group was set to 0.

% gittrac #731
\item Fixed a bug that caused the \Code{ctime} in the event log header
  to always be zero.

% #862 commit 9a432e2f3497e5dce120db5c733e79212257f6a5
\item Fixed the output of \Condor{status} when used with the command-line
  options \Opt{-java} or \Opt{-vm}.

\item Fixed a problem in the \Condor{schedd} daemon introduced in
  7.3.2.  For \Condor{schedd} daemons with lots of jobs having periodic release
  expressions, this bug could result in the \Condor{schedd} taking a long
  time while evaluating periodic expressions, causing it to become
  unresponsive to queries and other tasks.
  With a job queue of 30,000 jobs,
  a period of unresponsiveness of an hour was observed,
  whereas the evaluation of periodic expressions in this same environment
  normally takes less than 5 seconds.

\item Potential bugs and memory leaks were identified and 
fixed throughout Condor.  The Condor Team is not aware of anyone having 
encountered these bugs.

% #692 commit 8bc6bb4e06f11b2fdca28214d98c68c34c0ab9a4
\item The \Condor{starter} cleans up working directories in more
situations.  Previously during some error conditions, the working
directory within \MacroUNI{EXECUTE} might be left behind.

% #692 commit 8bc6bb4e06f11b2fdca28214d98c68c34c0ab9a4
\item If the user log cannot be accessed when a local universe
job starts, the job would fail and immediately be retried.  Now
the job is placed on hold.

% 826 
\item Fixed a bug in the \Condor{startd} in which vacating jobs would not 
respect the value of \Attr{JobLeaseDuration}.

% 802
\item Updated the detection of \Attr{HasVM} within the \Condor{startd}
 to publish an update to the \Condor{collector},
 when the configuration variable \MacroNI{VM\_RECHECK\_INTERVAL} is specified.

% commit 68f06088fa36eb0eb332a4f72a5c48ccd48b1d5a
\item Fixed a bug in which the \Condor{gridmanager} could, in rare cases,
waste a
small amount of memory and processor time checking for proxy files no longer
being used by any active jobs.

% commit bc66aa432e1f4e69d88a5b769204a4fce0648bfc
\item The setting \Macro{CREAM\_GAHP} was added to the default configuration 
file with a value of \File{\$(SBIN)/cream\_gahp}.
Existing installations desiring to 
submit jobs to CREAM should add this setting.

% #702
\item Fixed a bug where \Condor{restart} would fail on a \Condor{collector}
daemon configured for high availability with multiple \Condor{collector}
daemons.

% commit f44a68fb351e528ea5b251dd2c3cf9767b0c1fba
\item Fixed a bug in which multiple entries of output from 
the command
\Condor{status} \Opt{-negotiator}
would be on a single line.  They are now listed one per line.

% #778
\item Fixed a bug in which the command
\Condor{submit} \Opt{-dump} would crash if multiple
jobs were queued from within a single submit file.

% #742
\item Fixed a bug in which a slot whose associated job disappeared
could remain in the Claimed/Idle state until the claim lease expired.
The slot should now promptly return to the Unclaimed/Idle state.

% commit 0d5e3ad8fc85f0cd0dc58f73b503c76c0ad49bc4
\item Fixed a bug in which a \Condor{startd} using dynamic slots could
crash on shutdown or reconfiguration.



\end{itemize}

\noindent Known Bugs:

\begin{itemize}

% gittrac #1337
\item The \Condor{kbdd} has a chance of entering an infinite loop
on platforms that use X-Windows.  Microsoft Windows and Mac OS X
are not affected.  Removing KBDD from \MacroNI{DAEMON\_LIST} is a
workaround, although this impairs Condor's ability to detect
console usage.  This bug is fixed in Condor version 7.4.3.

% gittrac #161, #935, #936, #1020
\item There are multiple bugs related to using VOMS attributes.
In Condor version 7.4.0, VOMS support should be disabled by setting
the configuration variable \Expr{USE\_VOMS\_ATTRIBUTES = FALSE}.

\item A configuration variable of  \Macro{USE\_VISIBLE\_DESKTOP} set 
to \Expr{True} will corrupt the visible desktop.
  This bug is present back through Condor version 7.2.4.
This configuration variable did not work at all in 7.2 releases
prior to 7.2.4.  This bug will be fixed in Condor version 7.4.1.

% gittrac #934
\item If the global event log (see section~\ref{param:EventLog}) is
turned on, \Condor{dagman} will hang when running any DAG that has
POST scripts.

% gittrac #979
\item \Condor{dagman} will hang on Windows when running any DAG that
uses more than one log file for the node jobs.

\end{itemize}

\noindent Additions and Changes to the Manual:

\begin{itemize}

\item See section~\ref{sec:Condor-HDFS} and 
section~\ref{sec:HDFS-Config-File-Entries} for preliminary documentation of
Condor's integration with the Hadoop Distributed File System (HDFS). 

\end{itemize}


% as of April 2011, Karen no longer wants to include these older
% version histories with the 7.6 and beyond manuals.
%\input{version-history/7-3.history.tex}
%\input{version-history/7-2.history.tex}
%\input{version-history/7-1.history.tex}
%\input{version-history/7-0.history.tex}
% Oct 2009, as we release 7.4, Karen commented out inclusion of the
% 6.9 and 6.8 histories
%\input{version-history/6-9.history.tex}
%\input{version-history/6-8.history.tex}
% Dec 2007, as we release 7.x, Karen commented out the older histories
%\input{version-history/6-7.history.tex}
%\input{version-history/6-6.history.tex}
% Feb 2007 -- still in the manual source, just not incorporating
% these old histories into the finished product, thereby
% reducing the size of the manual by 200 pages
%\input{version-history/6-5.history.tex}
%\input{version-history/6-4.history.tex}
%\input{version-history/6-3.history.tex}
%\input{version-history/6-2.history.tex}
%\input{version-history/6-1.history.tex}
%\input{version-history/6-0.history.tex}
