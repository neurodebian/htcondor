\begin{ManPage}{\label{man-condor-configure}\Condor{configure}}{1}
{Configure or install HTCondor}

\index{HTCondor commands!condor\_configure}
\index{HTCondor commands!condor\_install}
\index{condor\_configure command}
\index{condor\_install command}

\Synopsis \SynProg{\Condor{configure} or \Condor{install}}
[\verb@--@\textbf{help}]
[\verb@--@\textbf{usage}]

\SynProg{\Condor{configure}} or \SynProg{\Condor{install}}
[\verb@--@\textbf{install[=$<$path/to/release$>$] }]
[\verb@--@\textbf{install-dir=$<$path$>$}]
[\verb@--@\textbf{prefix=$<$path$>$}]
[\verb@--@\textbf{local-dir=$<$path$>$}]
[\verb@--@\textbf{make-personal-condor}]
[\verb@--@\textbf{bosco}]
[\verb@--@\textbf{type = $<$ submit, execute, manager $>$}]
[\verb@--@\textbf{central-manager = $<$ hostname$>$}]
[\verb@--@\textbf{owner = $<$ ownername $>$}]
[\verb@--@\textbf{maybe-daemon-owner}]
[\verb@--@\textbf{install-log = $<$ file $>$}]
[\verb@--@\textbf{overwrite}]
[\verb@--@\textbf{ignore-missing-libs}]
[\verb@--@\textbf{force}]
[\verb@--@\textbf{no-env-scripts}]
[\verb@--@\textbf{env-scripts-dir = $<$ directory $>$}]
[\verb@--@\textbf{backup}]
[\verb@--@\textbf{credd}]
[\verb@--@\textbf{verbose}]

% 2 hyphens in a row, without any spaces inbetween poses a problem
% for LaTeX.  Things tried, without success.
%   --      results in a single hyphen
%   - -     results in a 2 hyphens, with too much space inbetween
%   -\--    results in a 2 hyphens, with too much space inbetween
%  $--$     results in 2 long, very high up dashes, with nice spacing
%  $-$$-$   results in 2 long, very high up dashes, with too much spacing

% Alain's fix
%   {\tt--}          makes one nice dash in html
%                    makes 2 nice dashes in postscript
%   {\tt--}{\tt--}   makes 2 nice dashes that collide with following letter
%                       in html
%                    makes 4  nice dashes in postscript
%   {\small{\tt--}}{\small{\tt--}}  2 nice dashes in html
%                                   4 nice dashes in postscript

% Peter's fix
%   ---            makes 2 dashes exactly as required in the html
%                  makes 1 very long dash in postscript

% Karen's final fix:  do everything manually (no LaTeX macros),
%  and use the verbatim.

\Description 

\Condor{configure} and \Condor{install} refer to a single script that installs
and/or configures HTCondor on Unix machines.  As the names imply,
\Condor{install} is intended to perform a HTCondor installation, and
\Condor{configure} is intended to configure (or reconfigure) an
existing installation.
Both will run with Perl 5.6.0 or more recent versions.

\Condor{configure} (and \Condor{install}) are designed to be run more
than one time where required.
It can install HTCondor when invoked with a correct configuration via
\begin{verbatim}
condor_install
\end{verbatim}
or 
\begin{verbatim}
condor_configure --install
\end{verbatim}
or, it can change the configuration files when invoked via
\begin{verbatim}
condor_configure
\end{verbatim}
Note that changes in the configuration files do not result
in changes while HTCondor is running.
To effect changes while HTCondor is running,
it is necessary to further use the \Condor{reconfig} or \Condor{restart}
command.
\Condor{reconfig}  is required where the currently executing
daemons need to be informed of configuration changes.
\Condor{restart} is required where the options
\verb@--@\textbf{make-personal-condor} or
\verb@--@\textbf{type}
are used, since these affect which daemons are running.

Running \Condor{configure} or \Condor{install} with no options results
in a usage screen being printed.
The \verb@--@\textbf{help} option can be used to display a full help screen.

Within the options given below, 
the phrase \Term{release directories} is the list of directories that are
released with HTCondor.  This list includes: 
\File{bin}, \File{etc}, \File{examples}, \File{include},
\File{lib}, \File{libexec}, \File{man}, \File{sbin},
\File{sql} and \File{src}.

\begin{Options}
  \OptItem{\Opt{---help}} {Print help screen and exit}

  \OptItem{\Opt{---usage}} {Print short usage and exit}

  \OptItem{\Opt{---install[=$<$path/to/release$>$]}} {Perform installation, 
    assuming that
    the current working directory contains the release directory,
    if the optional \Expr{=<path/to/release>} is not specified.
    Without further options, the configuration is that of
    a Personal HTCondor, a complete one-machine pool.
    If used as an 
    upgrade within an existing installation directory, existing 
    configuration files and local directory are preserved.  This
    is the default behavior of \Condor{install}. }

  \OptItem{\Opt{---install-dir=$<$path$>$}} {Specifies the path
    where HTCondor should be installed or the path where it already is
    installed. The default is the current working directory.}

  \OptItem{\Opt{---prefix=$<$path$>$}} {This is an alias for
    \Opt{--install-dir}.}

  \OptItem{\Opt{---local-dir=$<$path$>$}} {Specifies the
    location of the local directory, which is the directory that generally 
    contains the local (machine-specific) configuration file as well as the
    directories where HTCondor daemons write their run-time information 
    (\File{spool}, \File{log}, \File{execute}).
    This location is indicated  by the \MacroNI{LOCAL\_DIR} 
    variable in the configuration file. 
    When installing (that is, if \Opt{--install} is specified),
    \Condor{configure} 
    will properly create the local directory in the location specified.
    If none is specified, the default value is given by the evaluation of
    \File{\$(RELEASE\_DIR)/local.\$(HOSTNAME)}.

    During subsequent invocations of \Condor{configure}
    (that is, without the ---install option),
    if the ---local-dir option is specified, the new directory
    will be created and the \File{log}, \File{spool} and \File{execute} 
    directories will be moved there from their current location.}

  \OptItem{\Opt{---make-personal-condor}} {Installs and configures for 
     Personal HTCondor, a fully-functional, one-machine pool.}

  \OptItem{\Opt{---bosco}} {Installs and configures Bosco, a personal
     HTCondor that submits jobs to remote batch systems.}

  \OptItem{\Opt{---type= $<$ submit, execute, manager $>$}} {One
    or more of the types may be listed.
    This determines the roles that a machine may play in a pool.
    In general, any machine can be a submit and/or execute machine,
    and there is one central manager per pool.
    In the case of a Personal HTCondor,
    the machine fulfills all three of these roles.}

  \OptItem {\Opt{---central-manager=$<$hostname$>$}} {Instructs
    the current HTCondor installation to use the specified machine
    as the central manager. 
    This modifies the configuration variable \MacroNI{COLLECTOR\_HOST}
    to point to the given host name.
    The central manager machine's HTCondor configuration needs
    to be independently configured to 
    act as a manager using the option \Opt{--type=manager}. }

  \OptItem{\Opt{---owner=$<$ownername$>$}} {Set configuration
    such that HTCondor daemons will be executed as the given owner.
    This modifies the 
    ownership on the \File{log}, \File{spool} and \File{execute}
    directories and sets the \MacroNI{CONDOR\_IDS} value
    in the configuration file,
    to ensure that HTCondor daemons start up as the specified effective user.
    The section on security within the HTCondor manual discusses
    UIDs in HTCondor.
    This is only applicable when \Condor{configure} is run by root.
    If not run as root, the owner is the user running
    the \Condor{configure} command.  }

  \OptItem{\Opt{---maybe-daemon-owner}} {If \Opt{--owner} is not specified and
    no appropriate user can be found to run Condor, then this option
    will allow the daemon user to be selected. This option is rarely
    needed by users but can be useful for scripts that invoke
    condor\_configure to install Condor.}

  \OptItem{\Opt{---install-log=$<$file$>$}} {Save information about the
    installation in the specified file. This is normally only needed
    when condor\_configure is called by a higher-level script, not when
    invoked by a person.}

  \OptItem{\Opt{---overwrite}} {
    Always overwrite the contents of the \File{sbin} directory in
    the installation directory.  By default, \Condor{install}
    will not install if it finds an existing \File{sbin} directory
    with HTCondor programs in it.  In this case, \Condor{install}
    will exit with an error message.  Specify
    \Opt{--overwrite} or \Opt{--backup} to tell \Condor{install}
    what to do.

    This prevents \Condor{install} from moving an \File{sbin}
    directory out of the way that it should not move.  This is
    particularly useful when trying to install HTCondor in a
    location used by other things (\File{/usr}, \File{/usr/local}, etc.)
    For example: \Condor{install} \Opt{--prefix=/usr}
         will not move
    \File{/usr/sbin} out of the way unless you specify the
    \Opt{--backup} option.

    The \Opt{--backup} behavior is used to
    prevent \Condor{install} from overwriting running daemons --
    Unix semantics will keep the existing binaries running, even
    if they have been moved to a new directory.}

  \OptItem{\Opt{---backup}} {
    Always backup the \File{sbin} directory in the installation
    directory.  By default, \Condor{install} will not install if
    it finds an existing \File{sbin} directory with HTCondor programs
    in it.  In this case, \Condor{install} with exit with an
    error message.  You must specify \Opt{--overwrite} or
    \Opt{--backup} to tell \Condor{install} what to do.

    This prevents \Condor{install} from moving an \File{sbin}
    directory out of the way that it should not move.  This is
    particularly useful if you're trying to install HTCondor in a
    location used by other things (\File{/usr}, \File{/usr/local}, etc.)
    For example: \Condor{install} \Opt{--prefix=/usr}
         will not move
    \File{/usr/sbin} out of the way unless you specify the
    \Opt{--backup} option.

    The \Opt{--backup} behavior is used to
    prevent \Condor{install} from overwriting running daemons --
    Unix semantics will keep the existing binaries running, even
    if they have been moved to a new directory.}

  \OptItem{\Opt{---ignore-missing-libs}} {
    Ignore missing shared libraries that are detected by
    \Condor{install}.  By default, \Condor{install} will detect
    missing shared libraries such as \File{libstdc++.so.5} on
    Linux; it will print messages and exit if missing libraries
    are detected.  The \Opt{---ignore-missing-libs} will cause
    \Condor{install} to not exit, and to proceed with the
    installation if missing libraries are detected.  }

  \OptItem{\Opt{---force}} {
    This is equivalent to enabling both the \Opt{---overwrite} and
    \Opt{---ignore-missing-libs} command line options.  }

  \OptItem{\Opt{---no-env-scripts}} {
    By default, \Condor{configure} writes simple sh and csh
    shell scripts which can be sourced by their respective
    shells to set the user's \Env{PATH} and \Env{CONDOR\_CONFIG}
    environment
    variables.  This option prevents \Condor{configure} from
    generating these scripts.  }

  \OptItem{\Opt{---env-scripts-dir=$<$directory$>$}} {
    By default, the simple \Prog{sh} and \Prog{csh} shell scripts (see
    \Opt{---no-env-scripts} for details) are created in 
    the root directory of the HTCondor installation.  This option
    causes \Condor{configure} to generate these scripts in the
    specified directory.  }

  \OptItem{\Opt{---credd}} {Configure the
    the \Condor{credd} daemon (credential manager daemon).}

  \OptItem{\Opt{---verbose}} {Print information about changes
    to configuration variables as they occur.}
\end{Options}

\ExitStatus

\Condor{configure} will exit with a status value of 0 (zero) upon success,
and it will exit with a nonzero value upon failure.

\Examples
Install HTCondor on the machine (machine1@cs.wisc.edu)
to be the pool's central manager.
On machine1,
within the directory that contains the unzipped HTCondor
distribution directories:
\footnotesize
\begin{verbatim}
% condor_install --type=submit,execute,manager
\end{verbatim}
\normalsize
This will allow the machine to submit and execute HTCondor jobs, 
in addition to being the central manager of the pool.


To change the configuration such that
machine2@cs.wisc.edu is an execute-only machine
(that is, a dedicated computing node)
within a pool with central manager on machine1@cs.wisc.edu,
issue the command on that machine2@cs.wisc.edu
from within the directory where HTCondor is installed:
\footnotesize
\begin{verbatim}
% condor_configure --central-manager=machine1@cs.wisc.edu --type=execute
\end{verbatim}
\normalsize



To change the location of the \MacroNI{LOCAL\_DIR} directory
in the configuration file, do (from the directory where HTCondor is installed):
\footnotesize
\begin{verbatim}
% condor_configure --local-dir=/path/to/new/local/directory
\end{verbatim}
\normalsize
This will move the \File{log},\File{spool},\File{execute} directories
to \File{/path/to/new/local/directory} from the current local directory.



\end{ManPage}
