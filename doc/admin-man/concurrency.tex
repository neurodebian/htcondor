%%%%%%%%%%%%%%%%%%%%%%%%%%%%%%%%%%%%%%%%%%%%%%%%%%%%%%%%%%%%%%%%%%%%%%%%%%%
\subsection{\label{sec:Concurrency-Limits}Concurrency Limits} 
%%%%%%%%%%%%%%%%%%%%%%%%%%%%%%%%%%%%%%%%%%%%%%%%%%%%%%%%%%%%%%%%%%%%%%%%%%%
\index{concurrency limits}

\Term{Concurrency limits} allow an administrator to limit the number
of concurrently running jobs that declare that they use some
pool-wide resource.  This limit is applied globally to all jobs
submitted from all schedulers across one HTCondor pool.  This is
useful in the case of a shared resource, like an NFS or database
server that some jobs use, and the administrator needs to limit the
number of jobs accessing the server.

The administrator must predefine the names and capacities of the
resources to be limited in the negotiator's configuration file.
The job submitter must declare in the job submit file which 
resources the job consumes.

For example, assume that there are 3 licenses for the X software,
so HTCondor should constrain the number of running jobs which need the
X software to 3.  Pick the name XSW as the name of the resource:

\begin{verbatim}
XSW_LIMIT = 3
\end{verbatim}
where \AdStr{XSW} is the invented name of this resource,
is appended with the string \MacroNI{\_LIMIT}.
With this limit, a maximum of 3 jobs declaring that they need this
resource may be executed concurrently.

In addition to named limits, such as in the example named limit \MacroNI{XSW},
configuration may specify a concurrency limit for all resources
that are not covered by specifically-named limits.
The configuration variable \Macro{CONCURRENCY\_LIMIT\_DEFAULT} sets
this value.  For example,
\begin{verbatim}
CONCURRENCY_LIMIT_DEFAULT = 1
\end{verbatim}
will enforce a limit of at most 1 running job that declares a
usage of an unnamed resource.  If
\MacroNI{CONCURRENCY\_LIMIT\_DEFAULT} is omitted from the
configuration, then no limits are placed on the number of concurrently
executing jobs of resources for which there is no specifically named
concurrency limits.

The job must declare its need for a resource by placing a command
in its submit description file or adding an attribute to the
job ClassAd.
In the submit description file, an example job that requires
the X software adds:
\begin{verbatim}
concurrency_limits = XSW
\end{verbatim}
This results in the job ClassAd attribute
\begin{verbatim}
ConcurrencyLimits = "XSW"
\end{verbatim}

Jobs may declare that they need more than one type of resource.
In this case, specify a comma-separated list of resources:

\begin{verbatim}
concurrency_limits = XSW, DATABASE, FILESERVER
\end{verbatim}

The units of these limits are arbitrary, by default each job consumes
one unit of resource.  Jobs can declare that they use more than one
unit, by following the resource name by a colon character and the
integer number of resources.  For example, if the above job
uses three times as many fileserver resources as a baseline job,
this could be declared as follows:

\begin{verbatim}
concurrency_limits = XSW, DATABASE, FILESERVER:3
\end{verbatim}

If there are many different types of resources which all have the same
default capacity, it may be tedious to define them all individually.
A shortcut to defining sets of consumable resources exists, by using a
common prefix.  After the common prefix, a period follows, then the
rest of the resource name.

\begin{verbatim}
CONCURRENCY_LIMIT_DEFAULT = 5
CONCURRENCY_LIMIT_DEFAULT_LARGE = 100
CONCURRENCY_LIMIT_DEFAULT_SMALL = 25
\end{verbatim}

The above configuration defines the prefixes named large and small,
along with the default limit of 5.  With this configuration a
concurrency limit named \Expr{"large.swlicense"} will have a default
limit of 100.  A concurrency limit named \Expr{"large.dbsession"} will
also have a limit of 100, as will any resource that begins with
\Expr{"large."}  A limit named \Expr{"small.dbsession"} will receive
the default limit of 25.  A concurrency limit named
\Expr{"other.license"} will receive the global default limit of 5, as
there is no value set for for
\MacroNI{CONCURRENCY\_LIMIT\_DEFAULT\_OTHER}.

Note that it is possible, in unusual circumstances, for more jobs to
be started than should be allowed by the concurrency limits feature.
In the presence of preemption and dropped updates from the
\Condor{startd} daemon to the \Condor{collector} daemon, it is
possible for the limit to be exceeded. If the limits are exceeded,
HTCondor will not kill any job to reduce the number of running jobs to
meet the limit.
